%%%%%%%%%%%%%%%%%%%%%%%%%%%%%%%%%%%%%%%%%%%%%%%%%%%%%%%%%%%%%%%%%%%%%%%%%%%%%%%%
% ABSTRACT


\begin{abstract}

%% neste arquivo abstract.tex
%% o texto do resumo e as palavras-chave têm que ser em Inglês para os documentos escritos em Português
%% o texto do resumo e as palavras-chave têm que ser em Português para os documentos escritos em Inglês
%% os nomes dos comandos \begin{abstract}, \end{abstract}, \keywords e \palavrachave não devem ser alterados

\selectlanguage{english}	%% para os documentos escritos em Português
%\selectlanguage{portuguese}	%% para os documentos escritos em Inglês

\hypertarget{estilo:abstract}{} %% uso para este Guia

The ionosphere is a layer of gas in the state of plasma that was ionized mainly by the effect of solar radiation. Ionospheric plasma distribution is not uniform in space and time, being  the generation of plasma irregularities triggered by the day-to-night transition. One of them is que Equatorial Ionization Anomaly, which coupled  with the plasma instability mechanism cause depletions, i.e. regions with low density of ions and electrons. Such structures are known as ionospheric bubbles and are generated at the magnetic equator just after sunset. They ascent to higher altitudes and migrate to low latitudes along the Earth magnetic field. RF signals are affected by the ionospheric bubbles. Ionospheric scintillation is the occurrence of perturbation of RF signals due to ionospheric irregularities, causing signal phase and amplitude fluctuations. This proposal addresses the use of knowledge discovery in databases in order to predict ionospheric scintillation in the Brazilian territory, in particular, in São José dos Campos. It is intended to employ historical data of ionospheric scintillation and other data including solar activity level, plasma vertical drift velocity and global magnetic activity. The algorithm used for the prediction, either formulated as a classification or a regression problem, is the Extreme Gradient Boosting (XGBoost), available in the Python programming environment, and preliminary results are presented here.

\keywords{%
	\palavrachave{Ionospheric Scintillation}%
	\palavrachave{Plasma Bubble}%
%	\palavrachave{VTEC}%
	\palavrachave{Data Mining}%
%	\palavrachave{Vale do Paraíba}
}

\selectlanguage{portuguese}	%% para os documentos escritos em Português
%\selectlanguage{english}	%% para os documentos escritos em Inglês

\end{abstract}
