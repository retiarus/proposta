%%%%%%%%%%%%%%%%%%%%%%%%%%%%%%%%%%%%%%%%%%%%%%%%%%%%%%%%%%%%%%%%%%%%%%%%%%%%%%%%
% ABSTRACT


\begin{abstract}

%% neste arquivo abstract.tex
%% o texto do resumo e as palavras-chave têm que ser em Inglês para os documentos escritos em Português
%% o texto do resumo e as palavras-chave têm que ser em Português para os documentos escritos em Inglês
%% os nomes dos comandos \begin{abstract}, \end{abstract}, \keywords e \palavrachave não devem ser alterados

\selectlanguage{english}	%% para os documentos escritos em Português
%\selectlanguage{portuguese}	%% para os documentos escritos em Inglês

\hypertarget{estilo:abstract}{} %% uso para este Guia

The ionosphere is a atmosphere layer which extends from about 60 Km to 1,000 Km altitude. This layer influences in the radio frequency signals transmitted by satellites to the terrestrial surface. It is composed of ionized gases and free electrons generated by the solar radiation. Variations in the solar flux generates electrical and magnetic field fluctuations in the space, and consequently, in the magnetic field of Earth causing perturbations in the ionization, and therefore, in the quantity of free electrons in the ionosphere, changing the transmission of the radio frequency signals. Among the various ionospheric disturbances, there is the equatorial anomaly, which consists in the formation of a high density electrons region between 15 and 20 degrees north and south of the magnetic equator. This anomaly is more significant in Brazil, particularly in the Vale do Paraíba, in the state of São Paulo. The radio frequency signals from Global Navigation Satellite System (GNSS) are affected by ionization fluctuations, which constitute the ionospheric scintillation phenomenon, that can be measured by the S4 indices. Scintillation affects GNSS signals, especially at the peak of the anomaly, affecting air navigation and other human activities that rely on signals received from satellites. On the other hand, the amount of free electrons in the ionosphere can be measured by the vertical total electronic content (VTEC), and regions with low VTEC values in relation to their neighborhood, characterize the so called ionospheric bubbles associated with scintillation. Given the existence of measuring stations that provided values of S4 and VTEC, this works seeks to correlate the values of these variables, as well as to analyze their spatial-temporal evolution through mining techniques and data visualization, considering as a case study the city of São José dos Campos. 

\keywords{%
	\palavrachave{Ionospheric Scintillation}%
	\palavrachave{S4 Index}%
	\palavrachave{Ionospheric Bubble}%
	\palavrachave{VTEC}%
	\palavrachave{Data Mining}%
	\palavrachave{GNSS}
	\palavrachave{Vale do Paraíba}
}

\selectlanguage{portuguese}	%% para os documentos escritos em Português
%\selectlanguage{english}	%% para os documentos escritos em Inglês

\end{abstract}
