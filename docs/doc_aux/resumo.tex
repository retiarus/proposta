%%%%%%%%%%%%%%%%%%%%%%%%%%%%%%%%%%%%%%%%%%%%%%%%%%%%%%%%%%%%%%%%%%%%%%%%%%%%%%%%
% RESUMO %% obrigatório

\begin{resumo}

%% neste arquivo resumo.tex
%% o texto do resumo e as palavras-chave têm que ser em Português para os documentos escritos em Português
%% o texto do resumo e as palavras-chave têm que ser em Inglês para os documentos escritos em Inglês
%% os nomes dos comandos \begin{resumo}, \end{resumo}, \palavraschave e \palavrachave não devem ser alterados

\hypertarget{estilo:resumo}{} %% uso para este Guia

A ionosfera é composta basicamente por uma camada de gás em estado de plasma cujo  processo de ionização tem a radiação solar como seu principal agente. A distribuição do plasma ionosférico não é uniforme no espaço e do tempo, sendo a transição entre dia e noite  um elemento importante na geração de irregularidades.  Uma delas é a Anomalia de Ionização Equatorial, a qual acoplada ao mecanismo de instabilidade do plasma ocasiona a formação de depleções, isto é, regiões com baixa densidade de íons e elétrons. Essas estruturas são conhecidas como bolhas ionosféricas, e são geradas no equador magnético após o pôr do sol para em seguida ascender a altitudes maiores e migrar para baixas latitudes ao longo do campo magnético da Terra. Os sinais de radiofrequência dos sistemas de navegação global por satélites são afetados pelas bolhas ionosféricas. Denomina-se cintilação ionosférica à ocorrência de pertubações nos sinais de radiofrequência decorrentes de irregularidades na ionosfera, que ocasionam flutuações de amplitude e fase do sinal. Esta proposta aborda o uso de técnicas de descoberta de conhecimento em bases de dados para predição de cintilação ionosférica no território brasileiro, especialmente em São José dos Campos. Pretende-se utilizar dados históricos de cintilação ionosférica e outros tais como nível de atividade solar, velocidade de deriva vertical do plasma e atividade magnética global. O algoritmo utilizado para a predição, formulada como um problema de classificação ou de regressão, é o Extreme Gradient Boosting (XGBoost), disponível no ambiente de programação Python, sendo apresentados resultados preliminares.

\palavraschave{%
  \palavrachave{Cintilação Ionosférica}%
  \palavrachave{Bolha Plasma}%
  \palavrachave{Índice S4}%
%  \palavrachave{VTEC}%
  \palavrachave{Mineração de dados}%
  \palavrachave{GNSS}
%  \palavrachave{Vale do Paraíba}
}
 
\end{resumo}
