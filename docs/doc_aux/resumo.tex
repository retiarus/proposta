%%%%%%%%%%%%%%%%%%%%%%%%%%%%%%%%%%%%%%%%%%%%%%%%%%%%%%%%%%%%%%%%%%%%%%%%%%%%%%%%
% RESUMO %% obrigatório

\begin{resumo}

%% neste arquivo resumo.tex
%% o texto do resumo e as palavras-chave têm que ser em Português para os documentos escritos em Português
%% o texto do resumo e as palavras-chave têm que ser em Inglês para os documentos escritos em Inglês
%% os nomes dos comandos \begin{resumo}, \end{resumo}, \palavraschave e \palavrachave não devem ser alterados

\hypertarget{estilo:resumo}{} %% uso para este Guia

A ionosfera é uma camada de gás em estado de plasma mais uma porção de matéria neutra cujo processo de ionização tem a radiação solar como seu principal agente. A distribuição do plasma é não uniforme ao longo do espaço e do tempo tendo a transição entre periódo iluminado e não iluminado como um elemento importante na geração de irregularidades, tal como a Anomalia de Ionização Equatorial (AIE). Acoplado AIE pelo mecanismo de instabilidade do plasma se tem a formação de depleções, isto é, regiões com baixa densidade de portadores de carga. Estas estruturas são conhecidas como bolhas ionosféricas, e são geradas no equador magnético após o pôr do sol, quando então ascendem a altas altitudes e migram para baixas latitudes ao longo do campo magnético da Terra. Os sinais de radiofrequência dos sistemas de navegação por satélites (GNSS - Global Navigation Satellite System) são afetadas por flutuações na densidade de elétrons, isto é, pelas bolhas de plasmas. A medida de tal fenômeno constitui a cintilação ionosférica, em outras palavras, é uma medida das pertubações do sinal de GNSS em príncipio decorrentes de irregularidades na ionosfera. Em geral, o sinal degrado após a pertubação terá flutuações de amplitude e fase. Nesta proposta, técnicas de descoberta de conhecimento em bases de dados são empregadas para predição de cintilação ionosférica no território brasileiro, especialmente em São José dos Campos. Empregam-se dados históricos de cintilação ionosférica e outros tais como nível de atividade solar, velocidadede deriva vertical do plasma, atividade magnética global. O algoritmo escolhido para predição é o Extreme Gradient Boosting (XGBoost), disponível no ambiente de programação Python.

\palavraschave{%
  \palavrachave{Cintilação Ionosférica}%
%  \palavrachave{Bolha Ionosférica}%
  \palavrachave{Índice S4}%
%  \palavrachave{VTEC}%
  \palavrachave{Mineração de dados}%
  \palavrachave{GNSS}
%  \palavrachave{Vale do Paraíba}
}
 
\end{resumo}
