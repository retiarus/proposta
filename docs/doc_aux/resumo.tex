%%%%%%%%%%%%%%%%%%%%%%%%%%%%%%%%%%%%%%%%%%%%%%%%%%%%%%%%%%%%%%%%%%%%%%%%%%%%%%%%
% RESUMO %% obrigatório

\begin{resumo}

%% neste arquivo resumo.tex
%% o texto do resumo e as palavras-chave têm que ser em Português para os documentos escritos em Português
%% o texto do resumo e as palavras-chave têm que ser em Inglês para os documentos escritos em Inglês
%% os nomes dos comandos \begin{resumo}, \end{resumo}, \palavraschave e \palavrachave não devem ser alterados

\hypertarget{estilo:resumo}{} %% uso para este Guia

%A ionosfera é uma camada da atmosfera que se estende de aproximadamente 60 km a 1000 km de altitude. Esta camada influi nos sinais de radiofrequência transmitidos por satélites para a superfície terrestre, sendo composta por gases ionizados principalmente pela radiação solar e elétrons livres. Variabilidades no fluxo solar geram alterações nos campos elétrico e magnético no espaço e, consequentemente, no campo magnético da Terra causando flutuações na ionização e, portanto, na quantidade de elétrons livres na ionosfera, alterando a transmissão de sinais de radiofrequência. Dentre as várias pertubações ionosféricas, há a anomalia magnética equatorial, que consiste na formação de uma região com alta densidade de elétrons ente 15 e 20 graus magnéticos ao norte e sul do equador. Entretanto, essa anomalia é mais significativa no Brasil, especificamente no Vale do Paraíba, estado de São Paulo. Os sinais de radiofrequência dos sistemas de navegação por satélites (GNSS - Global Navigation Satellite System) são afetados por flutuações da ionização, que constituem o fenômeno de cintilação ionosférica, que pode ser medidas pelo índices S4. A cintilação afeta sinais GNSS, especialmente no pico da anomalia, afetando a navegação aérea e outras atividades humanas que dependem de sinais recebidos de satélites.  Por outro lado, a quantidade de elétrons livres na ionosfera pode ser medida pelo conteúdo eletrônico total vertical (VTEC), sendo que regiões com baixos valores de VTEC em relação à sua vizinhança, caracterizam as denominadas de bolhas ionosféricas, associadas às cintilações. Dada a existência de redes de estações de medição que provém valores de S4 e VTEC, este trabalho busca correlacionar os valores destas variáveis, bem como analisar sua evolução espaço-temporal por meio de técnicas de mineração e visualização de dados, considerando como estudo de caso a cidade de São José dos Campos.

\palavraschave{%
  \palavrachave{Cintilação Ionosférica}%
%  \palavrachave{Bolha Ionosférica}%
  \palavrachave{Índice S4}%
%  \palavrachave{VTEC}%
  \palavrachave{Mineração de dados}%
  \palavrachave{GNSS}
%  \palavrachave{Vale do Paraíba}
}
 
\end{resumo}
