\chapter{IONOSFERA}

A ionosfera é uma região ionizada da alta atmosfera, estendendo-se de 60 até 1000 km de altitude, assim, englobando partes da mesosfera, termosfera, e exosfera. Esta camada constitui-se de íons e elétrons livres criados primariamente por processo de fotoionização, e gás neutro. A fotoionização ionosférica consiste de um processo físico-químico, onde alguma espécies químicas presentes na atmosfera ganham ou perdem elétrons decorrentes da absorção de radiação solar predominantemente nas faixas ultravioleta, extremo ultravioleta ultravioleta e raios-X \cite{RISNBETH:1969, NEGRETI:2012}. A ionização, também, pode ocorrer devido a colisões com partículas altamente energéticas, provindas do meio solar ou galácticas, o que é mais facilmente observado em altas latitudes, por fenômenos como auroral boreal.

A composição da ionosfera, assim, como a densidade do gases variam em função da altitude. Assim, a densidade de elétrons também varia, pois conforme a radiação penetra na atmosfera mais densa, a produção de elétrons, aumenta até atingir um valor de pico. Abaixo desta altitude, mesmo havendo um aumento na densidade da atmosfera neutra, a produção de elétrons decresce, pois a maior parte da radiação ionizante foi absorvida, e a taxa de recombinação predomina sobre a taxa de produção de elétrons. Devido as diferenças marcantes em termos de processos físicos e químicos que governam o comportamento daquela ionosfera, a mesma, pode ser dividida em camadas, onde em cada uma existe um processo predominante. Finalmente, devido as drásticas mudanças em quantidade de radiação absorvida devido a transição entre noite e dia, existirão camadas que não aparecem em cada instante ao longo de 24 horas.

Durante o período a camada F é a única que apresenta uma ionização significativa, enquanto as camadas E e D apresentam um valor extremamente baixo de ionização. Durante o dia, a camada D e E se tornam mais ionizadas, assim como a camada F, que se divide em duas regiões, F1 que é mais fracamente ionizada, e F2 que é mais intensamente ionizada. A camada F2 existe durante a noite e durante o dia, sendo a principal responsável pela reflexão e refração dos sinais de rádio.

A camada D é a mais interna, estando entre 60 e 90 km acima da superfície da Terra. Sua ionização é devido a radiação do hidrogênio ionizado na série de Lyman-alpha no comprimento de onda de 121.6 nm ionizando o óxido nítrico, $NO$, presente na camada. Além disso, raios X altamente energéticos, com comprimento de onda inferior de 1 nm podem ionizar as moléculas de $N_2$ e de $O_2$. A camada D apresenta apresenta a maior taxa de recombinação. Apresenta uma taxa de absorção considerável para ondas de radio de baixas e médias frequências, principalmente, devido a absorção de energia pelo elétrons livres, o que aumenta suas chances de colisão. Este efeito desaparece durante à noite, devido a uma menor ionização. Pode apresentar valores elevados de ionização em altas latitudes em decorrência de erupções solares com grandes quantidades de matéria hadrônica, prótons, em sua maioria, com uma duração de 24 à 48 horas.

A camada E é a intermediária e está situada entre 90 e 150 km acima da superfície da Terra. A ionização decorre principalmente devido ao espalhamento de raio-X leve (entre 1 e 10 nm) e ultravioleta distante (UV) provindos do Sol com moléculas de oxigênio. A estrutura vertical da camada E é determinada em sua maior parte pela competição entre efeitos de ionização e de recombinação. É importante pela presença de correntes elétricas que nela fluem e interagem com o campo magnético \cite{KIRCHHOFF:1991}. A noite, a camada E quase desaparece, pois sua fonte primaria de ionização não está presente.

A camada F se estende de 150 a mais de 500 km acima da superfície da Terra. Apresenta a maior concentração de elétrons, portanto, sinais que são capazes de penetrar até esta subcamada são capazes de escapar para o espaço. Predominam, nesta, a ionização de átomos de oxigênio por meio de radiação solar no espectro do extremo ultravioleta, entre, 10 e 100 nm. A camada é subdivida em duas regiões, a F2 que está presente durante o dia e a noite, e a F1 que aparece somente durante o dia. 

A subcamada F2 engloba toda a região superior da ionosfera, inclusive a região de pico da densidade de elétrons. Este máximo no perfil vertical de ionização decorre do balanço entre os processos de transporte de plasma e os processos físico-químicos. Acima deste pico, a ionosfera se encontra em equilíbrio difusivo, ou seja, o plasma se distribui com a sua própria escala de altura. A presença do campo magnético contribui para a distribuição da ionização.

\subsection{Termosfera}

A termosfera se inicia aproximadamente a 90 km de altitude e não apresenta um limite superior bem definido, mas que deve estar entre 500 e 600 km de altitude. Nesta surgem os fenômenos de aurora, e nela também estam localizadas as órbitas da estação espacial, do ônibus espacial, assim como de vários satélites. Assim, como todas as demais camadas da atmosfera, sua extensão varia conforme a latitude. A temperatura nesta camada podem alcançar valores superiores a 2000 graus Celsius.

\subsection{Magnetosfera}

\section{Anomalias na ionosfera}

A ionosfera apresenta várias anomalias, ou seja, várias irregularidades na distribuição de elétrons. Este trabalho tem interesse na anomalia equatorial. Esta aparece aproximadamente entre 15 e 20 graus de latitude magnética, tanto no hemisfério norte, quanto no hemisfério sul, na camada F2. Consiste na formação de uma região de alta densidade eletrônica, e é uma anomalia, pois a densidade de plasma deveria ser maior em regiões equatoriais, e não em latitudes magnéticas mais altas.

Sua origem decorre da deriva vertical do plasma da camada F na região equatorial: o processo de ionização da camada F faz surgir um campo elétrico, apontando para leste, enquanto o campo magnético aponta para o norte, considerando então $\vec{E}\times\vec{B}$, tem-se o surgimento de uma força perpendicular ao campo magnético e ao campo elétrico, o que neste caso, aponta para cima, deslocando o plasmas para regiões de mais alta altitudes. Agora, quando em altas altitudes, o plasma for efeito gravitacional e diferença de pressão é trazido de volta à altitudes mais baixas, porém este movimento de descida é mais eficiente ao longo das linhas de campo magnético, levando a um aumento na densidade de plasma em regiões de médias latitudes. 

A distribuição do plasma também pode ser alterada pela ação de outras variáveis, como o vento. O Vale do Paraíba, no estado de São Paulo, encontra-se na região da anomalia equatorial, mais especificamente no pico da anomalia, ou seja, na região onde a densidade de plasma, em altas altitudes, atinge seu valor máximo.

As anomalias que surgem na ionosfera apresentam um certo nível de organização devido às linhas de campo magnético da Terra. Isto ocorre pois partículas ionizadas ou carregas podem ser mover livremente ao longo das linhas magnéticas mas não entre elas. Assim, o estudo do campo magnético da Terra se faz relevante para um grande número de aplicações, \cite{LAUNDAL:2017}. Um resultado imediato deste estudo é que o norte geográfico e magnético não coincidem, e que simplificadamente o campo magnético poderia ser descrito por um dipolo magnético, com centro comum ao da Terra, porém inclinado em relação a linha que liga o norte e sul geográfico. Atualmente, existe vários sistemas de coordenadas magnéticas cujo proposito dependem da região, da aplicação e da faixa de altitude de interesse, para uma revisão entre os sistemas mais comuns consulte a referência \cite{LAUNDAL:2017}. Para este trabalho foi adotado o sistema AACGM, pois é mais adequado à altura ionosférica, contudo ele pertence a classe de sistemas não-ortogonais.

A cintilação ionosférica é uma variação rápida de amplitude e fase em sinais de radio frequência quando estes atravessam irregularidades no plasma ionosférico, como uma bolha de plasma, que é de particular interesse para este trabalho. Neste trabalho se realiza um estudo sobre algumas variáveis relacionadas com o efeito de cintilação ionosférica e com a estrutura de bolha no plasma.

\section{Algumas variáveis importantes para o estudo da cintilação ionosférica}

Existem várias quantidades que são necessárias para um estudo completo da dinâmica ionosférica como, por exemplo, medidas do fluxo de radiação solar, do campo magnético da Terra, da composição da atmosfera. Contudo, este trabalho foca em duas quantidades principais o VTEC e o índice S4, pois são as que fornecem mais informação à respeito do fenômeno de cintilação ionosférica, e as anomalias que a causam.

O Total Eletronic Content (TEC), ou conteúdo total de elétrons em português, é uma quantidade descritiva utilizada para avaliar a densidade de elétrons no plasma ionosférico. É o número total de elétrons integrado ao longo da trajetória entre um transmissor, no espaço, e um receptor, na Terra, em uma seção unitária \cite{HOFMANN:2013}. Por sua definição o TEC é uma quantidade que depende da trajetória, seu cálculo fica mais claro, ao dizer que é a integral de um densidade de elétrons dependente de posição ao longo de uma trajetória que atravessa a ionosfera:

\begin{equation}
    \mbox{TEC}=\int{n_{e}(s)}ds\mbox{,}~
\end{equation}

onde $ds$ especifica o elemento de integração na trajetória. Geralmente é reportada em unidades de TEC (TECU), definido por 1 $TECU=10^{16}$ elétrons/m$^2$. É importante para determinar a cintilação e os atrasos de fase e de grupo em ondas de rádio no meio. VTEC, ou conteúdo total de elétrons vertical é projeção do TEC, ao longo de uma linha normal a superfície da Terra, em outras palavras, ela fornece o conteúdo de elétrons ao longo da normal para cada ponto da superfície da Terra.

O índice S4 é utilizado para medir, avaliar, a cintilação ionosférica. Corresponde ao desvio padrão da intensidade do sinal de GPS de um minuto de dados, coletados com 50 amostras por segundo:

\begin{equation}
    S_4^2=\frac{\braket{I^2}-\braket{I}^2}{\braket{I}^2}\mbox{.}
\end{equation}

\section{Bolhas de Plasma}

As bolhas ionosféricas podem ser definidas como regiões de baixa densidade de plasma ionosférico quando comparadas com a sua vizinhança. Utilizando medidas de VTEC é possível definir essa diferença como 30-50 TECU \cite{TAKAHASHI:2006}.

São originadas na região equatorial, após a rápida elevação do plasma, devido a anomalia equatorial, isto é, o plasma ao acender cria regiões de baixa densidade. Após sua formação podem evoluir para altas altitudes (centenas de quilômetros), estendendo-se ao longo das linhas de campo magnético (milhares de quilômetros) nas direções norte-sul, alcançado em torno de 20 graus de latitude magnética.
