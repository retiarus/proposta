\chapter{Metodologia}

Este trabalho utiliza o índice S4, uma lista de estações que realizam medidas desta variável e o VTEC. O trabalho adota a ideia de pesquisa reprodutível, para tal se faz uso da tecnologia de ``notebooks"~em Python, \cite{PEREZ:2007}. Python notebooks consistem de um ambiente computacional interativo de código aberto, onde pode-se combinar execução de código, com textos e expressões matemáticas em HTML e LaTex, gráficos, imagens, vídeos, entre diversos outros objetos. O projeto inicial evolui para o Jupyter notebook que apresenta as mesmas ideias, porém apresenta suporte para várias linguagens de programação, tal como R, Julia e Scala. 

Os dados de S4 foram originalmente disponibilizados em formato de texto, juntamente com uma lista de estações ao longo do território brasileiro. Cada arquivo contém uma lista de medidas ordenadas pelo tempo, com múltiplas medidas por minuto, cada uma associada a um satélite. A etapa inicial consistiu em armazenar e organizar estes dados em um banco de dados, pois então, técnicas como filtragem permitem uma rápida seleção inicial dos dados por estação e elevação. Também foi adicionado ao banco de dados uma tabela com informações sobre as estações. Neste trabalho, adotou-se o PostgreSql.

Os notebooks estão organizados tal que os primeiros dois dígitos indiquem uma ordem de execução, por exemplo, o notebook com inicial $00$ precisa ser executado antes do notebook com inicial $01$. Existem notebooks com os mesmos valores de dígitos, isto significa que um não apresenta dependência em relação ao outro e podem ser executados ao mesmo tempo.

\section{Seleção inicial dos dados de S4 e estações}

Inicialmente, realizou-se uma consulta a tabela de estações no banco de dados. Utilizando, então, a lista de estações retornada, fez-se iterativamente uma consulta para cada estação selecionando apenas medidas cuja elevação é superior a 30.0. Somente continuaramm as estações que contém dados, gerando assim, uma tabela com os dados das estações, que entender-se-á como o conjunto inicial de estações, e um grupo de arquivos com dados de S4, um para cada estação válida. Os dados de S4 armazenados em uma tabela ordenada, indexada pelo tempo, podem ser vistos como uma série temporal. Esta fase foi desenvolvida no notebook 00\_extract\_valid\_stations.ipynb. 

\section{Geração da série espaço-temporal para os dados de VTEC}

Os dados de VTEC estão inicialmente organizados em arquivos de texto, as duas primeiras linhas são de cabeçalho, onde a primeira, denota o instante dos dados, e a segunda fornece o significado de cada coluna. A primeira coluna é a longitude, a segunda a latitude e a terceira o VTEC. Medidas de VTEC $999.000$ denotam ausência de valor. Finalmente, cada arquivo constitui um mapa de VTEC, onde cada linha fornece o valor de VTEC para um ponto no espaço. Assim, o papel do notebook 00\_generate\_vtec\_dataframe.ipynb é o de converter este conjunto de arquivos em um tabela, indexada pelo tempo, onde cada linha contém uma matriz, que realiza o papel do mapa. Pode-se observar esta estrutura também como uma série espaço-temporal, onde um índice de tempo indexa um mapa de VTEC. Este notebook também realiza um ordenamento temporal nos dados, o que é necessário posteriormente para o cálculo de derivadas temporais, traduzidas aqui como diferenças finitas no tempo. Esta etapa é realizada pelo notebook 00\_generate\_vtec\_dataframe.ipynb.

\section{Mapa das estações com os meridianos magnéticos} 

As bolhas ionosféricas evoluem, como mencionado, ao longo de meridianos magnéticos. Assim, é interessante visualizar o conjunto de estações por meio de um mapa juntamente com os meridianos magnéticos que as atravessam, para tal, inicialmente, realiza-se um ordenamento na tabela de estações por estado e cidade. Seguindo de um agrupamento por cidade, uma vez que existem cidades com mais de um estação, e neste trabalho se adotou empregar apenas uma estação por cidade. 

Por meio do pacote Python AACGMV2, que implementa um modelo numérico para o sistema de coordenadas geomagnética AACGM, foi possível gerar as latitudes e longitudes geomagnéticas para as estações, utilizando os dados de localização geográfica, altitude, bem como uma data de referência.

Utilizando os dados geomagnéticos se plotou um mapa contendo as estações, assim, como os meridianos magnéticos que passam por estas. O mapa plotado na figura \ref{fig:mapstations} fornece uma visualização das estações que tendem a apresentam maior correlação entre si, pois pertencem a uma mesma eventual trajetória. Esta etapa é desenvolvida no notebook 01\_show\_stations.ipynb.

\begin{figure}[H]
\centering
\makebox[\textwidth][c]{\includesvg[width=1.6\columnwidth]{./Figuras/map_stations.svg}}
\vspace{-60pt}
\caption{Representação dos meridianos magnéticos que atravessam as estações que realizam medidas de cintilação ionosférica. Fonte: próprio autor.}
\label{fig:mapstations}
\end{figure}

\section{Suavização da série temporal de S4}

Os dados de S4 apresentam intensas variabilidades, devido a rápidas flutuações, ruídos e a existem de grandes intervalos de tempo sem dados, portanto é necessário realizar um pré-processamento nos dados de S4. Primeiramente, toma-se o valor médio de todas as medidas realizada por minuto, isto é, realiza-se uma reamostragem combinando as medidas de diferentes satélites feitas no mesmo minuto, tal que se tenha um valor de S4 por minuto. 

Após esta etapa é feita uma interpolação spline de ordem 4 para tratar instantes sem amostras. Em seguida, é feita uma reamostragem para intervalos de 10 min, novamente utilizando o valor médio como função de agregação. O intervalo adotado é o mesmo do dados de VTEC. Nas Figuras \ref{fig:savgol} e \ref{fig:gaussian} é possível observar em azul uma amostra do sinal S4, pode-se notar que este apresenta rápidas flutuações e ruído, como comentado. Assim, optou-se por aplicar uma função de suavização. Em \ref{fig:savgol}, tem-se a utilização do filtro de Savitzky–Golay com janela de 9 pontos e polinômio interpolador de terceira ordem, enquanto na figura \ref{fig:gaussian} foi utilizado um filtro baseado em média móvel com pesos gaussianos com uma janela também de 9 pontos.

\begin{figure}[H]
\centering
\makebox[\textwidth][c]{\includegraphics[width=1.2\linewidth]{./Figuras/s4_signal_noise_and_smooth_savgol.eps}}
\caption{Suavização de uma amostra da série temporal de S4, em São José dos Campos - SP, por meio do filtro de Savitzky–Golay, no intervalo entre 18:00 UT do dia 25/01/2013 até 12:00 UT do dia 27/01/2013. As linhas verticais vermelhas representam o por do sol, enquanto as amarelas representam o nascer do sol. Fonte: próprio autor.}
\label{fig:savgol}
\end{figure}

\begin{figure}[H]
\makebox[\textwidth][c]{\includegraphics[width=1.2\linewidth]{./Figuras/s4_signal_noise_and_smooth_gaussian.eps}}
\caption{Suavização de uma amostra da série temporal de S4, em São José dos Campos - SP, por meio de uma média móvel com peso gaussiano, no intervalo entre 18:00 UT do dia 25/01/2013 até 12:00 UT do dia 27/01/2013. As linhas verticais vermelhas representam o por do sol, enquanto as amarelas representam o nascer do sol. Fonte: próprio autor.}
\label{fig:gaussian}
\end{figure}

Finalmente, optou-se por utilizar uma combinação das duas técnicas de suavização aplicando primeiro do filtro de Savitzky–Golay seguido da média móvel com pesos gaussianos, ambos com os parâmetros especificados no parágrafo anterior. Essa escolha gera um sinal mais suave do que a aplicação de um ou outro. Nos gráficos estão presentes também linhas verticais vermelhas e amarelas que representam respectivamente o pôr do sol e o nascer do sol em UT calculadas segundo a localização geográfica da estação. Tais linhas estão presentes entre outros gráficos ao longo do texto, mantendo o mesmo padrão de cores.

Nesta etapa adicionalmente foi gerado um tabela com todos os dados de S4, em passos de 10 min, onde as colunas representam o conjunto inicial de estações. Os códigos desenvolvidos nessa fase são implementados pelo notebook 02\_preprocessing\_s4\_data.ipynb.

\section{Extração da série temporal de VTEC, para algumas estações}

Dados os mapas de VTEC em sua forma tabela contendo matrizes é conveniente, em termos de uso e desempenho, extrair as séries temporais para as localizações em que existam estações onde o índice S4 é medido. Estes dados então são preprocessados aplicando o mesmo processo de suavização utilizados nos dados de cintilação. Nas Figuras \ref{fig:vtecsj2gauss} e \ref{fig:vtecsj2savi} são possíveis visualizar amostras do VTEC para São José dos Campos, juntamente com a aplicação das duas técnicas de suavização. Na figura \ref{fig:vtecsignal} há uma amostra do sinal de VTEC para o conjunto inicial de estações, observe que no intervalo entre às 9:00 e 15:00 UT o VTEC é aproximadamente similar entre as diversas estações, e que a partir das 18:00 UT os valores começam a divergir entre si, apresentado grandes diferenças após 00:00 UT, até 06:00 UT, onde então começam a se agrupar. Tal padrão se apresenta ao longo de todo o período amostrado para este trabalho.
\vspace{-16pt}

\begin{figure}[H]
\centering
\makebox[\textwidth][c]{\includegraphics[width=1.2\columnwidth]{./Figuras/vtec_sj2_gauss.eps}}
\caption{Amostra de um sinal de VTEC, em São José dos Campos - SP, juntamente com sua suavização utilizando média móvel com peso gaussiano. O correspondente período foi de 00:00 UT do dia 01/12/2013 até 12:00 UT do dia 04/12/2013. As linhas verticais vermelhas representam o por do sol, enquanto as amarelas representam o nascer do sol. Fonte: próprio autor.}
\label{fig:vtecsj2gauss}
%\end{figure}
%
%\begin{figure}[H]
\centering
\makebox[\textwidth][c]{\includegraphics[width=1.2\columnwidth]{./Figuras/vtec_sj2_savi.eps}}
\caption{Amostra de um sinal de VTEC, em São José dos Campos - SP, juntamente com suas suavização pelo método Savitzky–Golay. O correspondente período foi de 00:00 UT do dia 01/12/2013 até 12:00 UT do dia 04/12/2013. As linhas verticais vermelhas representam o por do sol, enquanto as amarelas representam o nascer do sol. Fonte: próprio autor.}
\label{fig:vtecsj2savi}
\end{figure}

\vspace{-16pt}
Esta é implementada pelo notebook 03\_extract\_vtec\_stations.ipynb, o qual também gera uma tabela, onde cada coluna representa uma estação diferente com os dados de VTEC indexados pelo tempo, de modo que, pode-se falar em uma tabela de séries temporais para os dados de VTEC.

\begin{figure}[H]
\makebox[\textwidth][c]{\includegraphics[width=1.3\columnwidth]{./Figuras/vtec_signal.eps}}
\caption{Amostra de um sinal de VTEC, para múltiplas estações. Note a convergência da série para a vizinhança das 12:00 UT e sua divergência próximo das 00:00 UT. O gráfico correspondente ao período de 12:00 UT do dia 01/12/2013 até 12:00 UT do dia 03/12/2013. As linhas verticais vermelhas representam o por do sol, enquanto as amarelas representam o nascer do sol. Fonte: próprio autor.}
\label{fig:vtecsignal}
\end{figure}

\vspace{-16pt}
\section{Seleção fina dos dados de S4}

Após o pré-processamento dos dados de S4, é realizada uma seleção mais fina das estações que serão utilizadas neste trabalho. O primeiro conjunto de estações descartas o foi, pois apresentava poucos pontos ao longo do período de janeiro de 2013 à dezembro de 2014 o que levou a curvas interpoladas que não são condizentes com a variável observada, o que foi evidenciado pela plotagem das séries temporais. 

Os dados de VTEC foram amostrados em um intervalo menor do que o S4, assim, foi realizado um recorte na série temporal de S4, tal que ambas as séries tenham o mesmo intervalo de amostragem. Assim, o segundo conjunto de estações descartadas são aquelas que não apresentam medidas no período de tempo selecionado. A Tabela \ref{tab:stations} apresenta o grupo final de estações selecionados para o trabalho. A figura \ref{fig:s4stations} exibe a série temporal S4 para algumas estações, enquanto a figura \ref{fig:mapstationsre} apresenta um mapa com todas as estações selecionadas. Esta fase foi desenvolvida no notebook 04\_reanalize\_data.ipynb.

\begin{table}
\addtolength{\leftskip} {-2cm} % increase (absolute) value if needed
\addtolength{\rightskip}{-2cm}
\small
\begin{tabular}{|l|l|l|c|c|c|c|c|}
\hline
Cidade              & Est.  & Cód. de Id.           &  Alt.     &   Lat.     &  Lon.      &  Lat. Mag.    &  Lon. Mag.       \\ \hline
Belo Horizonte      &    MG &                   bhz &   858.000 & -19.868500 & -43.954200 &    -25.426147 &      24.786619   \\ \hline
Brasília            &    DF &                   bsa &  1050.000 & -15.764200 & -47.869400 &    -24.348659 &      22.352744   \\ \hline
Cachoeira Paulista  &    SP &                   cpa &   580.000 & -22.410000 & -45.000000 &    -24.456556 &      22.960540   \\ \hline
Campo Grande        &    MS &                    32 &       NaN & -20.497000 & -54.615000 &    -21.417704 &      14.873907   \\ \hline
Cuiabá              &    MT &                   cub &   278.000 & -15.555200 & -56.069800 &    -14.336068 &      14.530440   \\ \hline
Dourados            &    MS &                   dou &   756.120 & -22.110000 & -54.550000 &    -23.627266 &      14.698554   \\ \hline
Fortaleza           &    CE &                    24 &       NaN &  -3.742000 & -38.539000 &           NaN &            NaN   \\ \hline
Guaratinguetá       &    SP &                    33 &       NaN & -22.789000 & -45.220000 &    -24.188879 &      22.620120   \\ \hline
Ilhéus              &    BA &                   ios &     0.000 & -14.470000 & -39.100000 &    -13.470248 &      30.548727   \\ \hline
Inconfidentes       &    MG &                    25 &       NaN & -22.318000 & -46.329000 &    -26.299459 &      22.004117   \\ \hline
Macaé               &    RJ &                    11 &       NaN & -22.823000 & -41.785700 &    -20.542047 &      25.191448   \\ \hline
Natal               &    RN &                   nta &     0.000 &  -5.836162 & -35.121000 &           NaN &            NaN   \\ \hline
Palmas              &    RO &                     3 &       NaN & -10.200000 & -48.312000 &    -12.264838 &      23.425112   \\ \hline
Pirassununga        &    SP &                    30 &       NaN & -21.989000 & -47.334000 &    -23.990783 &      21.003125   \\ \hline
Porto Alegre        &    RS &                     4 &       NaN & -30.071000 & -51.119000 &    -22.954879 &      15.550843   \\ \hline
Presidente Prudente &    SP &                     6 &       NaN & -22.120000 & -51.407000 &    -21.640946 &      17.249042   \\ \hline
Rio de Janeiro      &    RJ &                    34 &       NaN & -22.823000 & -43.238000 &    -20.105803 &      23.888647   \\ \hline
Salvador            &    BA &                    26 &       NaN & -13.001000 & -38.508000 &    -12.123350 &      31.680944   \\ \hline
Santa Maria         &    RS &                   sta &   110.100 & -29.712591 & -53.717206 &    -22.659740 &      13.628064   \\ \hline
São José dos Campos &    SP &                   sj2 &   593.440 & -23.207000 & -45.859000 &    -24.835610 &      22.002028   \\ \hline
Tefé                &    AM &                   tfe &     0.057 &  -3.180000 & -64.440000 &      6.385157 &       9.314963   \\ \hline
\end{tabular}

\vspace{12pt}

\begin{center}
\begin{tabular}{|l|c|c|c|}
\hline
Cidade              &   Alt. da Cidade &  Lat. da Cidade &  Lon. da Cidade \\ \hline
Belo Horizonte      &            767.0 &       -19.81570 &        -43.9542 \\ \hline
Brasília            &           1130.0 &       -15.78010 &        -47.9292 \\ \hline
Cachoeira Paulista  &            545.0 &       -22.67370 &        -44.9973 \\ \hline
Campo Grande        &            612.0 &       -20.44350 &        -54.6478 \\ \hline
Cuiabá              &            180.0 &       -15.59890 &        -56.0949 \\ \hline
Dourados            &            448.0 &       -22.22180 &        -54.8064 \\ \hline
Fortaleza           &             14.0 &        -3.71839 &        -38.5434 \\ \hline
Guaratinguetá       &            526.0 &       -22.81620 &        -45.1935 \\ \hline
Ilhéus              &              9.0 &       -14.79730 &        -39.0355 \\ \hline
Inconfidentes       &            864.0 &       -22.31710 &        -46.3284 \\ \hline
Macaé               &              7.0 &       -22.37170 &        -41.7857 \\ \hline
Natal               &             38.0 &        -5.79448 &        -35.2110 \\ \hline
Palmas              &            260.0 &       -10.16890 &        -48.3317 \\ \hline
Pirassununga        &            625.0 &       -21.99600 &        -47.4268 \\ \hline
Porto Alegre        &             22.0 &       -30.02770 &        -51.2287 \\ \hline
Presidente Prudente &            471.0 &       -22.12760 &        -51.3856 \\ \hline
Rio de Janeiro      &             20.0 &       -22.90350 &        -43.2096 \\ \hline
Salvador            &             12.0 &       -12.97040 &        -38.5124 \\ \hline
Santa Maria         &            139.0 &       -29.69140 &        -53.8008 \\ \hline
São José dos Campos &            593.0 &       -23.17910 &        -45.8872 \\ \hline
Tefé                &             28.0 &        -3.32073 &        -64.7236 \\ \hline
\end{tabular}
\end{center}

\vspace{12pt}

\caption{Conjunto de estações que realizam medidas do índice S4 juntamente com seus atributos. Fonte: próprio autor.}
\label{tab:stations}
\end{table}

\begin{figure}[H]
\centering
\makebox[\textwidth][c]{\includegraphics[width=1.2\columnwidth]{./Figuras/s4_stations_sample.eps}}
\caption{Amostra do sinal S4 de dezembro de 2013 até março de 2014. Fonte: próprio autor.}
\label{fig:s4stations}
\end{figure}

\begin{figure}[H]
\centering
\makebox[\textwidth][c]{\includesvg[width=1.6\columnwidth]{./Figuras/map_stations_re.svg}}
\vspace{-60pt}
\caption{Conjunto final de estações que serão utilizadas para este trabalho. Fonte: próprio autor.}
\label{fig:mapstationsre}
\end{figure}

\section{Visualização da séries temporais para S4 e VTEC}\label{sec:viss4vtec}

Feita uma seleção mais fina das estações que fornecem as séries temporais de S4, assim, como um recorte adequado no tempo, é adequado plotar a série temporal do VTEC, juntamente com a do S4. Esta etapa é realizada no notebook 05\_visualize\_vtec\_s4\_data. Os gráficos são feitos em dois grupos distintos, o primeiro apresenta uma amostra das séries temporais, fornecendo um resolução visual melhor, enquanto o segundo fornece a série completa. Algumas observações pode ser feitas analisando visualmente ambos os conjuntos. Nota-se, por exemplo:

\begin{itemize}
\item uma periodicidade em ambos os dados; 
\item os valores de S4 sobem conforme o do VTEC diminui;
\item os valores de pico de S4 aparecem em quedas e mínimos do VTEC;
\item os dados de S4 apresentam maior ruído e menor disponibilidade;
\item flutuações mais intensas do S4 aparecem no máximo do VTEC;
\item após o por do sol existe um aumento no valor de VTEC correspondente ao pico de pré reversão, por sua vez quando confrontado com o S4, observa-se uma certa sobreposição entre a cintilação e o pico de pré reversão.~
\end{itemize}

Nas Figuras \ref{fig:s4vtecsample} e \ref{fig:s4vteccomplete} é possível observar respectivamente uma amostra da série temporal de S4 contra VTEC, e a série completa para algumas estações selecionadas.

\begin{figure}[H]
\centering
\makebox[\textwidth][c]{\includegraphics[width=1.2\columnwidth]{./Figuras/s4_vtec_sample.eps}}
\caption{Amostra do sinal S4 e VTEC, no intervalo entre 00:00 UT do dia 01/12/2013 até 12:00 UT do dia 04/12/2013. As linhas verticais vermelhas representam o por do sol, enquanto as amarelas representam o nascer do sol. A Figura \ref{fig:scattervtec} é um recorte para a estão de São José dos Campos onde estão indicados possíveis picos de pré reversão, estes são as regiões com aumento no VTEC logo após o por do sol. Fonte: próprio autor.}
\label{fig:s4vtecsample}
\end{figure}

\begin{figure}[H]
\centering
\makebox[\textwidth][c]{\includegraphics[width=1.2\columnwidth]{./Figuras/s4_vtec_complete.eps}}
\caption{Série temporal para S4 e VTEC, ao longo de todo o período de coleta, isto é de de dezembro de 2013 até março de 2014. Fonte: próprio autor.}
\label{fig:s4vteccomplete}
\end{figure}

\section{Variáveis derivadas do VTEC}\label{sec:vdt}

Existem várias formas de se buscar por padrões e correlações em um conjunto de dados, tal como a visualização por meio da plotagem de um gráfico representativo de uma série temporal. Assim, a seção \ref{sec:viss4vtec} forneceu um indicativo da correlação em sua conclusão. Observado tal fato o próximo passo é buscar por um modelo que faça um mapeamento entre VTEC e S4. Este pode ter como conjunto de entrada o valor de VTEC e o de saída o S4, uma vez que ambas as variáveis são contínuas, busca-se realizar uma regressão. 

Poder-se-ia buscar um mapeamento tal que a cada valor de VTEC seja associado a um valor de S4 por estação, entretanto, usando os padrões observados juntamente das referências \cite{RAGHUNATH:2016, RAGHAVARAO:1998,RAY:2006}, derivou-se um conjunto adicional de variáveis, que são a derivada temporal primeira e segunda do VTEC, a diferença do VTEC entre São José dos Campos e Pirassununga, a diferença do VTEC entre São José dos Campos e Brasília, assim como as derivadas temporais primeira de ambas as diferenças. 

As derivadas temporais são interessantes, pois os valores de S4 aumentam conforme o valor de VTEC diminui, portanto existe uma variação no tempo que pode ser melhor extraída pela derivada temporal. As bolhas ionosféricas se deformam e se propagam ao longo de um meridiano magnético, as estações de Brasília, Pirassununga e São José dos Campos se encontram aproximadamente sobre o mesmo meridiano magnético, usando ambos os fatos considere que exista uma bolha em Brasília, e não em São José dos Campos, a diferença de VTEC terá um valor positivo, enquanto que se estiverem em ambas as cidades ter-se-á um valor aproximadamente nulo, e com a bolha apenas em São José dos Campos um valor de diferença negativa, isto exibe claramente a propriedade da diferença espacial do VTEC em mapear a localização da bolha. A derivada temporal das diferenças espaciais fornece um indicativo da dinâmica da bolha.

Sumarizando, ficou-se com o seguinte conjunto de variáveis, denominado conjunto ``original'' de atributos, mais uma variável alvo S4:

\begin{itemize}
\item {\bf vtec} - conteúdo eletrônico total vertical em São José dos Campos;
\item {\bf vtec\_dt} - diferença finita de primeira ordem no tempo do VTEC, calculada por $vtec_i-vtec_{i-1}$;
\item {\bf vtec\_dt2} - diferença finita de segunda ordem no tempo do VTEC, calculada por $vtec_{i+1}-2vtec_i+vtec_{i-1}$;
\item {\bf gvtec1} - diferença entre o VTEC de São José dos Campos e Pirassununga;
\item {\bf gvtec1\_dt} - diferença finita de primeira ordem no tempo do gvtec1, calculada por $gvtec1_i-gvtec1_{i-1}$;
\item {\bf gvtec2} - diferença entre o VTEC de São José dos Campos e Brasília;
\item {\bf gvtec2\_dt} - diferença finita de primeira ordem no tempo do gvtec2, calculada por $gvtec2_i-gvtec2_{i-1}$;
\item {\bf S4} - índice de cintilação ionosférico em São José dos Campos.
\end{itemize}

Nas definições acima de diferença finita, descartou-se valores constantes, pois eles seriam naturalmente eliminados pelo processo de normalização utilizado antes de aplicar os dados nos algoritmos de aprendizagem de máquina. 

O papel do notebook 06\_analise\_sj2.ipynb é o de construir as variáveis definidas, assim como o de concatenar tais variáveis em uma tabela que possa ser utilizada em algoritmos de aprendizagem de maquina para desenvolvimento de um modelo. Nas Figuras \ref{fig:scattervtec} até \ref{fig:scattergvtec2} é possível observar amostras das variáveis elaboras, enquanto na figura \ref{fig:vtec_vtec_dt}, tem-se uma representação gráfica da variável {\bf vtec} contra {\bf vtec\_dt}.

\begin{figure}[H]
\centering
\makebox[\textwidth][c]{\includegraphics[width=1.2\columnwidth]{./Figuras/vtec_scatter_mod.eps}}

\caption{Amostra dos sinais {\bf vtec} e {\bf S4}, no período entre 00:00 UT do dia 01/12/2013 até 12:00 UT do dia 04/12/2013. As linhas verticais vermelhas representam o por do sol, enquanto as amarelas representam o nascer do sol. Possíveis picos de pré reversão são indicados pelas setas.  Fonte: próprio autor.}
\label{fig:scattervtec}
\end{figure}


\begin{figure}[H]
\centering
\makebox[\textwidth][c]{\includegraphics[width=1.2\columnwidth]{./Figuras/vtec_dt_scatter.eps}}
\caption{Amostra dos sinais {\bf vtec\_dt} e {\bf S4}, no período entre 00:00 UT do dia 01/12/2013 até 12:00 UT do dia 04/12/2013. As linhas verticais vermelhas representam o por do sol, enquanto as amarelas representam o nascer do sol. Fonte: próprio autor.}
\label{fig:scattervtecdt}
\end{figure}

\begin{figure}[H]
\centering
\makebox[\textwidth][c]{\includegraphics[width=1.2\columnwidth]{./Figuras/vtec_dt2_scatter.eps}}
\caption{Amostra dos sinais {\bf vtec\_dt2} e {\bf S4}, no período entre 00:00 UT do dia 01/12/2013 até 12:00 UT do dia 04/12/2013. As linhas verticais vermelhas representam o por do sol, enquanto as amarelas representam o nascer do sol. Fonte: próprio autor.}
\label{fig:scattervtecdt2}
\end{figure}

\begin{figure}[H]
\centering
\makebox[\textwidth][c]{\includegraphics[width=1.2\columnwidth]{./Figuras/gvtec1_scatter.eps}}
\caption{Amostras dos sinais {\bf gvtec1} contra {\bf S4}, no período entre 00:00 UT do dia 01/12/2013 até 12:00 UT do dia 04/12/2013. As linhas verticais vermelhas representam o por do sol, enquanto as amarelas representam o nascer do sol. Fonte: próprio autor.}
\label{fig:scattergvtec1}
\end{figure}

\begin{figure}[H]
\centering
\makebox[\textwidth][c]{\includegraphics[width=1.2\columnwidth]{./Figuras/gvtec1_dt_scatter.eps}}
\caption{Amostras dos sinais {\bf gvtec1\_dt} e {\bf S4}, no período entre 00:00 UT do dia 01/12/2013 até 12:00 UT do dia 04/12/2013. As linhas verticais vermelhas representam o por do sol, enquanto as amarelas representam o nascer do sol. Fonte: próprio autor.}
\label{fig:scattergvtec1dt}
\end{figure}

\begin{figure}[H]
\centering
\makebox[\textwidth][c]{\includegraphics[width=1.2\columnwidth]{./Figuras/gvtec2_scatter.eps}}
\caption{Amostras dos sinais {\bf gvtec2} e {\bf S4}, no período entre 00:00 UT do dia 01/12/2013 até 12:00 UT do dia 04/12/2013. As linhas verticais vermelhas representam o por do sol, enquanto as amarelas representam o nascer do sol. Fonte: próprio autor.}
\label{fig:scattergvtec2}
\end{figure}

\begin{figure}[H]
\centering
\makebox[\textwidth][c]{\includegraphics[width=1.2\columnwidth]{./Figuras/gvtec2_dt_scatter.eps}}
\caption{Amostras dos sinais {\bf gvtec2\_dt} e {\bf S4}, no período entre 00:00 UT do dia 01/12/2013 até 12:00 UT do dia 04/12/2013. As linhas verticais vermelhas representam o por do sol, enquanto as amarelas representam o nascer do sol. Fonte: próprio autor.}
\label{fig:scattergvtec2dt}
\end{figure}

\begin{figure}[H]
\centering
\makebox[\textwidth][c]{\includegraphics[width=1.2\columnwidth]{./Figuras/vtec_and_vtec_dt.eps}}

\caption{Relação entre o {\bf vtec} e o {\bf vtec\_dt}, do dia 01/12/2013 até 05/12/2013. As linhas verticais vermelhas representam o por do sol, enquanto as amarelas representam o nascer do sol. Possíveis picos de pré reversão são indicados pelas setas. Os gráficos no meio apresentam normalização, para o intervalo $[0,1]$, enquanto os gráficos acima e abaixo apresentam seu valores reais.  Fonte: próprio autor.}
\label{fig:vtec_vtec_dt}
\end{figure}


\section{Análises: S4$\times$VTEC em São José dos Campos}

Utilizando as variáveis derivadas do vtec, realizou-se uma transformação de normalização para restringir o valor destas ao intervalo $[0,1]$. Os dados normalizados constituem um conjunto com 12.772 amostras ordenadas no tempo, destes os 772 últimos formaram um conjunto de validação, que foi utilizado para o cálculo do erro relativo médio do modelo. Os 12.000 elementos restantes foram divididos em dois conjuntos não-ordenadas num esquema de ``hold-out" aleatório (não ordenado no tempo), o de treinamento com 70\% dos dados e o de teste com 30\%. Em posse, de um conjunto de amostras de treinamento, segui-se para a avaliação dos estimadores, ou algoritmos de aprendizagem de máquina. Para este trabalho, testaram-se separadamente 3 algoritmos de aprendizagem de máquina, todos com a finalidade de estimar o S4 por meio de uma regressão, estes foram uma árvore, uma floresta aleatória e uma máquina de vetor de suporte. Cada estimador  foi avaliado utilizando ``10-fold cross validation". Os estimadores estão respectivamente nos notebooks 07\_analise\_sj2\_tree.ipynb, 07\_analise\_sj2\_random\_forest.ipynb e 07\_analise\_sj2\_svm.ipynb. Além disso, para todos estimadores realizou-se uma análise da sensibilidade aos atributos, removendo uma variável de cada vez, e verificando o desempenho de estimação/regressão para o novo estimador. 

O erro foi originalmente estimado utilizando o erro relativo médio, entretanto uma segunda abordagem utilizando uma matriz de confusão foi desenvolvida, para tal se considerou que quando para cada amostra no tempo, se o valor real e o estimado forem menores que um limiar tem-se verdadeiros negativos (TN); se o valor e o estimado forem maiores que o limiar tem-se um verdadeiro positivo (TV); se o valor real for maior que o do limiar e o estimado for menor, tem-se um falso negativo (FN); se o valor real for menor que o limiar enquanto o estimado maior, tem-se um falso positivo (FV). Na literatura, o limiar do índice S4 para ocorrência de cintilação é 0.2, isto é, todos os valores abaixo deste são tratados com ruídos e somente valores acima serão identificados como cintilação, portanto este valor foi adotado como limiar. Utilizando-se a matriz de confusão resultante, pode-se calcular a probabilidade de detecção (POD), a razão de falsos alarmes (FAR) e acurácia (ACC), cujas fórmulas são:

\begin{eqnarray}
\mbox{POD}&=&\mbox{TP}/(\mbox{TP}+\mbox{FN})\mbox{,}\\
\mbox{FAR}&=&\mbox{FP}/(\mbox{TP}+\mbox{FP})\mbox{,}\\ 
\mbox{ACC}&=&(\mbox{TN}+\mbox{TP})/(\mbox{TN}+\mbox{TP}+\mbox{FN}+\mbox{FP})\mbox{.}
\end{eqnarray}

Finalmente, realizou-se uma variação da análise de componentes principais (PCA - Principal Component Analsys), a qual objetiva reduzir o conjunto de atributos ou variáveis. A PCA é definida matematicamente como uma transformação linear ortogonal que mapeia os dados para um novo sistema de coordenadas por meio de uma sucessão rotações tal que a maior variância para qualquer projeção dos dados fique ao longo da primeira coordenada, a segunda maior variância fica ao longo da segunda coordenada, até que a menor variância fique para a última componente. Isso é feito a partir da decomposição em autovalores da matriz de covariância dos dados, ou por uma decomposição em valores singulares da matriz de dados. Uma decomposição baseada em autovalores consiste numa rotação no espaço multidimensional dos dados, que é o caso deste trabalho, utilizando-se o pacote ``psych'' do ambiente R. Entre diversas opções de rotação, foi utilizada a rotação ``varimax'', que busca uma nova base ortonormal, tal que cada vetor da nova base seja uma combinação linear de (preferivelmente) apenas alguns elementos da base original, ou seja, alguns pesos serão próximos de zero e outros próximos da unidade. Utilizando-se, então, a variância parcial de cada componente, tem-se uma medida indireta de quanto cada atributo é representativo no conjunto de atributos.

\section{Tamanho de passo para as diferenças finitas no tempo}\label{sec:tpdft}

É interessante observar o que acontece com a diferença finita no tempo quando são escolhidos diferentes passos, intervalos de tempo, para o cálculo. Tal análise é implementada pelo notebook 06\_analise\_fin\_dif\_vtec.ipynb, e consistiu em um laço que cálculo e armazenou as várias diferenças. Além disso, buscou-se um instante de referência com cintilação, para a construção dos gráficos comparativos, que fornecem o valor da derivada para o ponto de referência, com diferentes $\Delta{t}$. Tais gráficos também foram gerados para o {\bf gvtec1} e {\bf gvtec2}, os quais são apresentados na seção de resultados.

\section{Novas variáveis, novos testes}\label{sec:nvnt}

Utilizando os resultados da seção anterior \ref{sec:tpdft}, adicionou-se novas variáveis. O novo conjunto de variáveis é particionada em subconjuntos levando em consideração o momento de sua introdução, e sua formação. As variáveis definidas na seção \ref{sec:vdt} pertencem ao subconjunto das variáveis originais.

As variáveis pertencente ao subconjunto ``lag" correspondem à:

\begin{itemize}
\item {\bf gvtec1\_dt\_lag\_9} - diferença finita de primeira ordem no tempo do {\bf gvtec1} (diferença de VTEC entre São José dos Campos e Pirassununga) com passo 9 , calculada por $gvtec1_i-gvtec1_{i-9}$;
\item {\bf gvtec2\_dt\_lag\_20} - diferença finita de primeira ordem no tempo do {\bf gvtec2} (diferença de VTEC entre São José dos Campos e Brasília) com passo 20 , calculada por $gvtec2_i-gvtec2_{i-20}$.
\end{itemize}

Uma representação gráfica de uma amostra das variáveis {\bf gvtec1\_dt\_lag\_9} e {\bf gvtec2\_dt\_lag\_20} pode ser vista na figura \ref{fig:scattergvtec1_lag9} e  \ref{fig:scattergvtec2_lag_20}.

\begin{figure}[H]
\centering
\makebox[\textwidth][c]{\includegraphics[width=1.2\columnwidth]{./Figuras/gvtec1_lag_9_scatter.eps}}
\caption{Amostras dos sinais {\bf gvtec1\_dt\_lag\_9} e {\bf S4}, no período entre 12:00 UT do dia 01/12/2013 até 00:00 UT do dia 05/12/2013. As linhas verticais vermelhas representam o por do sol, enquanto as amarelas representam o nascer do sol. Fonte: próprio autor.}
\label{fig:scattergvtec1_lag9}
\end{figure}

\begin{figure}[H]
\centering
\makebox[\textwidth][c]{\includegraphics[width=1.2\columnwidth]{./Figuras/gvtec2_lag_20_scatter.eps}}
\caption{Amostras dos sinais {\bf gvtec2\_dt\_lag\_20} e {\bf S4}, no período entre 12:00 UT do dia 01/12/2013 até 00:00 UT do dia 05/12/013. As linhas verticais vermelhas representam o por do sol, enquanto as amarelas representam o nascer do sol. Fonte: próprio autor.}
\label{fig:scattergvtec2_lag_20}
\end{figure}

As variáveis pertencente ao subconjunto ``tempo" correspondem à:

\begin{itemize}
\item {\bf state\_day} - indica que uma amostra está entre o nascer do sol e o por do sol;
\item {\bf state\_night} - indica que uma amostra está entre o por do sol e 00:00 UT;
\item {\bf state\_dawn} - indica que uma amostra está entre as 00:00 UT e o nascer do sol.
\end{itemize}

Cada um dessas variáveis pode assumir o valor 0.0 ou 1.0, a primeira indica falso e a segunda verdadeiro. Para sua avaliação uma amostra é tem seu índice decomposto em uma parte associada a data e a outra a hora, então usando a data e a localização da estação que forneceu a medida se determina o horário de nascer e por do sol, a última etapa é avaliação da parte das horas em relações as condições que definem cada variável, isto é, por exemplo, testa-se se a hora está entre o nascer e o por do sol.

As variáveis pertencente ao subconjunto ``mdv1" correspondem à:

\begin{itemize}
\item {\bf vm1} - valor médio do VTEC calculado entre o nascer e o por do sol;
\item {\bf vd1} - desvio padrão do VTEC calculado entre o nascer e o por do sol.
\end{itemize}

Os valores de {\bf vm1} e {\bf vd1} tem validade até o nascer do sol do dia seguinte, portanto, as amostras neste intervalo intermediário também irão assumir este valor.

As variáveis pertencente ao subconjunto ``mdv2" correspondem à:

\begin{itemize}
\item {\bf vm2} - valor médio do VTEC calculado entre o nascer e o por do sol mais uma hora;
\item {\bf vd2} - desvio padrão do VTEC calculado entre o nascer e o por do sol mais uma hora.
\end{itemize}

De modo análogo as variáveis {\bf vm1} e {\bf vd1} as amostras nos intervalos intermediário irão assumir o valor de {\bf vm2} e {\bf vd2}.

Também foi definida uma variável denominada {\bf vtec\_dt\_lag\_3} que correspondem a uma diferença finita com passo 3 do {\bf vtec}, dada por $vtec_i-vtec_{i-3}$;

Estas novas variáveis são implementadas no notebook 08\_update\_analise\_sj2\_data.ipynb. Em, posse dos subconjuntos que agrupam essas variáveis, o próximos passo foi a realização de uma bateria de testes, onde foram utilizadas diferentes uniões deste subconjuntos, assim como algumas variáveis do subconjunto original de maneira independente, independente da configuração o algoritmo adotado foi a floresta aleatória. Os testes estão implementados no notebook 09\_all\_analise.ipynb e os todos os resultados, isto é, estimações dos valores de S4 serão apresentados em uma tabela.
