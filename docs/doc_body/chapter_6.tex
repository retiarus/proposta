\chapter{CONCLUSÃO}

Este trabalho busca correlacionar o índice de cintilação S4 com o conteúdo eletrônico total vertical VTEC e suas variáveis derivadas, de forma a prover subsídios para uma possível futura estimação de S4 a partir destas variáveis e de outras relacionadas ao estado da ionosfera terrestre, não apresentadas aqui. Estudos desenvolvidos anteriormente para realizar essa estimação \cite{REZENDE:2009, GLAUSTON:2014, GLAUSTON:2015} poderiam então ser revistos e ampliados com a inclusão do VTEC e suas variáveis derivadas. Isso justifica-se uma vez que a ocorrência de cintilação é causada por zonas de rarefação na ionosfera, denominadas bolhas ionosféricas, nas quais o valor de VTEC é obviamente baixo.

Analisar a evolução espaço-temporal das bolhas ionosféricas, como previsto inicialmente, não foi possível, pois demandaria uma rede geograficamente muito mais densa de estações de medição de S4 e de VTEC. Entretanto, foi possível correlacionar o índice S4 com o VTEC e suas variáveis derivadas, adotando-se uma estratégia de tentar estimar S4 a partir dessas variáveis por meio de uma técnica de regressão. Esta análise foi efetuada para um estudo de caso correspondente ao intervalo de tempo de 3 meses em um período de atividade magnética calma para a cidade de São José dos Campos, sendo avaliadas 3 técnicas de regressão, floresta aleatória, árvore de regressão CART e máquina de vetor de suporte, sendo a primeira a que apresentou melhor desempenho. Consequentemente o conjunto completo de testes de regressão para várias combinações de variáveis foi efetuado utilizando-se a floresta aleatória.

Concluiu-se que as variáveis avaliadas tem importância similar na estimação do S4 e que portanto podem ser consideradas correlacionadas com esse índice, conforme demonstrado nesse estudo de caso. Consequentemente, tais variáveis tem uso potencial numa proposta futura de predição de ocorrência de cintilação.

À continuação deste trabalho, pretende-se estender os testes aumentado-se o conjunto de amostras para incluir noites sem ocorrência de cintilação, refinar os parâmetros relativos ás variáveis derivadas do VTEC para otimizar a regressão, e repetir todo o estudo para um período de atividade magnética perturbado. Espera-se que o refinamento desses parâmetros possa melhorar muito o desempenho da regressão, uma vez que seria novamente guiada pelo conhecimento de especialistas da área. Em particular, pretende-se identificar automaticamente, a partir de dados do VTEC, a ocorrência do pico da pré reversão. Outras possibilidades são relativas ao uso de algoritmos ainda não testados, dada a diversidade de algoritmos de aprendizado de máquina disponíveis no ambiente Python. Em relação ao rastreamento espaço-temporal das bolhas ionosféricas, pretende-se obter mais dados e possivelmente utilizar a técnica de ``kriging", que permitiria estimar, no caso, o campo de VTEC sobre parte do território brasileiro, a partir das medidas de um conjunto de estações utilizando interpolação.
