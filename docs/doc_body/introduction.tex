\chapter{INTRODUÇÃO}

A necessidade de localização espacial levou a humanidade ao desenvolvimento de diversas ferramentas, tais como os sistemas de coordenadas, a bússola, os mapas e, mais recentemente, o Sistema de Posicionamento Global (GPS). Este sistema, desenvolvido pelos norte americanos, se tornou completamente operacional em 1995, com um custo estimado de 10 bilhões de dólares. Consiste de uma constelação de 24 satélites, cada um circulando a Terra duas vezes por dia, em uma configuração em que ao menos 4 satélites sejam visíveis de qualquer ponto da Terra. O receptor do sinal utiliza a informação enviada pelo satélite para calcular sua distância a cada um destes utilizando a informação entre o instante de recebimento e o instante de transmissão. 

Em 1991, a Organização da Aviação Civil Internacional (OACI) utilizou pela primeira vez o termo sistema de navegação por satélite (GNSS - Global Navigation Satellite System), para denominar todo e qualquer sistema semelhante ao GPS, que atualmente é usado para denominar o sistema americano, também conhecido como Navstar GPS. Outro sistema que se encontra completamente operacional é o russo GLONASS. O sistema chinês COMPASS e o europeu GALILEO se encontram em fase de implementação. Considerando a diversidade de sistemas é possível notar sua grande relevância. Estes sistemas de GNSS estão sujeitos a pertubações e interferências, sendo que dentre elas a mais relevante é a cintilação ionosférica.

A cintilação ionosférica, medida por meio do índice S4, é definida como o desvio padrão da intensidade do sinal de GPS em um intervalo de um minuto, com 50 amostras por segundo, portanto, quanto maior a pertubação do sinal, maior o valor deste índice. Uma observação que pode ser extraída desta definição, é que dada uma causa para a perturbação do sinal de GPS, é possível que outros sinais de radiofrequência, em regiões relativamente próximas do espectro apresentam interferência semelhante, logo, o índice S4 pode ser interpretado como uma medida de pertubação em sinais de radiofrequência.

Os fenômenos de cintilação, de interesse para este trabalho, decorrem da trajetória do sinal percorrer regiões da ionosfera com baixa densidade de elétrons, com períodos de duração de algumas horas. Estas regiões migram na ionosfera, geralmente no sentido do Sul e do Leste magnético, podendo se expandir ou contrair, sendo denominadas de bolhas ionosféricas. Estas surgem no Hemisfério Sul e sua ocorrência começa a se intensificar de outro a novembro, apresentado picos ao longo do verão, a atividade começa a reduzir em Março. Sua formação começa entre 22:00 UT e 23:00 UT na região do equador magnético, encerrando-se entre as 04:00 e 05:00 UT. Além da atividade mais intensa durante o verão é possível observar uma grande variabilidade dia a dia no decorrer do ano, o que torna difícil sua previsão, \cite{TAKAHASHI:2006}.

A ionosfera apresenta uma riqueza de fenômenos devido a vários fatores, sendo influenciada não somente pelo campo magnético da Terra. Este campo, por sua vez, sofre a influência de todo o sistema solar, principalmente do Sol, que é a principal fonte de radiação ionizante e que rege o clima espacial e os campos magnético e elétrico no espaço, os quais influem diretamente no campo magnético terrestre. Devido à inclinação do plano de rotação da Terra em relação ao seu plano de translação, surgem anomalias em certas regiões desse campo magnético. Dentre estas, uma de particular interesse, é a anomalia da ionização equatorial (EIA) que consiste na formação de uma região de alta densidade de elétrons entre 15 e 20 graus de latitude magnética, logo após ao por do Sol. O máximo dessa densidade ocorre no denominado de pico da anomalia magnética (EIA), localizado vale do Paraíba, estado de São Paulo.

Atualmente, não existe um modelo matemático baseado em primeiros princípios, que expresse a física do fenômeno, capaz de modelar e predizer o surgimento e a evolução das bolhas, de forma, que a predição do decorrente fenômeno de cintilação também não é possível. Todavia, os impactos decorrentes desse fenômeno, como a perda de sistemas de navegação, em uma sociedade que apresenta cada vez mais um consumo por este tipo de informação, seja em sistemas de produção como na agricultura \cite{STAFFORD:2000}, ou na aviação, ou para o simples uso pessoal ao percorrer uma cidade, podem ser catastróficas, implicando, por exemplo, em perdas de vidas humanas, ou na redução na produção de alimentos. No caso da aviação, há uma tendência mundial de se utilizar unicamente navegação e procedimentos de pouso/decolagem baseados em GNSS.

Assim, faz-se desejável uma abordagem que permita prever a formação, a evolução das bolhas ionosféricas e intensidade da cintilação dada pelo índice S4. Este índice somente fornece informação indireta sobre a bolha, uma vez que é uma medida da qualidade do sinal de GPS, e este pode sofrer interferências de outras origens. Assim, outras medidas quantitativas são desejáveis para rastrear ou monitorar bolhas ionosféricas. Uma delas é o conteúdo eletrônico total vertical (VTEC - Vertical Total Electronic Content), definido como a quantidade de elétrons, em uma seção vertical da atmosfera de área unitária, cuja unidade de medida é o TECU$=10^{16}$ elétrons/m$^2$. Segundo a literatura, bolhas apresentam valores de VTEC de 30 a 50 TECU inferiores à sua vizinhança.

Na ausência de um modelo que possa simular um fenômeno tão complexo, um enfoque possível é a mineração de dados, área da computação que permite inferir conhecimento a partir de uma massa de dados específica de um fenômeno ou evento qualquer.

Dentro do escopo mais ambicioso de predição da ocorrência de cintilação no pico da anomalia magnética a partir de dados observacionais coletados por estações de monitoramento no território brasileiro, o objetivo deste trabalho se restringe à utilização de técnicas de mineração e visualização de dados históricos da ionosfera, de forma a analisar a dinâmica espaço-temporal das bolhas ionosféricas causadoras da cintilação, ao longo de um meridiano magnético que passa pelo pico da anomalia. Tal analise é realizada buscando um conjunto de atributos que permita predizer o valor de S4 com base no VTEC. A estimativa de S4 por meio do VTEC é uma das maneiras de buscar e observar correlações entre essas variáveis, que serão utilizadas futuramente para prever a ocorrência do S4. 

Esta pesquisa complementa trabalhos anteriores relacionados à predição da ocorrência de cintilação \cite{REZENDE:2009, GLAUSTON:2014, GLAUSTON:2015}, além de trabalhos relacionados ao objetivo específico aqui abordado, como por exemplo a correlação entre o gradiente do TEC e a cintilação \cite{RAGHAVARAO:1998, RAY:2006}, ou então a correlação entre a derivada temporal do TEC e a cintilação \cite{RAGHUNATH:2016}. Estes dois últimos trabalhos sugerem possíveis indicações de variáveis que podem ser utilizadas para mapear o VTEC no S4, o primeiro levou a definição da diferença espacial entre o VTEC de São José dos Campos e Pirassununga, e o VTEC de São José dos Campos e Brasília, enquanto o segundo levou a definição da diferença no tempo para o VTEC e para as diferenças espaciais, juntamente do próprio VTEC, estes são os atributos analisados em relação ao S4.

O capítulo 2 deste trabalho discute sobre a ionosfera, concentrado-se nos pontos principais necessários para entender a associação de bolhas ionosféricas e cintilações ionosféricas, assim como a definição das variáveis base deste trabalho, S4 e VTEC. O capítulo 3 apresenta uma introdução à mineração de dados e descoberta de conhecimento em base de dados, seguindo com uma discussão das técnicas de mineração utilizadas neste trabalho. O capítulo 4 desenvolve a metodologia utilizada neste trabalho, iniciando pela etapa de seleção inicial dos dados, pré-processamento até aplicação dos algoritmos de mineração. O capítulo 5 contém uma descrição dos testes realizados, assim como uma análise dos resultados obtidos. O capítulo 6 apresenta as conclusões finais alcançadas e estabelece metas almejadas para futuros desenvolvimentos.

