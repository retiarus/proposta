\chapter{INTRODUÇÃO}

A necessidade de localização espacial levou a humanidade ao desenvolvimento de diversas ferramentas, tais como os sistemas de coordenadas, a bússola, os mapas e, mais recentemente, o Sistema de Posicionamento Global (GPS). Este sistema, desenvolvido pelos norte americanos, se tornou completamente operacional em 1995, com um custo estimado de 10 bilhões de dólares. Consiste de uma constelação de 24 satélites, cada um circulando a Terra duas vezes por dia, em uma configuração em que ao menos 4 satélites sejam visíveis de qualquer ponto da Terra. O receptor do sinal utiliza a informação enviada pelo satélite para calcular sua distância a cada um destes utilizando a informação entre o instante de recebimento e o instante de transmissão. 

Em 1991, a Organização da Aviação Civil Internacional (OACI) utilizou pela primeira vez o termo sistema de navegação por satélite (GNSS - Global Navigation Satellite System), para denominar todo e qualquer sistema semelhante ao GPS, que atualmente é usado para denominar o sistema americano, também conhecido como Navstar GPS. Outro sistema que se encontra completamente operacional é o russo GLONASS. O sistema chinês COMPASS e o europeu GALILEO se encontram em fase de implementação. Considerando a diversidade de sistemas é possível notar sua grande relevância. Estes sistemas de GNSS estão sujeitos a pertubações e interferências, sendo que dentre elas a mais relevante é a cintilação ionosférica.

A cintilação ionosférica, medida por meio do índice $S_4$, é definida como o desvio padrão da intensidade do sinal de GPS em um intervalo de um minuto, com 50 amostras por segundo, portanto, quanto maior a pertubação do sinal, maior o valor deste índice. Uma observação que pode ser extraída desta definição, é que dada uma causa para a perturbação do sinal de GPS, é possível que outros sinais de radiofrequência, em regiões relativamente próximas do espectro apresentam interferência semelhante, logo, o índice S4 pode ser interpretado como uma medida de pertubação em sinais de radiofrequência.

Os fenômenos de cintilação, de interesse para este trabalho, decorrem da trajetória do sinal percorrer regiões da ionosfera com baixa densidade de elétrons, com períodos de duração de algumas horas. Estas regiões migram na ionosfera, geralmente no sentido do Sul e do Leste magnético, podendo se expandir ou contrair, sendo denominadas de bolhas ionosféricas. Estas surgem na região do equador magnético e sua ocorrência começa a se intensificar de outubro a novembro, apresentado picos ao longo do verão, a atividade começa a reduzir em Março. Sua formação começa entre 19-20 LT (22-23 UT), encerrando-se em sua maioria entre as 01-02 LT (04-05 UT). Além da atividade mais intensa durante o verão é possível observar uma grande variabilidade dia a dia no decorrer do ano, o que torna difícil sua previsão, \cite{TAKAHASHI:2006}.

A ionosfera apresenta uma riqueza de fenômenos devido a vários fatores, sendo influenciada não somente pelo campo magnético da Terra. Este campo, por sua vez, sofre a influência de todo o sistema solar, principalmente do Sol, que é a principal fonte de radiação ionizante e que rege o clima espacial e os campos magnético e elétrico no espaço. Um fenômeno de particular interesse, é a anomalia da ionização equatorial (EIA) que consiste na formação de uma região de alta densidade de elétrons entre 15 e 20 graus de latitude magnética, logo após ao por do Sol.

Atualmente, não existe um modelo matemático baseado em primeiros princípios, que expresse a física do fenômeno, capaz de modelar e predizer o surgimento e a evolução das bolhas, de forma, que a predição do decorrente fenômeno de cintilação também não é possível. Todavia, os impactos decorrentes desse fenômeno, como a perda de sistemas de navegação, em uma sociedade que apresenta cada vez mais um consumo por este tipo de informação, seja em sistemas de produção como na agricultura \cite{STAFFORD:2000}, ou na aviação, ou para o simples uso pessoal ao percorrer uma cidade, podem ser catastróficas, implicando, por exemplo, em perdas de vidas humanas, ou na redução na produção de alimentos. No caso da aviação, há uma tendência mundial de se utilizar unicamente navegação e procedimentos de pouso/decolagem baseados em GNSS.

Assim, faz-se desejável uma abordagem que permita prever a formação, a evolução das bolhas ionosféricas e intensidade da cintilação dada pelo índice S4. Este índice somente fornece informação indireta sobre a bolha, uma vez que é uma medida da qualidade do sinal de GPS, e este pode sofrer interferências de outras origens.

Na ausência de um modelo que possa simular um fenômeno tão complexo, um enfoque possível é a mineração de dados, área da computação que permite inferir conhecimento a partir de uma massa de dados específica de um fenômeno ou evento qualquer. Naturalmente a aplicação de tais técnicas pode apresentar grande resistência por parte da comunidade de clima espacial, haja visto um ponto principal: apesar de varias das técnicas de aprendizagem de máquina terem sido introduzidas nos anos 90, somente com o avanço da computação na última década frutos puderam ser coletados, como por exemplo, o grande sucesso da aplicação de redes convolucionais para análise de imagens, e apesar dos resultados promissores pouco foi feito no contexto de clima espacial, por exemplo, técnicas como redes convolucionais quase não foram aplicadas neste contexto \cite{CAMPOREALE:2019}.

Dentro os diversos problemas em clima espacial onde aprendizado de máquina já foi aplicado ao menos em alguma quantidade, o problema de predição de cintilação ionosférica é o objeto de estudo desta proposta. O primeiro trabalho neste tópico utilizando mineração de dados foi apresentado em \cite{REZENDE:2009, REZENDE:2010}. Estes modelos baseados em dados apresentam dependência em relação a localização da coleta de dados, por isso, é mais adequado dizer que o trabalho \cite{REZENDE:2009, REZENDE:2010} foi pioneiro na região de São José dos Campos, pois outros trabalhos também alegam pioneirismo como \cite{MCGRANAGHAN:2018}, porém neste caso para a região do polo norte. Um ponto interessante a observar aqui é que este último data de 2018, o que junto de sua proposição fornece evidencia de que existe muito a ser explorado neste problema. Outros dois trabalhos relevantes na predição de cintilação ionosférica são \cite{GLAUSTON:2014} e \cite{GLAUSTON:2015}. O primeiro tinha como objetivo correlacionar a cintilação de São José dos Campos com São Luiz, enquanto o prever a cintilação em São Luiz, observando dados anteriores nesta localização.

Nesta tese, pretende-se estender as abordagens de predição de ocorrência de cintilação existentes, do autor ou do grupo de pesquisa associado, de forma a se obter uma abordagem robusta o suficiente com o objetivo de permitir que seja empregada operacionalmente pelo programa de Clima Espacial do INPE para a predição com antecedência horária ou maior. A contribuição da tese seria constituída pela análise e seleção de atributos de informação e pela escolha, análise e seleção do(s) algoritmo(s) de predição(s) para atingir esse objetivo, considerando o ineditismo de se obter uma abordagem robusta para esse fim e a dificuldade inerente ao problema.

Esta proposta é composta pelos seguintes capítulos:

\begin{itemize}
\item {\bf Capítulo 2: Ionosfera}, apresenta a estrutura da ionosfera em termos de camadas, assim como uma ideia geral dos mecanismos de dínamos presentes nestas responsáveis pela formação da anomalia de ionização equatorial. Também neste capítulo é discutido as variáveis utilizadas neste trabalho, e o fenômeno de cintilação ionosférica;
\item {\bf Capítulo 3: Mineração de dados}, são tratados os principais fundamentos sobre mineração de dados e o processo de descoberta de dados em base de dados (KDD). Seguido de uma discussão sobre algoritmos de aprendizagem baseado em árvores e suas extensões por meio da aplicação das técnicas de ensemble. Finalmente, apresentado a base teórica do algoritmo XGBoost;
\item {\bf Capítulo 4: Predição de cintilação ionosférica com antecedência horária}, primeiramente é feita uma reprodução parcial do problema de predição de cintilação ionosférica com antecedência horária apresentado no trabalho \cite{REZENDE:2009} (problema de regressão), seguida se uma adaptação em um problema de classificação. Posteriormente, pela adição de novos atributos preditores e a revisão de outros o mesmo problema é tratado, para então generalizar para o território brasileiro;
\item {\bf Capítulo 5: Conteúdo eletrônico total}, apresenta os resultados obtidos na qualificação atualizados quando da procura de correlações entre os VTEC e variáveis derivadas e o índice de cintilação $S_4$;
\item {\bf Capítulo 6: Proposta}, é apresentado o objetivo da tese e as etapas e passos futuros do trabalho segundo este.
\end{itemize}
