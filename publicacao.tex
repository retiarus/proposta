\documentclass[%
%%% PARA ESCOLHER O ESTILO TIRE O SIMBOLO %(COMENTÁRIO)
%SemVinculoColorido,
%SemFormatacaoCapitulo,
SemFolhaAprovacao,
%SemImagens,
%CitacaoNumerica, %% o padrão é citação tipo autor-data
%PublicacaoDissOuTese, %% (é também o "default") com ficha catal. e folha de aprovação em branco. Caso tenha lista de símbolos e lista de siglas e abreviaturas retirar os comentários dos arquivos siglas.tex e abreviaturasesiglas.tex. Retirar também os comentários indicados nesse arquivo, nos includes
%PublicacaoArtigoOuRelatorio, %% texto sequencial, sem quebra de páginas nem folhas em branco
PublicacaoProposta, %% igual tese/dissertação, mas sem ficha catal. e fol. de aprov.
%PublicacaoLivro, %% com capítulos
%PublicacaoLivro,SemFormatacaoCapitulo, %% sem capítulos
english,portuguese %% para os documentos em Português com abstract.tex em Inglês
%portuguese,english %% para os documentos em Inglês com abstract.tex em Português
,LogoINPE% comentar essa linha para fazer aparecer o logo do Governo
,CCBYNC	% as opções de licença são: CCBY, CCBYSA, CCBYND, CCBYNC, CCBYNCSA, CCBYNCND, GPLv3 e INPECopyright
]{tdiinpe}
%]{../../../../../iconet.com.br/banon/2008/03.25.01.19/doc/tdiinpe}

% PARA EXIBIR EM ARIAL TIRAR O COMENTÁRIO DAS DUAS LINHAS SEGUINTES
%\renewcommand{\rmdefault}{phv} % Arial
%\renewcommand{\sfdefault}{phv} % Arial

% PARA PUBLICAÇÕES EM INGLÊS:
% renomear o arquivo: abnt-alf.bst para abnt-alfportuguese.bst
% renomear o arquivo: abnt-alfenglish.bst para abnt-alf.bst


%%%%%%%%%%%%%%%%%%%%%%%%%%%%%%%%%%%%%%%%%%%%%
%%% Pacotes já previamente carregados:      %
%%%%%%%%%%%%%%%%%%%%%%%%%%%%%%%%%%%%%%%%%%%%%%%%%%%%%%%%%%%%%%%%%%%%%%%%
%%% ifthen,calc,graphicx,color,inputenc,babel,hyphenat,array,setspace, %
%%% bigdelim,multirow,supertabular,tabularx,longtable,lastpage,lscape, %
%%% rotate,caption2,amsmath,amssymb,amsthm,subfigure,tocloft,makeidx,  %
%%% eso-pic,calligra,hyperref,ae,fontenc                               %
%%%%%%%%%%%%%%%%%%%%%%%%%%%%%%%%%%%%%%%%%%%%%%%%%%%%%%%%%%%%%%%%%%%%%%%%
%%% insira neste campo, comandos de LaTeX %%%
%%% \usepackage{_exemplo_}
% etc.
%%%%%%%%%%%%%%%%%%%%%%%%%%%%%%%%%%%%%%%%%%%%%

%\watermark{Revisão No. ##} %% use o comando \watermark para identificar a versão de seu documento
%% comente este comando quando for gerar a versão final
%\usepackage{subcaption}
\usepackage{braket}
\usepackage{adjustbox}
\usepackage{rotating}
\usepackage{hyperref}
\usepackage{float}
\usepackage{svg}
\usepackage{epstopdf}
\usepackage{rotating}
\usepackage{dsfont}
\usepackage{comment}
\usepackage{algorithm}
\usepackage[noend]{algpseudocode}
\usepackage{tikz}
\usetikzlibrary{shapes.geometric, arrows}
%%%%%%%%%%%%%%%%%%%CAPA%%%%%%%%%%%%%%%%%%%%%%%%%%%%%%%%
%\serieinpe{INPE-NNNNN-TDI/NNNN} %% não mais usado

\titulo{XXXXXXXXXXX}
\title{Escrever o título em Inglês para publicações escritas em Português e em Português para publicações escritas em Inglês} %%
\author{Pedro Alexandre dos Santos} %% coloque o nome do(s) autor(es)
\descriccao{Monografia para Exame de Proposta do Curso de Pós-Graduação em Computação Aplicada, orientada pelo Dr. Stephan Stephany, aprovada em xxxxxx de agosto de 2019.}
\repositorio{aa/bb/cc/dd} %% repositório onde está depositado este documento - na omissão, será preenchido pelo SID
\tipoDaPublicacao{TDI}	%% tipo da publicação (NTC, RPQ, PRP, MAN, PUD, TDI, TAE e PRE) na ausência do número de série INPE, caso contrário deixar vazio
\IBI{xx/yy} %% IBI (exemplo: J8LNKAN8PW/36CT2G2) quando existir, caso contrário o nome do repositório onde está depositado o documento

\date{2019}%ano da publicação

%%%%%%%%%%%%%%%%%%%%%%%%%%VERSO DA CAPA%%%%%%%%%%%%%%%%%%%%%%%%%%%%%%%%%%%%%%%%%%%%%%%
\tituloverso{\vspace{-0.9cm}\textbf{\PublicadoPor:}}
\descriccaoverso{Instituto Nacional de Pesquisas Espaciais - INPE\\
Gabinete do Diretor (GB)\\
Serviço de Informação e Documentação (SID)\\
Caixa Postal 515 - CEP 12.245-970\\
São José dos Campos - SP - Brasil\\
Tel.:(012) 3945-6923/6921\\
Fax: (012) 3945-6919\\
E-mail: {\url{pubtc@sid.inpe.br}}
}

\descriccaoversoA{\textbf{\ConselhoDeEditoracao:}\\
\textbf{\Presidente:}\\
Marciana Leite Ribeiro - Serviço de Informação e Documentação (SID)\\
\textbf{\Membros:}\\
Dr. Gerald Jean Francis Banon - Coordenação Observação da Terra (OBT)\\
Dr. Amauri Silva Montes - Coordenação Engenharia e Tecnologia Espaciais (ETE)\\
Dr. André de Castro Milone - Coordenação Ciências Espaciais e Atmosféricas (CEA)\\
Dr. Joaquim José Barroso de Castro -  Centro de Tecnologias Espaciais (CTE)\\
Dr. Manoel Alonso Gan - Centro de Previsão de Tempo e Estudos Climáticos (CPT)\\
Drª Maria do Carmo de Andrade Nono - Conselho de Pós-Graduação\\
Dr. Plínio Carlos Alvalá - Centro de Ciência do Sistema Terrestre (CST)\\
\textbf{\BibliotecaDigital:}\\
Dr. Gerald Jean Francis Banon - Coordenação de Observação da Terra (OBT)\\
Clayton Martins Pereira - Serviço de Informação e Documentação (SID)\\
%Jefferson Andrade Ancelmo - Serviço de Informação e Documentação (SID)\\
%Simone A. Del-Ducca Barbedo - Serviço de Informação e Documentação (SID)\\
%Deicy Farabello - Centro de Previsão de Tempo  e Estudos Climáticos (CPT)\\
\textbf{\RevisaoNormalizacaoDocumentaria:}\\
Simone Angélica Del Ducca Barbedo - Serviço de Informação e Documentação (SID) \\
%Marilúcia Santos Melo Cid - Serviço de Informação e Documentação (SID)\\
Yolanda Ribeiro da Silva Souza - Serviço de Informação e Documentação (SID)\\
\textbf{\EditoracaoEletronica:}\\
Marcelo de Castro Pazos - Serviço de Informação e Documentação (SID)\\
André Luis Dias Fernandes - Serviço de Informação e Documentação (SID)\\
}

%%%%%%%%%%%%%%%%%%%FOLHA DE ROSTO

%%%%%%%%%%%%%%%FICHA CATALOGRÁFICA
% NÃO PREENCHER - SERÁ PREENCHIDO PELO SID

\cutterFICHAC{Cutter}
\autorUltimoNomeFICHAC{Sobrenome, Nomes} %% exemplo: Fuckner, Marcus André
\autorFICHAC {Nome Completo do Autor1; Nome Completo do Autor2} %% Campo opcional (se não usado prevalece \author)
\tituloFICHAC{Titulo da publicação}
\instituicaosigla{INPE}
\instituicaocidade{São José dos Campos}
\paginasFICHAC{\pageref{numeroDePáginasDoPretexto} + \pageref{LastPage}} %% número total de páginas
%\serieinpe{INPE-00000-TDI/0000} %% não mais usado
\palavraschaveFICHAC{1.~Palavra chave. 2.~Palavra chave 3.~Palavra chave. 4.~Palavra chave. 5.~Palavra chave  I.~\mbox{Título}.} %% recomenda-se pelo menos 5 palavras-chaves - \mbox{} é para evitar hifenização
\numeroCDUFICHAC{000.000} %% número do CDU

% Nota da ficha (para TD)
\tipoTD{Dissertação ou Tese} % Dissertação ou Tese
\cursoFA{Mestrado ou Doutorado em Nome do Curso}
\instituicaoDefesa{Instituto Nacional de Pesquisas Espaciais}
\anoDefesa{AAAA} % ano de defesa
\nomeAtributoOrientadorFICHAC{Orientador}	% pode ser: Orientador, Orientadora ou Orientadores
\valorAtributoOrientadorFICHAC{José da Silva} % nome(s) completo(s)

%%%%%%%%%%%%%%%FOLHA DE APROVAÇAO PELA BANCA EXAMINADORA
\tituloFA{\textbf{ATENÇÃO! A FOLHA DE APROVAÇÃO SERÁ INCLUIDA POSTERIORMENTE.}}
%\cursoFA{\textbf{}}
\candidatoOUcandidataFA{}
\dataAprovacaoFA{}
\membroA{}{}{}
\membroB{}{}{}
\membroC{}{}{}
\membroD{}{}{}
\membroE{}{}{}
\membroF{}{}{}
\membroG{}{}{}
\ifpdf

%%%%%%%%%%%%%%NÍVEL DE COMPRESSÃO {0 -- 9}
\pdfcompresslevel 9
\fi
%%% define em 80% a largura das figuras %%%
\newlength{\mylenfig}
\setlength{\mylenfig}{0.8\textwidth}
%%%%%%%%%%%%%%%%%%%%%%%%%%%%%%%%%%%%%%%%%%%

%%%%%%%%%%%%%%COMANDOS PESSOAIS
\newcommand{\vetor}[1]{\mathit{\mathbf{#1}}}
 %% faça as modificações pertinentes no arquivo configuracao.tex

\makeindex  %% não alterar, gera INDEX, caso haja algum termo indexado no texto

\begin{document} %% início do documento %% não mexer

%\marcaRegistrada{}	% comando opcional usado para informar abaixo da ficha catalográfica sobre marca registrada
\marcaRegistrada{Informar aqui sobre marca registrada (a modificação desta linha deve ser feita no arquivo publicacao.tex).}

\maketitle  %% não alterar, gera páginas obrigatórias (folha de rosto, ficha catalográfica e folha de aprovação), automaticamente

%%% Comente as linhas opcionais abaixo caso não as deseje
%%%%%%%%%%%%%%%%%%%%%%%%%%%%%%%%%%%%%%%%%%%%%%%%%%%%%%%%%%%%%%%%%%%%%%%%%%%%%%%%%
% Epígrafe %% opcional

\begin{epigrafe} %% insira sua epígrafe abaixo; estilo livre

\hypertarget{estilo:epigrafe}{} %% uso para este Guia
 
\textit{\large``A vida será mais complicada se você possuir uma curiosidade ativa, além de aumentarem as chances de você entrar em apuros, mas será mais divertida''.}

\vspace{1cm}

\hspace{4cm} \emph{\textsc{Edward Speyer}}\\\hspace{4cm} em \textsl{``Seis Caminhos a Partir de Newton''}, 1994

\end{epigrafe}
 %% Opcional
%%%%%%%%%%%%%%%%%%%%%%%%%%%%%%%%%%%%%%%%%%%%%%%%%%%%%%%%%%%%%%%%%%%%%%%%%%%%%%%%%
% Dedicatória %% opcional

\begin{dedicatoria} %% insira sua dedicatória abaixo; estilo livre

\hypertarget{estilo:dedicatoria}{} %% uso para este Guia
 
%% use 'a meus' em vez de 'aos meus', isto é, não use o artigo definido com pronomes possessivos

\newcommand{\mytext}{A meus pais \textbf{Nicanor} e \textbf{Jaci}, à minha irmã \textbf{Luciana} e ao meu esposo \textbf{William}}

\begin{comment}
%%% sugestão de estilo
\ifcalligra %% fonte calligra presente nas versões mais novas do MiKTeX (>= 2.4)
  \calligra\Large \mytext %% exemplo usando estilo de fonte caligráfica, caso haja
\else
	\itshape\Large \mytext 
\fi
\end{comment}

	\itshape\Large \mytext 

\end{dedicatoria} %% Opcional
%%%%%%%%%%%%%%%%%%%%%%%%%%%%%%%%%%%%%%%%%%%%%%%%%%%%%%%%%%%%%%%%%%%%%%%%%%%%%%%%%
% AGRADECIMENTOS %% opcional

\begin{agradecimentos}  %% insira abaixo seus agradecimentos

\hypertarget{estilo:agradecimentos}{} %% uso para este Guia
Agradecemos à MsC Andriana Susana Lopes de Oliveira Campanharo que gentilmente cedeu 
parte dos textos de sua dissertação para este estilo. O original de sua dissertação
encontra-se na Biblioteca Digital do INPE, no endereço \url {http://urlib.net/sid.inpe.br/MTC-m13@80/2006/11.07.12.37}.
Agradecemos também ao Dr. Gerald Jean Francis Banon pelo desenvolvimento e disponibilização deste estilo.
\end{agradecimentos}


 %% Opcional
%%%%%%%%%%%%%%%%%%%%%%%%%%%%%%%%%%%%%%%%%%%%%%%%%%%%%%%%%%%%%%%%%%%%%%%%%%%%%%%%
% RESUMO %% obrigatório

\begin{resumo}

%% neste arquivo resumo.tex
%% o texto do resumo e as palavras-chave têm que ser em Português para os documentos escritos em Português
%% o texto do resumo e as palavras-chave têm que ser em Inglês para os documentos escritos em Inglês
%% os nomes dos comandos \begin{resumo}, \end{resumo}, \palavraschave e \palavrachave não devem ser alterados

\hypertarget{estilo:resumo}{} %% uso para este Guia

%A ionosfera é uma camada da atmosfera que se estende de aproximadamente 60 km a 1000 km de altitude. Esta camada influi nos sinais de radiofrequência transmitidos por satélites para a superfície terrestre, sendo composta por gases ionizados principalmente pela radiação solar e elétrons livres. Variabilidades no fluxo solar geram alterações nos campos elétrico e magnético no espaço e, consequentemente, no campo magnético da Terra causando flutuações na ionização e, portanto, na quantidade de elétrons livres na ionosfera, alterando a transmissão de sinais de radiofrequência. Dentre as várias pertubações ionosféricas, há a anomalia magnética equatorial, que consiste na formação de uma região com alta densidade de elétrons ente 15 e 20 graus magnéticos ao norte e sul do equador. Entretanto, essa anomalia é mais significativa no Brasil, especificamente no Vale do Paraíba, estado de São Paulo. Os sinais de radiofrequência dos sistemas de navegação por satélites (GNSS - Global Navigation Satellite System) são afetados por flutuações da ionização, que constituem o fenômeno de cintilação ionosférica, que pode ser medidas pelo índices S4. A cintilação afeta sinais GNSS, especialmente no pico da anomalia, afetando a navegação aérea e outras atividades humanas que dependem de sinais recebidos de satélites.  Por outro lado, a quantidade de elétrons livres na ionosfera pode ser medida pelo conteúdo eletrônico total vertical (VTEC), sendo que regiões com baixos valores de VTEC em relação à sua vizinhança, caracterizam as denominadas de bolhas ionosféricas, associadas às cintilações. Dada a existência de redes de estações de medição que provém valores de S4 e VTEC, este trabalho busca correlacionar os valores destas variáveis, bem como analisar sua evolução espaço-temporal por meio de técnicas de mineração e visualização de dados, considerando como estudo de caso a cidade de São José dos Campos.

\palavraschave{%
  \palavrachave{Cintilação Ionosférica}%
%  \palavrachave{Bolha Ionosférica}%
  \palavrachave{Índice S4}%
%  \palavrachave{VTEC}%
  \palavrachave{Mineração de dados}%
  \palavrachave{GNSS}
%  \palavrachave{Vale do Paraíba}
}
 
\end{resumo}
 %% obrigatório
%%%%%%%%%%%%%%%%%%%%%%%%%%%%%%%%%%%%%%%%%%%%%%%%%%%%%%%%%%%%%%%%%%%%%%%%%%%%%%%%
% ABSTRACT


\begin{abstract}

%% neste arquivo abstract.tex
%% o texto do resumo e as palavras-chave têm que ser em Inglês para os documentos escritos em Português
%% o texto do resumo e as palavras-chave têm que ser em Português para os documentos escritos em Inglês
%% os nomes dos comandos \begin{abstract}, \end{abstract}, \keywords e \palavrachave não devem ser alterados

\selectlanguage{english}	%% para os documentos escritos em Português
%\selectlanguage{portuguese}	%% para os documentos escritos em Inglês

\hypertarget{estilo:abstract}{} %% uso para este Guia

The ionosphere is a layer of gas in the state of plasma that was ionized mainly by the effect of solar radiation. Ionospheric plasma distribution is not uniform in space and time, being  the generation of plasma irregularities triggered by the day-to-night transition. One of them is que Equatorial Ionization Anomaly, which coupled  with the plasma instability mechanism cause depletions, i.e. regions with low density of ions and electrons. Such structures are known as ionospheric bubbles and are generated at the magnetic equator just after sunset. They ascent to higher altitudes and migrate to low latitudes along the Earth magnetic field. RF signals are affected by the ionospheric bubbles. Ionospheric scintillation is the occurrence of perturbation of RF signals due to ionospheric irregularities, causing signal phase and amplitude fluctuations. This proposal addresses the use of knowledge discovery in databases in order to predict ionospheric scintillation in the Brazilian territory, in particular, in São José dos Campos. It is intended to employ historical data of ionospheric scintillation and other data including solar activity level, plasma vertical drift velocity and global magnetic activity. The algorithm used for the prediction, either formulated as a classification or a regression problem, is the Extreme Gradient Boosting (XGBoost), available in the Python programming environment, and preliminary results are presented here.

\keywords{%
	\palavrachave{Ionospheric Scintillation}%
	\palavrachave{S4 Index}%
	\palavrachave{Plasma Bubble}%
%	\palavrachave{VTEC}%
	\palavrachave{Data Mining}%
	\palavrachave{GNSS}
%	\palavrachave{Vale do Paraíba}
}

\selectlanguage{portuguese}	%% para os documentos escritos em Português
%\selectlanguage{english}	%% para os documentos escritos em Inglês

\end{abstract}
 %% obrigatório

\includeListaFiguras %% obrigatório caso haja mais de 3 figuras, gerado automaticamente
\includeListaTabelas %% obrigatório caso haja mais de 3 tabelas, gerado automaticamente

%%%%%%%%%%%%%%%%%%%%%%%%%%%%%%%%%%%%%%%%%%%%%%%%%%%%%%%%%%%%%%%%%%%%%%%%%%%%%%%%
%abreviaturas e siglas  %% opcional, mas recomendado

\begin{abreviaturasesiglas}  %% insira abaixo suas abreviaturas conforme o modelo.

%% sigla (separador: &--&) significado (quebra de linha: \\)
\\
GPS   &--& Sistema de Posicionamento Global\\
OACI   &--& Organização da Aviação Civil Internacional\\
GNSS &--& Sistema de Navegação Global por Satélite\\
EIA &--& Anomalia da Ionização Equatorial\\
VTEC &--& Conteúdo eletrônico total vertical\\
AACGM &--& Altitude Adjusted Corrected Geomagnetic Coordinates\\
KDD &--& Descoberta de Conhecimento em Base de dados\\
CART &--& Classification and Regression Trees\\
SVM &--& Support Vector Machine\\
\end{abreviaturasesiglas}
 %% opcional %% altere o arquivo siglaseabreviaturas.tex

%%%%%%%%%%%%%%%%%%%%%%%%%%%%%%%%%%%%%%%%%%%%%%%%%%%%%%%%%%%%%%%%%%%%%%%%%%%%%%%%%
% simbolos

\begin{simbolos}

%% o comando: \hypertarget{estilo:simbolos}{} abaixo é de uso para este Guia
%% e pode ser retirado

\hypertarget{estilo:simbolos}{}
\\
a   &--& primeira contante \\
b   &--& segunda constante \\
$\rho$  &--& densidade de um fluido\\
$\nu$   &--& viscosidade cinemática\\
$R_{e}$  &--& número de Reynolds\\
$\alpha$  &--& constante de Kolmogorov\\
$k$ &--&  número de onda\\
$K$ &--&  curtose\\
$D_{0}$ &--& dimensão de contagem de caixas\\
$D_{1}$ &--& dimensão de informação\\
$D_{2}$  &--& dimensão de correlação\\
$\lambda_{1}$  &--& expoente de Lyapunov dominante\\
 

\end{simbolos}

 %% opcional %% altere o arquivo simbolos.tex

\includeSumario  %% obrigatório, gerado automaticamente

\inicioIntroducao %% não altere este comando

\chapter{IONOSFERA}

A ionosfera é uma região ionizada da alta atmosfera, estendendo-se de 60 até 1000 km de altitude, assim, englobando partes da mesosfera, termosfera, e exosfera. Esta camada constitui-se de íons e elétrons livres criados primariamente por processo de fotoionização, e gás neutro. A fotoionização ionosférica consiste de um processo físico-químico, onde alguma espécies químicas presentes na atmosfera ganham ou perdem elétrons decorrentes da absorção de radiação solar predominantemente nas faixas ultravioleta, extremo ultravioleta ultravioleta e raios-X \cite{RISNBETH:1969, NEGRETI:2012}. A ionização, também, pode ocorrer devido a colisões com partículas altamente energéticas, provindas do meio solar ou galácticas, o que é mais facilmente observado em altas latitudes, por fenômenos como auroral boreal.

A composição da ionosfera, assim, como a densidade do gases variam em função da altitude. Assim, a densidade de elétrons também varia, pois conforme a radiação penetra na atmosfera mais densa, a produção de elétrons, aumenta até atingir um valor de pico. Abaixo desta altitude, mesmo havendo um aumento na densidade da atmosfera neutra, a produção de elétrons decresce, pois a maior parte da radiação ionizante foi absorvida, e a taxa de recombinação predomina sobre a taxa de produção de elétrons. Devido as diferenças marcantes em termos de processos físicos e químicos que governam o comportamento daquela ionosfera, a mesma, pode ser dividida em camadas, onde em cada uma existe um processo predominante. Finalmente, devido as drásticas mudanças em quantidade de radiação absorvida devido a transição entre noite e dia, existirão camadas que não aparecem em cada instante ao longo de 24 horas.

Durante o período a camada F é a única que apresenta uma ionização significativa, enquanto as camadas E e D apresentam um valor extremamente baixo de ionização. Durante o dia, a camada D e E se tornam mais ionizadas, assim como a camada F, que se divide em duas regiões, F1 que é mais fracamente ionizada, e F2 que é mais intensamente ionizada. A camada F2 existe durante a noite e durante o dia, sendo a principal responsável pela reflexão e refração dos sinais de rádio.

A camada D é a mais interna, estando entre 60 e 90 km acima da superfície da Terra. Sua ionização é devido a radiação do hidrogênio ionizado na série de Lyman-alpha no comprimento de onda de 121.6 nm ionizando o óxido nítrico, $NO$, presente na camada. Além disso, raios X altamente energéticos, com comprimento de onda inferior de 1 nm podem ionizar as moléculas de $N_2$ e de $O_2$. A camada D apresenta apresenta a maior taxa de recombinação. Apresenta uma taxa de absorção considerável para ondas de radio de baixas e médias frequências, principalmente, devido a absorção de energia pelo elétrons livres, o que aumenta suas chances de colisão. Este efeito desaparece durante à noite, devido a uma menor ionização. Pode apresentar valores elevados de ionização em altas latitudes em decorrência de erupções solares com grandes quantidades de matéria hadrônica, prótons, em sua maioria, com uma duração de 24 à 48 horas.

A camada E é a intermediária e está situada entre 90 e 150 km acima da superfície da Terra. A ionização decorre principalmente devido ao espalhamento de raio-X leve (entre 1 e 10 nm) e ultravioleta distante (UV) provindos do Sol com moléculas de oxigênio. A estrutura vertical da camada E é determinada em sua maior parte pela competição entre efeitos de ionização e de recombinação. É importante pela presença de correntes elétricas que nela fluem e interagem com o campo magnético \cite{KIRCHHOFF:1991}. A noite, a camada E quase desaparece, pois sua fonte primaria de ionização não está presente.

A camada F se estende de 150 a mais de 500 km acima da superfície da Terra. Apresenta a maior concentração de elétrons, portanto, sinais que são capazes de penetrar até esta subcamada são capazes de escapar para o espaço. Predominam, nesta, a ionização de átomos de oxigênio por meio de radiação solar no espectro do extremo ultravioleta, entre, 10 e 100 nm. A camada é subdivida em duas regiões, a F2 que está presente durante o dia e a noite, e a F1 que aparece somente durante o dia. 

A subcamada F2 engloba toda a região superior da ionosfera, inclusive a região de pico da densidade de elétrons. Este máximo no perfil vertical de ionização decorre do balanço entre os processos de transporte de plasma e os processos físico-químicos. Acima deste pico, a ionosfera se encontra em equilíbrio difusivo, ou seja, o plasma se distribui com a sua própria escala de altura. A presença do campo magnético contribui para a distribuição da ionização.

\subsection{Termosfera}

A termosfera se inicia aproximadamente a 90 km de altitude e não apresenta um limite superior bem definido, mas que deve estar entre 500 e 600 km de altitude. Nesta surgem os fenômenos de aurora, e nela também estam localizadas as órbitas da estação espacial, do ônibus espacial, assim como de vários satélites. Assim, como todas as demais camadas da atmosfera, sua extensão varia conforme a latitude. A temperatura nesta camada podem alcançar valores superiores a 2000 graus Celsius.

\subsection{Magnetosfera}

\section{Anomalias na ionosfera}

A ionosfera apresenta várias anomalias, ou seja, várias irregularidades na distribuição de elétrons. Este trabalho tem interesse na anomalia equatorial. Esta aparece aproximadamente entre 15 e 20 graus de latitude magnética, tanto no hemisfério norte, quanto no hemisfério sul, na camada F2. Consiste na formação de uma região de alta densidade eletrônica, e é uma anomalia, pois a densidade de plasma deveria ser maior em regiões equatoriais, e não em latitudes magnéticas mais altas.

Sua origem decorre da deriva vertical do plasma da camada F na região equatorial: o processo de ionização da camada F faz surgir um campo elétrico, apontando para leste, enquanto o campo magnético aponta para o norte, considerando então $\vec{E}\times\vec{B}$, tem-se o surgimento de uma força perpendicular ao campo magnético e ao campo elétrico, o que neste caso, aponta para cima, deslocando o plasmas para regiões de mais alta altitudes. Agora, quando em altas altitudes, o plasma for efeito gravitacional e diferença de pressão é trazido de volta à altitudes mais baixas, porém este movimento de descida é mais eficiente ao longo das linhas de campo magnético, levando a um aumento na densidade de plasma em regiões de médias latitudes. 

A distribuição do plasma também pode ser alterada pela ação de outras variáveis, como o vento. O Vale do Paraíba, no estado de São Paulo, encontra-se na região da anomalia equatorial, mais especificamente no pico da anomalia, ou seja, na região onde a densidade de plasma, em altas altitudes, atinge seu valor máximo.

As anomalias que surgem na ionosfera apresentam um certo nível de organização devido às linhas de campo magnético da Terra. Isto ocorre pois partículas ionizadas ou carregas podem ser mover livremente ao longo das linhas magnéticas mas não entre elas. Assim, o estudo do campo magnético da Terra se faz relevante para um grande número de aplicações, \cite{LAUNDAL:2017}. Um resultado imediato deste estudo é que o norte geográfico e magnético não coincidem, e que simplificadamente o campo magnético poderia ser descrito por um dipolo magnético, com centro comum ao da Terra, porém inclinado em relação a linha que liga o norte e sul geográfico. Atualmente, existe vários sistemas de coordenadas magnéticas cujo proposito dependem da região, da aplicação e da faixa de altitude de interesse, para uma revisão entre os sistemas mais comuns consulte a referência \cite{LAUNDAL:2017}. Para este trabalho foi adotado o sistema AACGM, pois é mais adequado à altura ionosférica, contudo ele pertence a classe de sistemas não-ortogonais.

A cintilação ionosférica é uma variação rápida de amplitude e fase em sinais de radio frequência quando estes atravessam irregularidades no plasma ionosférico, como uma bolha de plasma, que é de particular interesse para este trabalho. Neste trabalho se realiza um estudo sobre algumas variáveis relacionadas com o efeito de cintilação ionosférica e com a estrutura de bolha no plasma.

\section{Algumas variáveis importantes para o estudo da cintilação ionosférica}

Existem várias quantidades que são necessárias para um estudo completo da dinâmica ionosférica como, por exemplo, medidas do fluxo de radiação solar, do campo magnético da Terra, da composição da atmosfera. Contudo, este trabalho foca em duas quantidades principais o VTEC e o índice S4, pois são as que fornecem mais informação à respeito do fenômeno de cintilação ionosférica, e as anomalias que a causam.

O Total Eletronic Content (TEC), ou conteúdo total de elétrons em português, é uma quantidade descritiva utilizada para avaliar a densidade de elétrons no plasma ionosférico. É o número total de elétrons integrado ao longo da trajetória entre um transmissor, no espaço, e um receptor, na Terra, em uma seção unitária \cite{HOFMANN:2013}. Por sua definição o TEC é uma quantidade que depende da trajetória, seu cálculo fica mais claro, ao dizer que é a integral de um densidade de elétrons dependente de posição ao longo de uma trajetória que atravessa a ionosfera:

\begin{equation}
    \mbox{TEC}=\int{n_{e}(s)}ds\mbox{,}~
\end{equation}

onde $ds$ especifica o elemento de integração na trajetória. Geralmente é reportada em unidades de TEC (TECU), definido por 1 $TECU=10^{16}$ elétrons/m$^2$. É importante para determinar a cintilação e os atrasos de fase e de grupo em ondas de rádio no meio. VTEC, ou conteúdo total de elétrons vertical é projeção do TEC, ao longo de uma linha normal a superfície da Terra, em outras palavras, ela fornece o conteúdo de elétrons ao longo da normal para cada ponto da superfície da Terra.

O índice S4 é utilizado para medir, avaliar, a cintilação ionosférica. Corresponde ao desvio padrão da intensidade do sinal de GPS de um minuto de dados, coletados com 50 amostras por segundo:

\begin{equation}
    S_4^2=\frac{\braket{I^2}-\braket{I}^2}{\braket{I}^2}\mbox{.}
\end{equation}

\section{Bolhas de Plasma}

As bolhas ionosféricas podem ser definidas como regiões de baixa densidade de plasma ionosférico quando comparadas com a sua vizinhança. Utilizando medidas de VTEC é possível definir essa diferença como 30-50 TECU \cite{TAKAHASHI:2006}.

São originadas na região equatorial, após a rápida elevação do plasma, devido a anomalia equatorial, isto é, o plasma ao acender cria regiões de baixa densidade. Após sua formação podem evoluir para altas altitudes (centenas de quilômetros), estendendo-se ao longo das linhas de campo magnético (milhares de quilômetros) nas direções norte-sul, alcançado em torno de 20 graus de latitude magnética.

\chapter{REZENDE}\label{ch:revisonrezende}

\section{Revisão}

O trabalho \cite{REZENDE:2009} foi pioneiro na utilização de modelos direcionados por dados para a exploração do problema de cintilação ionosférica, entretanto este trabalho não apresenta uma abordagem completamente reprodutível, visto ausência dos códigos e da base de dados. O objetivo desta revisão é entender as dificuldades deste trabalho, assim como sua incompletude em relação a proposta a ser estabelecida nessa dissertação.

\subsection{Principais Pontos}

A previsão de cintilação de curto prazo (uma hora) foi realizada utilizando {\bf amostras com intervalos de 5 minutos}, entretanto alguns dos atributos, como já discutido, apresentam resolução inferior, sendo portanto interpolados, ou copiados de sua vizinhança, enquanto outros apresentam resolução mais altas, e neste caso precisaram ser agrupados, segundo algum critério, por exemplo, por uma operação de máximo.

Os atributos foram:

\begin{itemize}
\item {\bf Hm\_Eq} representa a hora no equador (em São Luis), em intervalos de 5 minutos;
\item {\bf Vel\_Der} é a velocidade máxima de deriva vertial do plasma medida no equador entre as 17LT e 19LT (20UT e 22UT), com resolução de um valor por dia;
\item {\bf Kp} é a média do índice Kp definido pela expressão:
\begin{equation}
\sum_{i=1}^{n}\sum_{j=1}^{i}\frac{Kp_{j}}{ni}\mbox{,}~
\end{equation}
onde $Kp_1$ corresponde ao Kp medido entre 14-17LT, $Kp_2$ entre 11-14LT, até o valor medido entre 5-8LT, o valor de $n$ é 4. Este valor apresenta resolução diária;
\item {\bf F10.7} é o fluxo solar;
\item {\bf S4\_Eq} é o maior valor do índice S4 medido no equador, em um período de 5 minutos;
\item {\bf S4\_PA\_tempo\_real} é o maior valor do índice S4 medido no pico da anomalia (São José dos Campos), em um período de 5 minutos;
\item {\bf S4\_PA} é o S4 estimado com uma hora de antecedência para o pico da anomalia, com resolução de 5 minutos.
\end{itemize}

As primeiras 5 variáveis correspondem aos atributos preditores, enquanto a variável S4\_PA corresponde ao atributo reposta. Os atributos acima passam por algum pré-processamento antes e depois da seleção das amostras:

\begin{itemize}
\item {\bf S4}, antes da seleção, somente são utilizados os valores medidos para satélites com ângulo superior a 30 graus;
\item {\bf S4}, depois da seleção, a grande variabilidade destes dados leva a adoção de um filtro passa baixa, realizado por meio de uma suavização com média móvel, com 15 pontos;
\item {\bf Kp}, depois da seleção, somente mantidos amostras com valores de Kp inferiores a 3, pois valores maiores que este caracterizam forte pertubação magnética, associadas a eventos extremos como tempestades magnéticas, e nesta configuração a predição se torna inviável;
\item {\bf Dados}, depois da seleção, devem compreender o período  entre as 18-23LT (21-01UT) de forma a predizer os valores no intervalo 19-24LT (22-02UT).
\end{itemize}

Ao final, a base de dados deste trabalho apresentava um total de 80 dias de dados com 4680 amostras, coletadas entre 2000 e 2002. Este trabalho também realizou testes com predições com 1 dia de antecedência, entretanto não apresentaramm um resultado tão significativo e, portanto, não serão discutidos.

Restam definir dois elementos para que o problema fica completamente definido, as métricas, e os modelos. Uma vez que as métricas ficam restringidas segundo os modelos, ir-se-á estabelecer estes primeiros:

\begin{itemize}
\item agrupamento por expectation-maximization implementado no ambiente Weka;
\item regras de associação utilizando o algoritmo apriori, onde neste caso os atributos foram discretizados;
\item regressão, utilizando árvores CART com a estratégia de ensemble bagging, implementadas pelo autor, análogo à Floresta Randômica.
\end{itemize}

As métricas foram erro quadrático médio, e índice de correlação de Pearson para o problema de regressão, e inspeção para as demais. A aplicação de agrupamento permitiu concluir que se tratava de um problema altamente não linear, as regras de associação geram conclusões que já eram bem conhecidas da literatura do problema na área de aeronomia. Finalmente, os resultados mais interessantes foram estabelecidos pelo problema de regressão, com erro quadrático médio de 0.05 com correlação de Pearson de 0.985.

O trabalho concluí se apresentando como uma abordagem inédita para o problema da predição da cintilação ionosférica.

\subsection{Análise}

A análise consistiu em levantar e sintetizar alguns pontos que levam a uma definição parcialmente incompleta do problema, ou talvez errônea do problema:

\begin{enumerate}
\item uma vez que o código e a base de dados não é disponibilizada de maneira pública e nem devidamente documentada, a reprodução somente é possível em partes, visto que serão utilizados algoritmos que muito se assemelham, mas que devido a diferença devem levar a resultados diferentes;
\item do ponto de vista de implementação não foi definido uma representação para a variável {\bf Hm\_Eq} que pode ser então representada em segundos, minutos, entre outras opções, observar entretanto que isto não deveria levar a diferença significativas no resultado final;
\item a variável Kp é medida por padrão nos intervalos 00-03UT, 03-06UT, 06-09UT, 09-12UT, 12-15UT, 15-18UT, 18-21UT e 21-24UT, e não nos intervalos utilizados por, sendo portanto necessário definir como é feito este mapeamento, o que não está presente no texto;
\item a definição de como a média móvel é aplicada na quantidade S4 está um tanto incompleta, isto é, dado o i-ésimo elemento de um vetor, com um tamanho de janela de 15 pontos, ela poderia ser aplicada levando-se em consideração: os 14 pontos anteriores, $\{i-14, i-13, ..., i-1, i\}$; ou os 7 pontos anteriores e 7 posteriores, $\{i-7,...,i-1,i,i+1,...,i+7\}$, denominada de forma central; entre outras combinações;
\item levando em consideração que a forma centrada tenha sido adotada, a janela de 15 pontos exigirá que 7 pontos do futuro sejam conhecidos, neste caso, dado um instante $t$ seriam necessários 35 minutos de dados a frete deste para o cálculo da média e,  portanto, na verdade, o resultado não seria previsto com uma hora de antecedência, mas sim 25 minutos;
\item a métrica, erro quadrático médio, pode não contemplar o problema. Para entender tal proposição, considere a adaptação deste problema para uma classificação, ficará evidente que eventos com altos valores de cintilação são mais raros e, portanto, ter-se iá, um problema de classificação não balanceado. Retornando, a regressão, pode-se ocorrer do modelo predizer muito bem valores baixos, que então irão mascarar os efeitos de erros em valores mais altos, visto que irão predominar no processo de cálculo de média.
\end{enumerate}

\section{Reprodução}

\subsection{Original}

O trabalho \cite{REZENDE:2009} foi parcialmente, reproduzido no contexto desta proposta, pois somente uma linha de pesquisa do original foi explorado, o problema de regressão, e cuidados adicionais foram necessários, devido a utilização de mais anos. As variáveis adotadas são:

\begin{itemize}
\item {\bf ut} representa a hora em São Luiz em minutos, em intervalos de 5 minutos;
\item {\bf vhf} é a velocidade máxima de deriva vertical do plasma medida em São Luiz entre as 17LT e 18LT (20UT e 21UT), com resolução de um valor por dia;
\item {\bf ap} é a média do índice ap definido por:
\begin{equation}
\sum_{i=1}^{n}\sum_{j=1}^{i}\frac{ap_{j}}{ni}\mbox{,}~
\end{equation}
onde $ap_1$ corresponde à ap15\_18ut, $ap_2$ à ap12\_15ut até ap00\_03ut, mais ap21\_00ut e ap18\_21ut do dia anterior, totalizando um intervalo de 24 horas, com $n=8$. Este valor apresenta resolução diária;,
\item {\bf F10.7} é o fluxo solar;
\item {\bf s4\_sl} é o maior valor do índice S4 medido em São Luiz, em um período de 5 minutos;
\item {\bf s4\_sj} é o maior valor do índice S4 medido em São José dos Campos, em um período de 5 minutos;
\item {\bf s4\_sj\_shift\_1h} é o S4 estimado com uma hora de antecedência para o pico da anomalia, com resolução de 5 minutos.
\end{itemize}

E os cuidados, assim, como os pré-processamentos foram:

\begin{itemize}
\item A adoção do termo São Luis em relação a equador magnético foi preferida, pois esse esta em movimento, e ao longo do período coletado, assim como extensões deste período ele não estará no mesmo lugar, enquanto os dados são sempre coletados em uma estação fixa em São Luiz;
\item A adoção do termo São José dos Campos em relação a pico da anomalia ocorre, pois a localização do pico depende da quantidade de radiação emitida pelo Sol em seu regime, ciclo solar. Nos anos de 2000, 2001, 2002, o pico da anomalia estava em São José dos Campos, porém nos anos de 2018 e 2019 se encontra em Presidente Prudente. Finalmente, os dados foram sempre coletados na estação em São José dos Campos;
\item A altura hF não é amostrada em intervalos regulares ao longo do período de dados coletados, inicialmente, ela era amostrada em 15 min, e posteriormente passou a ser amostrado em 10 min, portanto neste trabalho se reamostrou toda a série para o intervalo de 10 min, que foi então interpolado por um spline de grau 3, até um máximo de x pontos ausentes na vizinhança do ponto a ser interpolado;
\item Quanto a variável S4, primeiro, somente serão aceitas medidas cuja elevação entre a estação e o receptor sejam maiores que 30 graus; segundo, os dados de cintilação apresentam resolução temporal de 1 min, e são coletados para cada satélite acima do plano de horizonte da estação, portanto, existem vários dados por minuto, com objetivo de ficar-se somente com um dado, estes foram agrupados tomando-se o maior valor de cintilação; os dados com resolução de um minuto são interpolados por spline de ordem 3 com limite de quatro valores ausentes; quarto, os dados são suavizados por um filtro de Savitz-Goley de ordem 3 com janela de tamanho 5, o que levaria a necessidade de apenas, 2 amostras de dados futuros; finalmente, os dados são reamostrados para um intervalo de 5 minutos;
\item A adoção de $ap$ em preferência a $kp$ se deve a esta apresentar uma escala linear e, portanto ser mais condizente com operações como média.
\end{itemize}

Uma abordagem de normalização para o intervalo $[0,05,\,0,95]$ é aplicado a todos os atributos preditores (uma para cada atributo) antes da utilização do algoritmo de aprendizagem de máquina. Este intervalo é adotado já prevendo a possibilidade de aplicação de redes neurais com função de ativação sigmoide, haja visto que esta sofre saturação para valores próximos de 0 e de 1.

Resta definir dois elementos para a serem definidos, o tipo de problema tratado e por consequência os algoritmos e as métricas a serem utilizados. Uma que vez que se trata de uma reprodução parcial, o problema de regressão foi tratado, neste caso utilizando a ferramente XGBOOST, que também consiste da utilização de árvores de decisão e regressão com algoritmos de ensemble, neste caso o boosting; as métricas adotadas foram o erro quadrático médio e o erro absoluto máximo, este último sendo capaz de lidar melhor com o desbalanceamento das amostras.

O segundo problema tratado foi de classificação, e para tal a variável {\bf s4\_sj\_shift\_1h} foi discretizada utilizando a proposta estabelecida por \cite{MUELLA:2008}, mais a adição de uma classe ausente:

\begin{table}
\begin{center}
\begin{tabular}{|c|c|}
\hline
{\bf INTENSIDADE} & {\bf $S_4$} \\ \hline
Saturado          & $S_4 > 1,0$ \\ \hline
Forte             & $0,6 \le S_4 \le 1,0$ \\ \hline
Moderado          & $0,4 \le S_4 \le 0,6$ \\ \hline
Fraco             & $0,2 \le S_4 \le 0,4$ \\ \hline
Ausente           & $ S_4 \le 0,2 $ \\ \hline
\end{tabular}
\end{center}
\end{table}

Neste caso, tem-se um problema com 5 classes como já mencionado não balanceado e uma abordagem de reasmotragem foi empregada de modo balancear o números de elementos tal que todos estejam próximos da cardinalidade da classe com o maior número de elementos. Esta etapa é realizada após a normalização e o algoritmo empregado foi o ADASYN. Finalmente, adotou-se como métrica a precisão balanceada, e a ferramenta utilizada também foi o XGBOOST.

\subsection{Resultados}

\chapter{REZENDE}

\section{Revisão}

O trabalho \cite{REZENDE:2009} foi pioneiro na utilização de modelos direcionados por dados para a exploração do problema de cintilação ionosférica, entretanto este trabalho não apresenta uma abordagem completamente reprodutível, visto ausência dos códigos e da base de dados. O objetivo desta revisão é entender as dificuldades deste trabalho, assim como sua incompletude em relação a proposta a ser estabelecida nessa dissertação.

\subsection{Principais Pontos}

A previsão de cintilação de curto prazo (uma hora) foi realizada utilizando {\bf amostras com intervalos de 5 minutos}, entretanto alguns dos atributos, como já discutido, apresentam resolução inferior, sendo portanto interpolados, ou copiados de sua vizinhança, enquanto outros apresentam resolução mais altas, e neste caso precisaram ser agrupados, segundo algum critério, por exemplo, por uma operação de máximo.

Os atributos foram:

\begin{itemize}
\item {\bf Hm\_Eq} representa a hora no equador (em São Luis), em intervalos de 5 minutos;
\item {\bf Vel\_Der} é a velocidade máxima de deriva vertial do plasma medida no equador entre as 17LT e 19LT (20UT e 22UT), com resolução de um valor por dia;
\item {\bf Kp} é a média do índice Kp definido pela expressão:
\begin{equation}
\sum_{i=1}^{n}\sum_{j=1}^{i}\frac{Kp_{j}}{ni}\mbox{,}~
\end{equation}
onde $Kp_1$ corresponde ao Kp medido entre 14-17LT, $Kp_2$ entre 11-14LT, até o valor medido entre 5-8LT, o valor de $n$ é 4. Este valor apresenta resolução diária;
\item {\bf F10.7} é o fluxo solar;
\item {\bf S4\_Eq} é o maior valor do índice S4 medido no equador, em um período de 5 minutos;
\item {\bf S4\_PA\_tempo\_real} é o maior valor do índice S4 medido no pico da anomalia (São José dos Campos), em um período de 5 minutos;
\item {\bf S4\_PA} é o S4 estimado com uma hora de antecedência para o pico da anomalia, com resolução de 5 minutos.
\end{itemize}

As primeiras 5 variáveis correspondem aos atributos preditores, enquanto a variável S4\_PA corresponde ao atributo reposta. Os atributos acima passam por algum pré-processamento antes e depois da seleção das amostras:

\begin{itemize}
\item {\bf S4}, antes da seleção, somente são utilizados os valores medidos para satélites com ângulo superior a 30 graus;
\item {\bf S4}, depois da seleção, a grande variabilidade destes dados leva a adoção de um filtro passa baixa, realizado por meio de uma suavização com média móvel, com 15 pontos;
\item {\bf Kp}, depois da seleção, somente mantidos amostras com valores de Kp inferiores a 3, pois valores maiores que este caracterizam forte pertubação magnética, associadas a eventos extremos como tempestades magnéticas, e nesta configuração a predição se torna inviável;
\item {\bf Dados}, depois da seleção, devem compreender o período  entre as 18-23LT (21-01UT) de forma a predizer os valores no intervalo 19-24LT (22-02UT).
\end{itemize}

Ao final, a base de dados deste trabalho apresentava um total de 80 dias de dados com 4680 amostras, coletadas entre 2000 e 2002. Este trabalho também realizou testes com predições com 1 dia de antecedência, entretanto não apresentaramm um resultado tão significativo e, portanto, não serão discutidos.

Restam definir dois elementos para que o problema fica completamente definido, as métricas, e os modelos. Uma vez que as métricas ficam restringidas segundo os modelos, ir-se-á estabelecer estes primeiros:

\begin{itemize}
\item agrupamento por expectation-maximization implementado no ambiente Weka;
\item regras de associação utilizando o algoritmo apriori, onde neste caso os atributos foram discretizados;
\item regressão, utilizando árvores CART com a estratégia de ensemble bagging, implementadas pelo autor, análogo à Floresta Randômica.
\end{itemize}

As métricas foram erro quadrático médio, e índice de correlação de Pearson para o problema de regressão, e inspeção para as demais. A aplicação de agrupamento permitiu concluir que se tratava de um problema altamente não linear, as regras de associação geram conclusões que já eram bem conhecidas da literatura do problema na área de aeronomia. Finalmente, os resultados mais interessantes foram estabelecidos pelo problema de regressão, com erro quadrático médio de 0.05 com correlação de Pearson de 0.985.

O trabalho concluí se apresentando como uma abordagem inédita para o problema da predição da cintilação ionosférica.

\subsection{Análise}

A análise consistiu em levantar e sintetizar alguns pontos que levam a uma definição parcialmente incompleta do problema, ou talvez errônea do problema:

\begin{enumerate}
\item uma vez que o código e a base de dados não é disponibilizada de maneira pública e nem devidamente documentada, a reprodução somente é possível em partes, visto que serão utilizados algoritmos que muito se assemelham, mas que devido a diferença devem levar a resultados diferentes;
\item do ponto de vista de implementação não foi definido uma representação para a variável {\bf Hm\_Eq} que pode ser então representada em segundos, minutos, entre outras opções, observar entretanto que isto não deveria levar a diferença significativas no resultado final;
\item a variável Kp é medida por padrão nos intervalos 00-03UT, 03-06UT, 06-09UT, 09-12UT, 12-15UT, 15-18UT, 18-21UT e 21-24UT, e não nos intervalos utilizados por, sendo portanto necessário definir como é feito este mapeamento, o que não está presente no texto;
\item a definição de como a média móvel é aplicada na quantidade S4 está um tanto incompleta, isto é, dado o i-ésimo elemento de um vetor, com um tamanho de janela de 15 pontos, ela poderia ser aplicada levando-se em consideração: os 14 pontos anteriores, $\{i-14, i-13, ..., i-1, i\}$; ou os 7 pontos anteriores e 7 posteriores, $\{i-7,...,i-1,i,i+1,...,i+7\}$, denominada de forma central; entre outras combinações;
\item levando em consideração que a forma centrada tenha sido adotada, a janela de 15 pontos exigirá que 7 pontos do futuro sejam conhecidos, neste caso, dado um instante $t$ seriam necessários 35 minutos de dados a frete deste para o cálculo da média e,  portanto, na verdade, o resultado não seria previsto com uma hora de antecedência, mas sim 25 minutos;
\item a métrica, erro quadrático médio, pode não contemplar o problema. Para entender tal proposição, considere a adaptação deste problema para uma classificação, ficará evidente que eventos com altos valores de cintilação são mais raros e, portanto, ter-se iá, um problema de classificação não balanceado. Retornando, a regressão, pode-se ocorrer do modelo predizer muito bem valores baixos, que então irão mascarar os efeitos de erros em valores mais altos, visto que irão predominar no processo de cálculo de média.
\end{enumerate}

\section{Reprodução}

\subsection{Original}

O trabalho \cite{REZENDE:2009} foi parcialmente, reproduzido no contexto desta proposta, pois somente uma linha de pesquisa do original foi explorado, o problema de regressão, e cuidados adicionais foram necessários, devido a utilização de mais anos. As variáveis adotadas são:

\begin{itemize}
\item {\bf ut} representa a hora em São Luiz em minutos, em intervalos de 5 minutos;
\item {\bf vhf} é a velocidade máxima de deriva vertical do plasma medida em São Luiz entre as 17LT e 18LT (20UT e 21UT), com resolução de um valor por dia;
\item {\bf ap} é a média do índice ap definido por:
\begin{equation}
\sum_{i=1}^{n}\sum_{j=1}^{i}\frac{ap_{j}}{ni}\mbox{,}~
\end{equation}
onde $ap_1$ corresponde à ap15\_18ut, $ap_2$ à ap12\_15ut até ap00\_03ut, mais ap21\_00ut e ap18\_21ut do dia anterior, totalizando um intervalo de 24 horas, com $n=8$. Este valor apresenta resolução diária;,
\item {\bf F10.7} é o fluxo solar;
\item {\bf s4\_sl} é o maior valor do índice S4 medido em São Luiz, em um período de 5 minutos;
\item {\bf s4\_sj} é o maior valor do índice S4 medido em São José dos Campos, em um período de 5 minutos;
\item {\bf s4\_sj\_shift\_1h} é o S4 estimado com uma hora de antecedência para o pico da anomalia, com resolução de 5 minutos.
\end{itemize}

E os cuidados, assim, como os pré-processamentos foram:

\begin{itemize}
\item A adoção do termo São Luis em relação a equador magnético foi preferida, pois esse esta em movimento, e ao longo do período coletado, assim como extensões deste período ele não estará no mesmo lugar, enquanto os dados são sempre coletados em uma estação fixa em São Luiz;
\item A adoção do termo São José dos Campos em relação a pico da anomalia ocorre, pois a localização do pico depende da quantidade de radiação emitida pelo Sol em seu regime, ciclo solar. Nos anos de 2000, 2001, 2002, o pico da anomalia estava em São José dos Campos, porém nos anos de 2018 e 2019 se encontra em Presidente Prudente. Finalmente, os dados foram sempre coletados na estação em São José dos Campos;
\item A altura hF não é amostrada em intervalos regulares ao longo do período de dados coletados, inicialmente, ela era amostrada em 15 min, e posteriormente passou a ser amostrado em 10 min, portanto neste trabalho se reamostrou toda a série para o intervalo de 10 min, que foi então interpolado por um spline de grau 3, até um máximo de x pontos ausentes na vizinhança do ponto a ser interpolado;
\item Quanto a variável S4, primeiro, somente serão aceitas medidas cuja elevação entre a estação e o receptor sejam maiores que 30 graus; segundo, os dados de cintilação apresentam resolução temporal de 1 min, e são coletados para cada satélite acima do plano de horizonte da estação, portanto, existem vários dados por minuto, com objetivo de ficar-se somente com um dado, estes foram agrupados tomando-se o maior valor de cintilação; os dados com resolução de um minuto são interpolados por spline de ordem 3 com limite de quatro valores ausentes; quarto, os dados são suavizados por um filtro de Savitz-Goley de ordem 3 com janela de tamanho 5, o que levaria a necessidade de apenas, 2 amostras de dados futuros; finalmente, os dados são reamostrados para um intervalo de 5 minutos;
\item A adoção de $ap$ em preferência a $kp$ se deve a esta apresentar uma escala linear e, portanto ser mais condizente com operações como média.
\end{itemize}

Uma abordagem de normalização para o intervalo $[0,05,\,0,95]$ é aplicado a todos os atributos preditores (uma para cada atributo) antes da utilização do algoritmo de aprendizagem de máquina. Este intervalo é adotado já prevendo a possibilidade de aplicação de redes neurais com função de ativação sigmoide, haja visto que esta sofre saturação para valores próximos de 0 e de 1.

Resta definir dois elementos para a serem definidos, o tipo de problema tratado e por consequência os algoritmos e as métricas a serem utilizados. Uma que vez que se trata de uma reprodução parcial, o problema de regressão foi tratado, neste caso utilizando a ferramente XGBOOST, que também consiste da utilização de árvores de decisão e regressão com algoritmos de ensemble, neste caso o boosting; as métricas adotadas foram o erro quadrático médio e o erro absoluto máximo, este último sendo capaz de lidar melhor com o desbalanceamento das amostras. 

O segundo problema tratado foi de classificação, e para tal a variável {\bf s4\_sj\_shift\_1h} foi discretizada utilizando a proposta estabelecida por \cite{MUELLA:2008}, mais a adição de uma classe ausente:

\begin{table}
\begin{center}
\begin{tabular}{|c|c|}
\hline
{\bf INTENSIDADE} & {\bf $S_4$} \\ \hline
Saturado          & $S_4 > 1,0$ \\ \hline
Forte             & $0,6 \le S_4 \le 1,0$ \\ \hline
Moderado          & $0,4 \le S_4 \le 0,6$ \\ \hline
Fraco             & $0,2 \le S_4 \le 0,4$ \\ \hline
Ausente           & $ S_4 \le 0,2 $ \\ \hline
\end{tabular}
\end{center}
\end{table}

Neste caso, tem-se um problema com 5 classes como já mencionado não balanceado e uma abordagem de reasmotragem foi empregada de modo balancear o números de elementos tal que todos estejam próximos da cardinalidade da classe com o maior número de elementos. Esta etapa é realizada após a normalização e o algoritmo empregado foi o ADASYN. Finalmente, adotou-se como métrica a precisão balanceada, e a ferramenta utilizada também foi o XGBOOST.

Para ambos os problemas uma busca randômica é empregado para encontrar os melhores valores para os hiper-parâmetros.

\subsection{Resultados}

%\chapter{INTRODUÇÃO}

%\include{./docs/doc_body/chapter_2}
%



%\chapter{Metodologia}

Este trabalho utiliza o índice S4, uma lista de estações que realizam medidas desta variável e o VTEC. O trabalho adota a ideia de pesquisa reprodutível, para tal se faz uso da tecnologia de ``notebooks"~em Python, \cite{PEREZ:2007}. Python notebooks consistem de um ambiente computacional interativo de código aberto, onde pode-se combinar execução de código, com textos e expressões matemáticas em HTML e LaTex, gráficos, imagens, vídeos, entre diversos outros objetos. O projeto inicial evolui para o Jupyter notebook que apresenta as mesmas ideias, porém apresenta suporte para várias linguagens de programação, tal como R, Julia e Scala. 

Os dados de S4 foram originalmente disponibilizados em formato de texto, juntamente com uma lista de estações ao longo do território brasileiro. Cada arquivo contém uma lista de medidas ordenadas pelo tempo, com múltiplas medidas por minuto, cada uma associada a um satélite. A etapa inicial consistiu em armazenar e organizar estes dados em um banco de dados, pois então, técnicas como filtragem permitem uma rápida seleção inicial dos dados por estação e elevação. Também foi adicionado ao banco de dados uma tabela com informações sobre as estações. Neste trabalho, adotou-se o PostgreSql.

Os notebooks estão organizados tal que os primeiros dois dígitos indiquem uma ordem de execução, por exemplo, o notebook com inicial $00$ precisa ser executado antes do notebook com inicial $01$. Existem notebooks com os mesmos valores de dígitos, isto significa que um não apresenta dependência em relação ao outro e podem ser executados ao mesmo tempo.

\section{Seleção inicial dos dados de S4 e estações}

Inicialmente, realizou-se uma consulta a tabela de estações no banco de dados. Utilizando, então, a lista de estações retornada, fez-se iterativamente uma consulta para cada estação selecionando apenas medidas cuja elevação é superior a 30.0. Somente continuaramm as estações que contém dados, gerando assim, uma tabela com os dados das estações, que entender-se-á como o conjunto inicial de estações, e um grupo de arquivos com dados de S4, um para cada estação válida. Os dados de S4 armazenados em uma tabela ordenada, indexada pelo tempo, podem ser vistos como uma série temporal. Esta fase foi desenvolvida no notebook 00\_extract\_valid\_stations.ipynb. 

\section{Geração da série espaço-temporal para os dados de VTEC}

Os dados de VTEC estão inicialmente organizados em arquivos de texto, as duas primeiras linhas são de cabeçalho, onde a primeira, denota o instante dos dados, e a segunda fornece o significado de cada coluna. A primeira coluna é a longitude, a segunda a latitude e a terceira o VTEC. Medidas de VTEC $999.000$ denotam ausência de valor. Finalmente, cada arquivo constitui um mapa de VTEC, onde cada linha fornece o valor de VTEC para um ponto no espaço. Assim, o papel do notebook 00\_generate\_vtec\_dataframe.ipynb é o de converter este conjunto de arquivos em um tabela, indexada pelo tempo, onde cada linha contém uma matriz, que realiza o papel do mapa. Pode-se observar esta estrutura também como uma série espaço-temporal, onde um índice de tempo indexa um mapa de VTEC. Este notebook também realiza um ordenamento temporal nos dados, o que é necessário posteriormente para o cálculo de derivadas temporais, traduzidas aqui como diferenças finitas no tempo. Esta etapa é realizada pelo notebook 00\_generate\_vtec\_dataframe.ipynb.

\section{Mapa das estações com os meridianos magnéticos} 

As bolhas ionosféricas evoluem, como mencionado, ao longo de meridianos magnéticos. Assim, é interessante visualizar o conjunto de estações por meio de um mapa juntamente com os meridianos magnéticos que as atravessam, para tal, inicialmente, realiza-se um ordenamento na tabela de estações por estado e cidade. Seguindo de um agrupamento por cidade, uma vez que existem cidades com mais de um estação, e neste trabalho se adotou empregar apenas uma estação por cidade. 

Por meio do pacote Python AACGMV2, que implementa um modelo numérico para o sistema de coordenadas geomagnética AACGM, foi possível gerar as latitudes e longitudes geomagnéticas para as estações, utilizando os dados de localização geográfica, altitude, bem como uma data de referência.

Utilizando os dados geomagnéticos se plotou um mapa contendo as estações, assim, como os meridianos magnéticos que passam por estas. O mapa plotado na figura \ref{fig:mapstations} fornece uma visualização das estações que tendem a apresentam maior correlação entre si, pois pertencem a uma mesma eventual trajetória. Esta etapa é desenvolvida no notebook 01\_show\_stations.ipynb.

\begin{figure}[H]
\centering
\makebox[\textwidth][c]{\includesvg[width=1.6\columnwidth]{./Figuras/map_stations.svg}}
\vspace{-60pt}
\caption{Representação dos meridianos magnéticos que atravessam as estações que realizam medidas de cintilação ionosférica. Fonte: próprio autor.}
\label{fig:mapstations}
\end{figure}

\section{Suavização da série temporal de S4}

Os dados de S4 apresentam intensas variabilidades, devido a rápidas flutuações, ruídos e a existem de grandes intervalos de tempo sem dados, portanto é necessário realizar um pré-processamento nos dados de S4. Primeiramente, toma-se o valor médio de todas as medidas realizada por minuto, isto é, realiza-se uma reamostragem combinando as medidas de diferentes satélites feitas no mesmo minuto, tal que se tenha um valor de S4 por minuto. 

Após esta etapa é feita uma interpolação spline de ordem 4 para tratar instantes sem amostras. Em seguida, é feita uma reamostragem para intervalos de 10 min, novamente utilizando o valor médio como função de agregação. O intervalo adotado é o mesmo do dados de VTEC. Nas Figuras \ref{fig:savgol} e \ref{fig:gaussian} é possível observar em azul uma amostra do sinal S4, pode-se notar que este apresenta rápidas flutuações e ruído, como comentado. Assim, optou-se por aplicar uma função de suavização. Em \ref{fig:savgol}, tem-se a utilização do filtro de Savitzky–Golay com janela de 9 pontos e polinômio interpolador de terceira ordem, enquanto na figura \ref{fig:gaussian} foi utilizado um filtro baseado em média móvel com pesos gaussianos com uma janela também de 9 pontos.

\begin{figure}[H]
\centering
\makebox[\textwidth][c]{\includegraphics[width=1.2\linewidth]{./Figuras/s4_signal_noise_and_smooth_savgol.eps}}
\caption{Suavização de uma amostra da série temporal de S4, em São José dos Campos - SP, por meio do filtro de Savitzky–Golay, no intervalo entre 18:00 UT do dia 25/01/2013 até 12:00 UT do dia 27/01/2013. As linhas verticais vermelhas representam o por do sol, enquanto as amarelas representam o nascer do sol. Fonte: próprio autor.}
\label{fig:savgol}
\end{figure}

\begin{figure}[H]
\makebox[\textwidth][c]{\includegraphics[width=1.2\linewidth]{./Figuras/s4_signal_noise_and_smooth_gaussian.eps}}
\caption{Suavização de uma amostra da série temporal de S4, em São José dos Campos - SP, por meio de uma média móvel com peso gaussiano, no intervalo entre 18:00 UT do dia 25/01/2013 até 12:00 UT do dia 27/01/2013. As linhas verticais vermelhas representam o por do sol, enquanto as amarelas representam o nascer do sol. Fonte: próprio autor.}
\label{fig:gaussian}
\end{figure}

Finalmente, optou-se por utilizar uma combinação das duas técnicas de suavização aplicando primeiro do filtro de Savitzky–Golay seguido da média móvel com pesos gaussianos, ambos com os parâmetros especificados no parágrafo anterior. Essa escolha gera um sinal mais suave do que a aplicação de um ou outro. Nos gráficos estão presentes também linhas verticais vermelhas e amarelas que representam respectivamente o pôr do sol e o nascer do sol em UT calculadas segundo a localização geográfica da estação. Tais linhas estão presentes entre outros gráficos ao longo do texto, mantendo o mesmo padrão de cores.

Nesta etapa adicionalmente foi gerado um tabela com todos os dados de S4, em passos de 10 min, onde as colunas representam o conjunto inicial de estações. Os códigos desenvolvidos nessa fase são implementados pelo notebook 02\_preprocessing\_s4\_data.ipynb.

\section{Extração da série temporal de VTEC, para algumas estações}

Dados os mapas de VTEC em sua forma tabela contendo matrizes é conveniente, em termos de uso e desempenho, extrair as séries temporais para as localizações em que existam estações onde o índice S4 é medido. Estes dados então são preprocessados aplicando o mesmo processo de suavização utilizados nos dados de cintilação. Nas Figuras \ref{fig:vtecsj2gauss} e \ref{fig:vtecsj2savi} são possíveis visualizar amostras do VTEC para São José dos Campos, juntamente com a aplicação das duas técnicas de suavização. Na figura \ref{fig:vtecsignal} há uma amostra do sinal de VTEC para o conjunto inicial de estações, observe que no intervalo entre às 9:00 e 15:00 UT o VTEC é aproximadamente similar entre as diversas estações, e que a partir das 18:00 UT os valores começam a divergir entre si, apresentado grandes diferenças após 00:00 UT, até 06:00 UT, onde então começam a se agrupar. Tal padrão se apresenta ao longo de todo o período amostrado para este trabalho.
\vspace{-16pt}

\begin{figure}[H]
\centering
\makebox[\textwidth][c]{\includegraphics[width=1.2\columnwidth]{./Figuras/vtec_sj2_gauss.eps}}
\caption{Amostra de um sinal de VTEC, em São José dos Campos - SP, juntamente com sua suavização utilizando média móvel com peso gaussiano. O correspondente período foi de 00:00 UT do dia 01/12/2013 até 12:00 UT do dia 04/12/2013. As linhas verticais vermelhas representam o por do sol, enquanto as amarelas representam o nascer do sol. Fonte: próprio autor.}
\label{fig:vtecsj2gauss}
%\end{figure}
%
%\begin{figure}[H]
\centering
\makebox[\textwidth][c]{\includegraphics[width=1.2\columnwidth]{./Figuras/vtec_sj2_savi.eps}}
\caption{Amostra de um sinal de VTEC, em São José dos Campos - SP, juntamente com suas suavização pelo método Savitzky–Golay. O correspondente período foi de 00:00 UT do dia 01/12/2013 até 12:00 UT do dia 04/12/2013. As linhas verticais vermelhas representam o por do sol, enquanto as amarelas representam o nascer do sol. Fonte: próprio autor.}
\label{fig:vtecsj2savi}
\end{figure}

\vspace{-16pt}
Esta é implementada pelo notebook 03\_extract\_vtec\_stations.ipynb, o qual também gera uma tabela, onde cada coluna representa uma estação diferente com os dados de VTEC indexados pelo tempo, de modo que, pode-se falar em uma tabela de séries temporais para os dados de VTEC.

\begin{figure}[H]
\makebox[\textwidth][c]{\includegraphics[width=1.3\columnwidth]{./Figuras/vtec_signal.eps}}
\caption{Amostra de um sinal de VTEC, para múltiplas estações. Note a convergência da série para a vizinhança das 12:00 UT e sua divergência próximo das 00:00 UT. O gráfico correspondente ao período de 12:00 UT do dia 01/12/2013 até 12:00 UT do dia 03/12/2013. As linhas verticais vermelhas representam o por do sol, enquanto as amarelas representam o nascer do sol. Fonte: próprio autor.}
\label{fig:vtecsignal}
\end{figure}

\vspace{-16pt}
\section{Seleção fina dos dados de S4}

Após o pré-processamento dos dados de S4, é realizada uma seleção mais fina das estações que serão utilizadas neste trabalho. O primeiro conjunto de estações descartas o foi, pois apresentava poucos pontos ao longo do período de janeiro de 2013 à dezembro de 2014 o que levou a curvas interpoladas que não são condizentes com a variável observada, o que foi evidenciado pela plotagem das séries temporais. 

Os dados de VTEC foram amostrados em um intervalo menor do que o S4, assim, foi realizado um recorte na série temporal de S4, tal que ambas as séries tenham o mesmo intervalo de amostragem. Assim, o segundo conjunto de estações descartadas são aquelas que não apresentam medidas no período de tempo selecionado. A Tabela \ref{tab:stations} apresenta o grupo final de estações selecionados para o trabalho. A figura \ref{fig:s4stations} exibe a série temporal S4 para algumas estações, enquanto a figura \ref{fig:mapstationsre} apresenta um mapa com todas as estações selecionadas. Esta fase foi desenvolvida no notebook 04\_reanalize\_data.ipynb.

\begin{table}
\addtolength{\leftskip} {-2cm} % increase (absolute) value if needed
\addtolength{\rightskip}{-2cm}
\small
\begin{tabular}{|l|l|l|c|c|c|c|c|}
\hline
Cidade              & Est.  & Cód. de Id.           &  Alt.     &   Lat.     &  Lon.      &  Lat. Mag.    &  Lon. Mag.       \\ \hline
Belo Horizonte      &    MG &                   bhz &   858.000 & -19.868500 & -43.954200 &    -25.426147 &      24.786619   \\ \hline
Brasília            &    DF &                   bsa &  1050.000 & -15.764200 & -47.869400 &    -24.348659 &      22.352744   \\ \hline
Cachoeira Paulista  &    SP &                   cpa &   580.000 & -22.410000 & -45.000000 &    -24.456556 &      22.960540   \\ \hline
Campo Grande        &    MS &                    32 &       NaN & -20.497000 & -54.615000 &    -21.417704 &      14.873907   \\ \hline
Cuiabá              &    MT &                   cub &   278.000 & -15.555200 & -56.069800 &    -14.336068 &      14.530440   \\ \hline
Dourados            &    MS &                   dou &   756.120 & -22.110000 & -54.550000 &    -23.627266 &      14.698554   \\ \hline
Fortaleza           &    CE &                    24 &       NaN &  -3.742000 & -38.539000 &           NaN &            NaN   \\ \hline
Guaratinguetá       &    SP &                    33 &       NaN & -22.789000 & -45.220000 &    -24.188879 &      22.620120   \\ \hline
Ilhéus              &    BA &                   ios &     0.000 & -14.470000 & -39.100000 &    -13.470248 &      30.548727   \\ \hline
Inconfidentes       &    MG &                    25 &       NaN & -22.318000 & -46.329000 &    -26.299459 &      22.004117   \\ \hline
Macaé               &    RJ &                    11 &       NaN & -22.823000 & -41.785700 &    -20.542047 &      25.191448   \\ \hline
Natal               &    RN &                   nta &     0.000 &  -5.836162 & -35.121000 &           NaN &            NaN   \\ \hline
Palmas              &    RO &                     3 &       NaN & -10.200000 & -48.312000 &    -12.264838 &      23.425112   \\ \hline
Pirassununga        &    SP &                    30 &       NaN & -21.989000 & -47.334000 &    -23.990783 &      21.003125   \\ \hline
Porto Alegre        &    RS &                     4 &       NaN & -30.071000 & -51.119000 &    -22.954879 &      15.550843   \\ \hline
Presidente Prudente &    SP &                     6 &       NaN & -22.120000 & -51.407000 &    -21.640946 &      17.249042   \\ \hline
Rio de Janeiro      &    RJ &                    34 &       NaN & -22.823000 & -43.238000 &    -20.105803 &      23.888647   \\ \hline
Salvador            &    BA &                    26 &       NaN & -13.001000 & -38.508000 &    -12.123350 &      31.680944   \\ \hline
Santa Maria         &    RS &                   sta &   110.100 & -29.712591 & -53.717206 &    -22.659740 &      13.628064   \\ \hline
São José dos Campos &    SP &                   sj2 &   593.440 & -23.207000 & -45.859000 &    -24.835610 &      22.002028   \\ \hline
Tefé                &    AM &                   tfe &     0.057 &  -3.180000 & -64.440000 &      6.385157 &       9.314963   \\ \hline
\end{tabular}

\vspace{12pt}

\begin{center}
\begin{tabular}{|l|c|c|c|}
\hline
Cidade              &   Alt. da Cidade &  Lat. da Cidade &  Lon. da Cidade \\ \hline
Belo Horizonte      &            767.0 &       -19.81570 &        -43.9542 \\ \hline
Brasília            &           1130.0 &       -15.78010 &        -47.9292 \\ \hline
Cachoeira Paulista  &            545.0 &       -22.67370 &        -44.9973 \\ \hline
Campo Grande        &            612.0 &       -20.44350 &        -54.6478 \\ \hline
Cuiabá              &            180.0 &       -15.59890 &        -56.0949 \\ \hline
Dourados            &            448.0 &       -22.22180 &        -54.8064 \\ \hline
Fortaleza           &             14.0 &        -3.71839 &        -38.5434 \\ \hline
Guaratinguetá       &            526.0 &       -22.81620 &        -45.1935 \\ \hline
Ilhéus              &              9.0 &       -14.79730 &        -39.0355 \\ \hline
Inconfidentes       &            864.0 &       -22.31710 &        -46.3284 \\ \hline
Macaé               &              7.0 &       -22.37170 &        -41.7857 \\ \hline
Natal               &             38.0 &        -5.79448 &        -35.2110 \\ \hline
Palmas              &            260.0 &       -10.16890 &        -48.3317 \\ \hline
Pirassununga        &            625.0 &       -21.99600 &        -47.4268 \\ \hline
Porto Alegre        &             22.0 &       -30.02770 &        -51.2287 \\ \hline
Presidente Prudente &            471.0 &       -22.12760 &        -51.3856 \\ \hline
Rio de Janeiro      &             20.0 &       -22.90350 &        -43.2096 \\ \hline
Salvador            &             12.0 &       -12.97040 &        -38.5124 \\ \hline
Santa Maria         &            139.0 &       -29.69140 &        -53.8008 \\ \hline
São José dos Campos &            593.0 &       -23.17910 &        -45.8872 \\ \hline
Tefé                &             28.0 &        -3.32073 &        -64.7236 \\ \hline
\end{tabular}
\end{center}

\vspace{12pt}

\caption{Conjunto de estações que realizam medidas do índice S4 juntamente com seus atributos. Fonte: próprio autor.}
\label{tab:stations}
\end{table}

\begin{figure}[H]
\centering
\makebox[\textwidth][c]{\includegraphics[width=1.2\columnwidth]{./Figuras/s4_stations_sample.eps}}
\caption{Amostra do sinal S4 de dezembro de 2013 até março de 2014. Fonte: próprio autor.}
\label{fig:s4stations}
\end{figure}

\begin{figure}[H]
\centering
\makebox[\textwidth][c]{\includesvg[width=1.6\columnwidth]{./Figuras/map_stations_re.svg}}
\vspace{-60pt}
\caption{Conjunto final de estações que serão utilizadas para este trabalho. Fonte: próprio autor.}
\label{fig:mapstationsre}
\end{figure}

\section{Visualização da séries temporais para S4 e VTEC}\label{sec:viss4vtec}

Feita uma seleção mais fina das estações que fornecem as séries temporais de S4, assim, como um recorte adequado no tempo, é adequado plotar a série temporal do VTEC, juntamente com a do S4. Esta etapa é realizada no notebook 05\_visualize\_vtec\_s4\_data. Os gráficos são feitos em dois grupos distintos, o primeiro apresenta uma amostra das séries temporais, fornecendo um resolução visual melhor, enquanto o segundo fornece a série completa. Algumas observações pode ser feitas analisando visualmente ambos os conjuntos. Nota-se, por exemplo:

\begin{itemize}
\item uma periodicidade em ambos os dados; 
\item os valores de S4 sobem conforme o do VTEC diminui;
\item os valores de pico de S4 aparecem em quedas e mínimos do VTEC;
\item os dados de S4 apresentam maior ruído e menor disponibilidade;
\item flutuações mais intensas do S4 aparecem no máximo do VTEC;
\item após o por do sol existe um aumento no valor de VTEC correspondente ao pico de pré reversão, por sua vez quando confrontado com o S4, observa-se uma certa sobreposição entre a cintilação e o pico de pré reversão.~
\end{itemize}

Nas Figuras \ref{fig:s4vtecsample} e \ref{fig:s4vteccomplete} é possível observar respectivamente uma amostra da série temporal de S4 contra VTEC, e a série completa para algumas estações selecionadas.

\begin{figure}[H]
\centering
\makebox[\textwidth][c]{\includegraphics[width=1.2\columnwidth]{./Figuras/s4_vtec_sample.eps}}
\caption{Amostra do sinal S4 e VTEC, no intervalo entre 00:00 UT do dia 01/12/2013 até 12:00 UT do dia 04/12/2013. As linhas verticais vermelhas representam o por do sol, enquanto as amarelas representam o nascer do sol. A Figura \ref{fig:scattervtec} é um recorte para a estão de São José dos Campos onde estão indicados possíveis picos de pré reversão, estes são as regiões com aumento no VTEC logo após o por do sol. Fonte: próprio autor.}
\label{fig:s4vtecsample}
\end{figure}

\begin{figure}[H]
\centering
\makebox[\textwidth][c]{\includegraphics[width=1.2\columnwidth]{./Figuras/s4_vtec_complete.eps}}
\caption{Série temporal para S4 e VTEC, ao longo de todo o período de coleta, isto é de de dezembro de 2013 até março de 2014. Fonte: próprio autor.}
\label{fig:s4vteccomplete}
\end{figure}

\section{Variáveis derivadas do VTEC}\label{sec:vdt}

Existem várias formas de se buscar por padrões e correlações em um conjunto de dados, tal como a visualização por meio da plotagem de um gráfico representativo de uma série temporal. Assim, a seção \ref{sec:viss4vtec} forneceu um indicativo da correlação em sua conclusão. Observado tal fato o próximo passo é buscar por um modelo que faça um mapeamento entre VTEC e S4. Este pode ter como conjunto de entrada o valor de VTEC e o de saída o S4, uma vez que ambas as variáveis são contínuas, busca-se realizar uma regressão. 

Poder-se-ia buscar um mapeamento tal que a cada valor de VTEC seja associado a um valor de S4 por estação, entretanto, usando os padrões observados juntamente das referências \cite{RAGHUNATH:2016, RAGHAVARAO:1998,RAY:2006}, derivou-se um conjunto adicional de variáveis, que são a derivada temporal primeira e segunda do VTEC, a diferença do VTEC entre São José dos Campos e Pirassununga, a diferença do VTEC entre São José dos Campos e Brasília, assim como as derivadas temporais primeira de ambas as diferenças. 

As derivadas temporais são interessantes, pois os valores de S4 aumentam conforme o valor de VTEC diminui, portanto existe uma variação no tempo que pode ser melhor extraída pela derivada temporal. As bolhas ionosféricas se deformam e se propagam ao longo de um meridiano magnético, as estações de Brasília, Pirassununga e São José dos Campos se encontram aproximadamente sobre o mesmo meridiano magnético, usando ambos os fatos considere que exista uma bolha em Brasília, e não em São José dos Campos, a diferença de VTEC terá um valor positivo, enquanto que se estiverem em ambas as cidades ter-se-á um valor aproximadamente nulo, e com a bolha apenas em São José dos Campos um valor de diferença negativa, isto exibe claramente a propriedade da diferença espacial do VTEC em mapear a localização da bolha. A derivada temporal das diferenças espaciais fornece um indicativo da dinâmica da bolha.

Sumarizando, ficou-se com o seguinte conjunto de variáveis, denominado conjunto ``original'' de atributos, mais uma variável alvo S4:

\begin{itemize}
\item {\bf vtec} - conteúdo eletrônico total vertical em São José dos Campos;
\item {\bf vtec\_dt} - diferença finita de primeira ordem no tempo do VTEC, calculada por $vtec_i-vtec_{i-1}$;
\item {\bf vtec\_dt2} - diferença finita de segunda ordem no tempo do VTEC, calculada por $vtec_{i+1}-2vtec_i+vtec_{i-1}$;
\item {\bf gvtec1} - diferença entre o VTEC de São José dos Campos e Pirassununga;
\item {\bf gvtec1\_dt} - diferença finita de primeira ordem no tempo do gvtec1, calculada por $gvtec1_i-gvtec1_{i-1}$;
\item {\bf gvtec2} - diferença entre o VTEC de São José dos Campos e Brasília;
\item {\bf gvtec2\_dt} - diferença finita de primeira ordem no tempo do gvtec2, calculada por $gvtec2_i-gvtec2_{i-1}$;
\item {\bf S4} - índice de cintilação ionosférico em São José dos Campos.
\end{itemize}

Nas definições acima de diferença finita, descartou-se valores constantes, pois eles seriam naturalmente eliminados pelo processo de normalização utilizado antes de aplicar os dados nos algoritmos de aprendizagem de máquina. 

O papel do notebook 06\_analise\_sj2.ipynb é o de construir as variáveis definidas, assim como o de concatenar tais variáveis em uma tabela que possa ser utilizada em algoritmos de aprendizagem de maquina para desenvolvimento de um modelo. Nas Figuras \ref{fig:scattervtec} até \ref{fig:scattergvtec2} é possível observar amostras das variáveis elaboras, enquanto na figura \ref{fig:vtec_vtec_dt}, tem-se uma representação gráfica da variável {\bf vtec} contra {\bf vtec\_dt}.

\begin{figure}[H]
\centering
\makebox[\textwidth][c]{\includegraphics[width=1.2\columnwidth]{./Figuras/vtec_scatter_mod.eps}}

\caption{Amostra dos sinais {\bf vtec} e {\bf S4}, no período entre 00:00 UT do dia 01/12/2013 até 12:00 UT do dia 04/12/2013. As linhas verticais vermelhas representam o por do sol, enquanto as amarelas representam o nascer do sol. Possíveis picos de pré reversão são indicados pelas setas.  Fonte: próprio autor.}
\label{fig:scattervtec}
\end{figure}


\begin{figure}[H]
\centering
\makebox[\textwidth][c]{\includegraphics[width=1.2\columnwidth]{./Figuras/vtec_dt_scatter.eps}}
\caption{Amostra dos sinais {\bf vtec\_dt} e {\bf S4}, no período entre 00:00 UT do dia 01/12/2013 até 12:00 UT do dia 04/12/2013. As linhas verticais vermelhas representam o por do sol, enquanto as amarelas representam o nascer do sol. Fonte: próprio autor.}
\label{fig:scattervtecdt}
\end{figure}

\begin{figure}[H]
\centering
\makebox[\textwidth][c]{\includegraphics[width=1.2\columnwidth]{./Figuras/vtec_dt2_scatter.eps}}
\caption{Amostra dos sinais {\bf vtec\_dt2} e {\bf S4}, no período entre 00:00 UT do dia 01/12/2013 até 12:00 UT do dia 04/12/2013. As linhas verticais vermelhas representam o por do sol, enquanto as amarelas representam o nascer do sol. Fonte: próprio autor.}
\label{fig:scattervtecdt2}
\end{figure}

\begin{figure}[H]
\centering
\makebox[\textwidth][c]{\includegraphics[width=1.2\columnwidth]{./Figuras/gvtec1_scatter.eps}}
\caption{Amostras dos sinais {\bf gvtec1} contra {\bf S4}, no período entre 00:00 UT do dia 01/12/2013 até 12:00 UT do dia 04/12/2013. As linhas verticais vermelhas representam o por do sol, enquanto as amarelas representam o nascer do sol. Fonte: próprio autor.}
\label{fig:scattergvtec1}
\end{figure}

\begin{figure}[H]
\centering
\makebox[\textwidth][c]{\includegraphics[width=1.2\columnwidth]{./Figuras/gvtec1_dt_scatter.eps}}
\caption{Amostras dos sinais {\bf gvtec1\_dt} e {\bf S4}, no período entre 00:00 UT do dia 01/12/2013 até 12:00 UT do dia 04/12/2013. As linhas verticais vermelhas representam o por do sol, enquanto as amarelas representam o nascer do sol. Fonte: próprio autor.}
\label{fig:scattergvtec1dt}
\end{figure}

\begin{figure}[H]
\centering
\makebox[\textwidth][c]{\includegraphics[width=1.2\columnwidth]{./Figuras/gvtec2_scatter.eps}}
\caption{Amostras dos sinais {\bf gvtec2} e {\bf S4}, no período entre 00:00 UT do dia 01/12/2013 até 12:00 UT do dia 04/12/2013. As linhas verticais vermelhas representam o por do sol, enquanto as amarelas representam o nascer do sol. Fonte: próprio autor.}
\label{fig:scattergvtec2}
\end{figure}

\begin{figure}[H]
\centering
\makebox[\textwidth][c]{\includegraphics[width=1.2\columnwidth]{./Figuras/gvtec2_dt_scatter.eps}}
\caption{Amostras dos sinais {\bf gvtec2\_dt} e {\bf S4}, no período entre 00:00 UT do dia 01/12/2013 até 12:00 UT do dia 04/12/2013. As linhas verticais vermelhas representam o por do sol, enquanto as amarelas representam o nascer do sol. Fonte: próprio autor.}
\label{fig:scattergvtec2dt}
\end{figure}

\begin{figure}[H]
\centering
\makebox[\textwidth][c]{\includegraphics[width=1.2\columnwidth]{./Figuras/vtec_and_vtec_dt.eps}}

\caption{Relação entre o {\bf vtec} e o {\bf vtec\_dt}, do dia 01/12/2013 até 05/12/2013. As linhas verticais vermelhas representam o por do sol, enquanto as amarelas representam o nascer do sol. Possíveis picos de pré reversão são indicados pelas setas. Os gráficos no meio apresentam normalização, para o intervalo $[0,1]$, enquanto os gráficos acima e abaixo apresentam seu valores reais.  Fonte: próprio autor.}
\label{fig:vtec_vtec_dt}
\end{figure}


\section{Análises: S4$\times$VTEC em São José dos Campos}

Utilizando as variáveis derivadas do vtec, realizou-se uma transformação de normalização para restringir o valor destas ao intervalo $[0,1]$. Os dados normalizados constituem um conjunto com 12.772 amostras ordenadas no tempo, destes os 772 últimos formaram um conjunto de validação, que foi utilizado para o cálculo do erro relativo médio do modelo. Os 12.000 elementos restantes foram divididos em dois conjuntos não-ordenadas num esquema de ``hold-out" aleatório (não ordenado no tempo), o de treinamento com 70\% dos dados e o de teste com 30\%. Em posse, de um conjunto de amostras de treinamento, segui-se para a avaliação dos estimadores, ou algoritmos de aprendizagem de máquina. Para este trabalho, testaram-se separadamente 3 algoritmos de aprendizagem de máquina, todos com a finalidade de estimar o S4 por meio de uma regressão, estes foram uma árvore, uma floresta aleatória e uma máquina de vetor de suporte. Cada estimador  foi avaliado utilizando ``10-fold cross validation". Os estimadores estão respectivamente nos notebooks 07\_analise\_sj2\_tree.ipynb, 07\_analise\_sj2\_random\_forest.ipynb e 07\_analise\_sj2\_svm.ipynb. Além disso, para todos estimadores realizou-se uma análise da sensibilidade aos atributos, removendo uma variável de cada vez, e verificando o desempenho de estimação/regressão para o novo estimador. 

O erro foi originalmente estimado utilizando o erro relativo médio, entretanto uma segunda abordagem utilizando uma matriz de confusão foi desenvolvida, para tal se considerou que quando para cada amostra no tempo, se o valor real e o estimado forem menores que um limiar tem-se verdadeiros negativos (TN); se o valor e o estimado forem maiores que o limiar tem-se um verdadeiro positivo (TV); se o valor real for maior que o do limiar e o estimado for menor, tem-se um falso negativo (FN); se o valor real for menor que o limiar enquanto o estimado maior, tem-se um falso positivo (FV). Na literatura, o limiar do índice S4 para ocorrência de cintilação é 0.2, isto é, todos os valores abaixo deste são tratados com ruídos e somente valores acima serão identificados como cintilação, portanto este valor foi adotado como limiar. Utilizando-se a matriz de confusão resultante, pode-se calcular a probabilidade de detecção (POD), a razão de falsos alarmes (FAR) e acurácia (ACC), cujas fórmulas são:

\begin{eqnarray}
\mbox{POD}&=&\mbox{TP}/(\mbox{TP}+\mbox{FN})\mbox{,}\\
\mbox{FAR}&=&\mbox{FP}/(\mbox{TP}+\mbox{FP})\mbox{,}\\ 
\mbox{ACC}&=&(\mbox{TN}+\mbox{TP})/(\mbox{TN}+\mbox{TP}+\mbox{FN}+\mbox{FP})\mbox{.}
\end{eqnarray}

Finalmente, realizou-se uma variação da análise de componentes principais (PCA - Principal Component Analsys), a qual objetiva reduzir o conjunto de atributos ou variáveis. A PCA é definida matematicamente como uma transformação linear ortogonal que mapeia os dados para um novo sistema de coordenadas por meio de uma sucessão rotações tal que a maior variância para qualquer projeção dos dados fique ao longo da primeira coordenada, a segunda maior variância fica ao longo da segunda coordenada, até que a menor variância fique para a última componente. Isso é feito a partir da decomposição em autovalores da matriz de covariância dos dados, ou por uma decomposição em valores singulares da matriz de dados. Uma decomposição baseada em autovalores consiste numa rotação no espaço multidimensional dos dados, que é o caso deste trabalho, utilizando-se o pacote ``psych'' do ambiente R. Entre diversas opções de rotação, foi utilizada a rotação ``varimax'', que busca uma nova base ortonormal, tal que cada vetor da nova base seja uma combinação linear de (preferivelmente) apenas alguns elementos da base original, ou seja, alguns pesos serão próximos de zero e outros próximos da unidade. Utilizando-se, então, a variância parcial de cada componente, tem-se uma medida indireta de quanto cada atributo é representativo no conjunto de atributos.

\section{Tamanho de passo para as diferenças finitas no tempo}\label{sec:tpdft}

É interessante observar o que acontece com a diferença finita no tempo quando são escolhidos diferentes passos, intervalos de tempo, para o cálculo. Tal análise é implementada pelo notebook 06\_analise\_fin\_dif\_vtec.ipynb, e consistiu em um laço que cálculo e armazenou as várias diferenças. Além disso, buscou-se um instante de referência com cintilação, para a construção dos gráficos comparativos, que fornecem o valor da derivada para o ponto de referência, com diferentes $\Delta{t}$. Tais gráficos também foram gerados para o {\bf gvtec1} e {\bf gvtec2}, os quais são apresentados na seção de resultados.

\section{Novas variáveis, novos testes}\label{sec:nvnt}

Utilizando os resultados da seção anterior \ref{sec:tpdft}, adicionou-se novas variáveis. O novo conjunto de variáveis é particionada em subconjuntos levando em consideração o momento de sua introdução, e sua formação. As variáveis definidas na seção \ref{sec:vdt} pertencem ao subconjunto das variáveis originais.

As variáveis pertencente ao subconjunto ``lag" correspondem à:

\begin{itemize}
\item {\bf gvtec1\_dt\_lag\_9} - diferença finita de primeira ordem no tempo do {\bf gvtec1} (diferença de VTEC entre São José dos Campos e Pirassununga) com passo 9 , calculada por $gvtec1_i-gvtec1_{i-9}$;
\item {\bf gvtec2\_dt\_lag\_20} - diferença finita de primeira ordem no tempo do {\bf gvtec2} (diferença de VTEC entre São José dos Campos e Brasília) com passo 20 , calculada por $gvtec2_i-gvtec2_{i-20}$.
\end{itemize}

Uma representação gráfica de uma amostra das variáveis {\bf gvtec1\_dt\_lag\_9} e {\bf gvtec2\_dt\_lag\_20} pode ser vista na figura \ref{fig:scattergvtec1_lag9} e  \ref{fig:scattergvtec2_lag_20}.

\begin{figure}[H]
\centering
\makebox[\textwidth][c]{\includegraphics[width=1.2\columnwidth]{./Figuras/gvtec1_lag_9_scatter.eps}}
\caption{Amostras dos sinais {\bf gvtec1\_dt\_lag\_9} e {\bf S4}, no período entre 12:00 UT do dia 01/12/2013 até 00:00 UT do dia 05/12/2013. As linhas verticais vermelhas representam o por do sol, enquanto as amarelas representam o nascer do sol. Fonte: próprio autor.}
\label{fig:scattergvtec1_lag9}
\end{figure}

\begin{figure}[H]
\centering
\makebox[\textwidth][c]{\includegraphics[width=1.2\columnwidth]{./Figuras/gvtec2_lag_20_scatter.eps}}
\caption{Amostras dos sinais {\bf gvtec2\_dt\_lag\_20} e {\bf S4}, no período entre 12:00 UT do dia 01/12/2013 até 00:00 UT do dia 05/12/013. As linhas verticais vermelhas representam o por do sol, enquanto as amarelas representam o nascer do sol. Fonte: próprio autor.}
\label{fig:scattergvtec2_lag_20}
\end{figure}

As variáveis pertencente ao subconjunto ``tempo" correspondem à:

\begin{itemize}
\item {\bf state\_day} - indica que uma amostra está entre o nascer do sol e o por do sol;
\item {\bf state\_night} - indica que uma amostra está entre o por do sol e 00:00 UT;
\item {\bf state\_dawn} - indica que uma amostra está entre as 00:00 UT e o nascer do sol.
\end{itemize}

Cada um dessas variáveis pode assumir o valor 0.0 ou 1.0, a primeira indica falso e a segunda verdadeiro. Para sua avaliação uma amostra é tem seu índice decomposto em uma parte associada a data e a outra a hora, então usando a data e a localização da estação que forneceu a medida se determina o horário de nascer e por do sol, a última etapa é avaliação da parte das horas em relações as condições que definem cada variável, isto é, por exemplo, testa-se se a hora está entre o nascer e o por do sol.

As variáveis pertencente ao subconjunto ``mdv1" correspondem à:

\begin{itemize}
\item {\bf vm1} - valor médio do VTEC calculado entre o nascer e o por do sol;
\item {\bf vd1} - desvio padrão do VTEC calculado entre o nascer e o por do sol.
\end{itemize}

Os valores de {\bf vm1} e {\bf vd1} tem validade até o nascer do sol do dia seguinte, portanto, as amostras neste intervalo intermediário também irão assumir este valor.

As variáveis pertencente ao subconjunto ``mdv2" correspondem à:

\begin{itemize}
\item {\bf vm2} - valor médio do VTEC calculado entre o nascer e o por do sol mais uma hora;
\item {\bf vd2} - desvio padrão do VTEC calculado entre o nascer e o por do sol mais uma hora.
\end{itemize}

De modo análogo as variáveis {\bf vm1} e {\bf vd1} as amostras nos intervalos intermediário irão assumir o valor de {\bf vm2} e {\bf vd2}.

Também foi definida uma variável denominada {\bf vtec\_dt\_lag\_3} que correspondem a uma diferença finita com passo 3 do {\bf vtec}, dada por $vtec_i-vtec_{i-3}$;

Estas novas variáveis são implementadas no notebook 08\_update\_analise\_sj2\_data.ipynb. Em, posse dos subconjuntos que agrupam essas variáveis, o próximos passo foi a realização de uma bateria de testes, onde foram utilizadas diferentes uniões deste subconjuntos, assim como algumas variáveis do subconjunto original de maneira independente, independente da configuração o algoritmo adotado foi a floresta aleatória. Os testes estão implementados no notebook 09\_all\_analise.ipynb e os todos os resultados, isto é, estimações dos valores de S4 serão apresentados em uma tabela.

%\chapter{TESTES E RESULTADOS}

Os dados utilizados neste trabalho se restringem ao período de atividade solar máxima de um ciclo solar, no caso, de 01/12/2013 até 28/02/2014, totalizando 90 dias/noites, destes aproximadamente 5 dias/noites foram usados para validação, aproximadamente 59 foram utilizados para treinamento e 25 para testes. Existe um total de 12772 amostras neste período, correspondendo a frequência de uma amostra a cada 10 min. 1382 amostras apresentam valores de S4 acima de 0.2, enquanto as demais 11390 apresentam cintilação inferior ou igual a 0.2. Devido a problemas técnicos e de equipamentos, sensores, existem muitas lacunas nestes dados e, portanto, técnicas de pré-processamento como suavização precisaram ser aplicadas.

\section{Estimação do S4 a partir do VTEC e suas variáveis derivadas}

Um dos primeiros resultados que foi buscado é o mapeamento entre VTEC e S4 por meio de uma regressão não linear que foi efetuada por diversos algoritmos de aprendizagem de máquina diferentes, os melhores resultados foram obtidos pela floresta aleatória. O objetivo dessa busca foi determinar variáveis derivadas do VTEC assim como ele próprio que poderiam ser utilizadas como indicativo do valor de S4, sendo que posteriormente, em novos estudos, as melhores variáveis poderiam ser utilizadas para predizer o valor de S4. As Figuras \ref{fig:regressioncart}, \ref{fig:regressionrf} e \ref{fig:regressionsvm} apresentam os resultados obtidos respectivamente pela árvore de decisão, pela floresta aleatória e pela máquina de vetor de suporte. 

\begin{figure}
\centering
\makebox[\textwidth][c]{\includegraphics[width=1.2\columnwidth]{./Figuras/regression_cart.eps}}
\caption{Regressão da variável S4 por meio de uma árvore CART, para São José dos Campos - SP, no período ente 24/02/2014 até 01/03/2014. As linhas verticais vermelhas representam o por do sol, enquanto as amarelas representam o nascer do sol. Fonte: próprio autor.}
\label{fig:regressioncart}
\end{figure}

\begin{figure}
\centering
\makebox[\textwidth][c]{\includegraphics[width=1.2\columnwidth]{./Figuras/regression_random_florest.eps}}
\caption{Regressão da variável S4 por meio de uma floresta aleatória, para São José dos Campos - SP, no período ente 24/02/2014 até 01/03/2014. As linhas verticais vermelhas representam o por do sol, enquanto as amarelas representam o nascer do sol. Fonte: próprio autor.}
\label{fig:regressionrf}
\end{figure}

\begin{figure}
\centering
\makebox[\textwidth][c]{\includegraphics[width=1.2\columnwidth]{./Figuras/regression_svm.eps}}
\caption{Regressão da variável S4 por meio de uma máquina de vetor de suporte, para São José dos Campos - SP, no período ente 24/02/2014 até 01/03/2014. As linhas verticais vermelhas representam o por do sol, enquanto as amarelas representam o nascer do sol. Fonte: próprio autor.}
\label{fig:regressionsvm}
\end{figure}

Uma análise visual dos resultados indica que as técnicas baseadas em árvores de decisão tiveram um resultado melhor. Para uma avaliação numérica entre os modelos inicialmente se optou por utilizar o erro relativo médio entre o valor real e o gerado pela regressão, o que foi de 17.4\% para árvore de decisão, 13.9\% para a floresta aleatória e 17.4\% para a máquina de vetor de suporte.

Considerando-se a estimação do S4 a partir das  7 variáveis descritas, repetiu-se a estimação eliminando-se uma delas de cada vez, de forma a avaliar sua relevância na estimação. A Tabela \ref{tab:relativeerror} apresenta o erro relativo médio do índice S4 entre a série real e a predita para cada variável descartada, considerando-se na estimação as demais 6 variáveis. Observa-se que o erro aumenta para qualquer variável descartada, em relação à estimação com todas as 7 variáveis, sendo que os erros que aparecem na tabela são similares para cada algoritmo utilizado, demonstrando que as variáveis tem relevância similar na estimação do índice S4. Entretanto, esse erro é um pouco maior ao se descartar a variável {\bf vtec}, que pode ser considerada a mais relevante.

\begin{table}
\addtolength{\leftskip} {-2cm} % increase (absolute) value if needed
\addtolength{\rightskip}{-2cm}
\small
\begin{tabular}{|l|c|c|c|c|c|c|c|}
\hline
&  {\bf vtec} &  {\bf vtec\_dt} &  {\bf vtec\_dt2} &  {\bf gvtec1} &  {\bf gvtec1\_dt} &  {\bf gvtec2} &  {\bf gvtec2\_dt} \\ \hline
Árvore de decisão & 19.1 & 15.8 & 16.5 & 16.5 & 16.0 & 19.7 & 17.3 \\ \hline
Floresta aleatória & 18.2 & 15.3 & 14.7 & 15.9 & 14.6 & 15.5 & 14.8 \\ \hline
Maquina de vetor de suporte & 34.3 & 32.6 & 30.2 & 33.7 & 29.7 & 34.4 & 29.5 \\ \hline
\end{tabular}
\vspace{12pt}
\caption{Erro relativo obtido pelo modelo após a exclusão da variável que nomeia a coluna. A estação de referência foi São José dos Campos, no período entre 24/02/2014 e 01/03/2014. Fonte: próprio autor.}
\label{tab:relativeerror}
\end{table}

A mesma conclusão pode ser obtida utilizando-se a análise de componentes principais (PCA), cujo resultado aparece na Tabela \ref{tab:pca1}, a qual apresenta os pesos de cada uma das 7 variáveis em cada uma das 7 rotações. Nota-se que cada variável tem um peso próximo ao valor unitário em uma única rotação. Além disso, a ferramenta computacional calculou os valores resultantes, para cada rotação, da soma obtida pela aplicação dos pesos em todas as amostras da base de dados, obtendo-se assim 7 séries de valores e as respectivas médias e variâncias. Estas últimas são apresentadas também na Tabela \ref{tab:pca1}, com valores normalizados. Observa-se que as variâncias correspondentes às 7 rotações são iguais, demonstrando que as variáveis são igualmente relevantes no espaço de atributos.

\begin{table}[hhh]
\begin{center}
\begin{tabular}{|l|c|c|c|c|c|c|c|} 
\hline
                 &   RC3 &   RC4 &  RC6 &   RC5 &   RC2 &   RC7 &   RC1 \\ \hline
{\bf vtec      } & -0.14 &  0.98 & 0.02 & -0.07 & -0.07 & -0.01 &  0.10 \\ \hline
{\bf vtec\_dt  } &  0.07 &  0.02 & 0.95 &  0.05 &  0.17 &  0.18 &  0.14 \\ \hline
{\bf vtec\_dt2 } &  0.98 & -0.14 & 0.06 & -0.02 & -0.01 & -0.05 & -0.08 \\ \hline 
{\bf gvtec1    } & -0.06 & -0.02 & 0.19 &  0.04 & -0.02 &  0.92 &  0.33 \\ \hline
{\bf gvtec1\_dt} & -0.02 & -0.06 & 0.04 &  0.99 &  0.14 &  0.04 &  0.01 \\ \hline
{\bf gvtec2    } & -0.09 &  0.12 & 0.15 &  0.01 &  0.03 &  0.32 &  0.92 \\ \hline
{\bf gvtec2\_dt} & -0.01 & -0.08 & 0.16 &  0.15 &  0.97 & -0.02 &  0.02 \\ \hline
\end{tabular}

\vspace{12pt}

\begin{tabular}{|l|c|c|c|c|c|c|c|}
\hline
                 &   RC3 &   RC4 &  RC6 &   RC5 &   RC2 &   RC7 &   RC1 \\ \hline
Proporção Explicada  & 0.14 & 0.14 & 0.14 & 0.14 & 0.14 & 0.14 & 0.14 \\ \hline
Proporção Cumulativa & 0.14 & 0.29 & 0.43 & 0.57 & 0.71 & 0.86 & 1.00 \\ \hline
\end{tabular}
\end{center}
\vspace{12pt}
\caption{Análise em termos Rotações, utilizando a técnica de Componentes Principais. RC1 até RC7 indicam as componentes da rotação. Note que cada componente apresenta contribuição predominante de apenas uma variável, e que a proporção de variância explicada é igual para todas as componentes. Fonte: próprio autor.}
\label{tab:pca1}
\end{table}


O instante de referência utilizada para avaliar diferentes intervalos de tempo de forma a otimizar a derivada temporal calculada por diferenças finitas foi 14/12/2013 às 00:00:00 UT. Este instante corresponde a um pico de cintilação com valor aproximado de 0.55. O intervalo foi escolhido tal que fosse possível varrer um período de um dia. As Figuras \ref{fig:findifgvtec1} até \ref{fig:findifgvtec2} exibem os resultados dessa avaliação.

\begin{figure}[H]
\centering
\makebox[\textwidth][c]{\includegraphics[width=1.2\columnwidth]{./Figuras/fin_dif_gvtec1.eps}}
\caption{Amplitude da diferença no tempo da variável {\bf gvtec1} para diferentes intervalos. Fonte: próprio autor.}
\label{fig:findifgvtec1}
\end{figure}

\begin{figure}[H]
\centering
\makebox[\textwidth][c]{\includegraphics[width=1.2\columnwidth]{./Figuras/fin_dif_gvtec2.eps}}
\caption{Amplitude da diferença no tempo da variável {\bf gvtec2} para diferentes intervalos. Fonte: próprio autor.}
\label{fig:findifgvtec2}
\end{figure}

Pode-se notar nestas figuras que variações no valor da amplitude da diferença, tais variações são esperadas devido a presença de periodicidade ao longo do tempo nestes atributos. Os picos, máximos, correspondem as maiores diferenças de sinais, isto é, situações, por exemplo, onde o valor do {\bf gvtec1} no passado é menor do que no ponto de referência. Finalmente, seria interessante avaliar modelos tanto com valores de mínimo, quanto de máximo, pois correspondem a indicativos de janelas no tempo que podem fornecer informação interessante para a regressão. Uma segunda rodada de testes vai levar em consideração variáveis derivadas da análise destes gráficos, e discussões realizados com colaboradores.

\section{Novos testes, novos resultados}

Novos testes foram realizados para estimação do índice S4 a partir do {\bf vtec} e suas variáveis derivadas, utilizando-se subconjuntos de novas variáveis que foram definidas na Seção \ref{sec:nvnt}. Em todos os testes optou-se pela regressão por meio de floresta aleatória. Os resultados aparecem na Tabela \ref{tab:final_result} considerando-se o valor do índice S4 categorizado e um limiar de ocorrência de cintilação de 0.2. O operador $-$ indica que dado um conjunto uma variável foi removida, o operador $+$ indica a união dos subconjuntos, ou combinação de variáveis. Variáveis individuais são denotadas por negrito, enquanto os subconjuntos adotam letras normais.

\begin{table}
\begin{center}
\begin{tabular}{|l|c|c|c|c|} 
\hline
 
	                          & Erro médio relativo	& POD	& FAR	& ACC  \\ \hline
original	                               & 14.34\%	& 0.88	& 0.55	& 0.84 \\ \hline
original - {\bf vtec}	                       & 17.78\%	& 0.85	& 0.63	& 0.79 \\ \hline
original - {\bf vtec\_dt}	               & 14.81\%	& 0.93	& 0.50	& 0.87 \\ \hline
original - {\bf vtec\_dt2}	               & 14.97\%	& 0.83	& 0.57	& 0.83 \\ \hline 
original - {\bf gvtec1}	                       & 16.22\%	& 0.80	& 0.64	& 0.79 \\ \hline
original - {\bf gvtec1\_dt}	               & 13.84\%	& 0.83	& 0.50	& 0.87 \\ \hline
original - {\bf gvtec2}	                       & 16.09\%	& 0.92	& 0.60	& 0.82 \\ \hline
original - {\bf gvtec2\_dt}	               & 14.81\%	& 0.94	& 0.49	& 0.88 \\ \hline
original - {\bf gvtec2} - {\bf gvtec2\_dt}     & 15.81\%        & 0.81  & 0.64  & 0.80 \\ \hline
original + tempo	                       & 13.54\%	& 0.88	& 0.59	& 0.83 \\ \hline
original + {\bf gvtec1\_dt\_lag\_9}            & 14.78\%	& 0.90	& 0.62	& 0.81 \\ \hline
original + {\bf gvtec2\_dt\_lag\_20}           & 14.34\%	& 0.81	& 0.63	& 0.81 \\ \hline
original + lag	                               & 15.53\%	& 0.86	& 0.63	& 0.81 \\ \hline
original + mdv1	                               & 13.69\%	& 0.69	& 0.60	& 0.84 \\ \hline
original + mdv2	                               & 15.40\%	& 0.69	& 0.66	& 0.80 \\ \hline
original + tempo + lag	                       & 15.00\%	& 0.85	& 0.64	& 0.80 \\ \hline
original + tempo + mdv2	                       & 13.62\%	& 0.76	& 0.62	& 0.82 \\ \hline
original + tempo + mdv2 + lag 	               & 12.20\%	& 0.77	& 0.54	& 0.85 \\ \hline
original + tempo + lag + mdv1 + mdv2	       & 12.23\%	& 0.77	& 0.54	& 0.86 \\ \hline
{\bf vtec}	                               & 15.43\%	& 0.29	& 0.70	& 0.82 \\ \hline
{\bf vtec} + {\bf gvtec1\_dt\_lag\_9}          & 16.43\%        & 0.67  & 0.67  & 0.79 \\ \hline
{\bf vtec} + {\bf gvtec2\_dt\_lag\_20}         & 14.81\%        & 0.56  & 0.64  & 0.82 \\ \hline
{\bf vtec} + {\bf vtec\_dt} + {\bf vtec\_dt2}  & 16.65\%	& 0.71	& 0.63	& 0.81 \\ \hline
{\bf vtec} + {\bf gvtec1} + {\bf gvtec2}       & 15.35\%	& 0.67	& 0.54	& 0.86 \\ \hline
{\bf vtec} + tempo	                       & 15.73\%	& 0.75	& 0.54	& 0.86 \\ \hline
{\bf vtec} + tempo + mdv1	               & 13.73\%	& 0.41	& 0.58	& 0.86 \\ \hline
{\bf vtec} + tempo + lag	               & 13.54\%	& 0.79	& 0.52	& 0.86 \\ \hline
{\bf vtec} + tempo + lag + mdv1	               & 12.21\%	& 0.71	& 0.48	& 0.88 \\ \hline
{\bf vtec} + tempo + lag + mdv2	               & 13.58\%	& 0.61	& 0.52	& 0.87 \\ \hline
{\bf vtec} + tempo + lag + mdv1 + mdv2	       & 11.95\%	& 0.52	& 0.54	& 0.86 \\ \hline
\end{tabular}
\end{center}
\vspace{12pt}
\caption{Desempenho da regressão para a estimação do índice S4, para diferentes conjuntos de atributos, considerando-se o limiar de cintilação de S4$=0.2$. Fonte: próprio autor.}
\label{tab:final_result}
\end{table}

Observando a Tabela \ref{tab:final_result} se nota que a adição individual do subconjunto tempo melhora o resultado, enquanto a adição do subconjunto ``lag"\, piora o resultado, resultado análogo para os subconjuntos mdv1 e mdv2. A variável {\bf gvtec2\_dt} parece uma boa candidata a exclusão, pois sua remoção apesar de levar a um aumento no erro relativo, este é pequeno, contudo, ouve significativa melhora nos valores de POD e FAR. {\bf gvtec1\_dt} também poderia ser removida, pois sua exclusão leva a um diminuição do erro relativo, apesar da redução do POD. A adição do subconjunto tempo melhora não somente os resultados do subconjunto original, como também da variável {\bf vtec}. A adição das variáveis {\bf vtec\_dt} e {\bf vtec\_dt2} levam a melhoras mais significativas para o POD e o FAR do as variáveis {\bf gvtec1} e {\bf gvtec1}. Os resultados mostram claramente que o subconjunto mdv1 é mais importante que o mdv2. A combinação dos subconjuntos lag, tempo juntos do original mostram relativa piora em relação ao original, comparando com o subconjunto original mais tempo, conclui-se que a adição do lag pode não ser adequada. Finalmente, seria interessante utilizar um algorítimo que teste cada combinação possível de variáveis de forma a determinar o melhor subconjunto de variáveis. 

A presença de baixos valores de POD e FAR com altos valores de ACC é um forte indicativo da predominância de verdadeiros negativos (TN) que neste caso correspondem a valores com S4 $<0.2$. A Tabela \ref{tab:final_result1} apresenta os resultados considerando o limiar de S4 como sendo de 0.5. Pode-se notar uma variação mais acentuada para o POD e o FAR do que o ACC, quando comparado com os resultados na Tabela \ref{tab:final_result}.

A Tabela \ref{tab:final_result2} apresenta uma variação dos resultados obtidos quando se troca a variável {\bf vtec\_dt} por {\bf vtec\_dt\_lag\_3}. Os testes realizados, então, foram aqueles que apresentavam essa variável, seja internamente na definição do conjunto original, quanto uma variável adicional. A principal característica que se nota foi em uma redução do POD e um aumento do FAR.


        A Tabela \ref{tab:final_result3} apresenta os resultados dos testes, quando no treinamento só foram consideradas amostras correspondentes ao período noturno e à madrugada, isto é, excluindo as amostras referentes ao período diurno. Nestes resultados, existem também variações do POD e do FAR, mas nota-se principalmente uma queda acentuada no ACC, uma vez que o número de amostras com S2 inferior a 0.2, correspondentes aos verdadeiros negativos, referem-se predominantemente ao período diurno, constituindo justamente as amostras descartadas.


      Existem dois pontos em comum observados nos testes realizados. Primeiramente, a variável {\bf vtec\_dt} que correspondem à diferença finita no tempo da variável{\bf vtec} não apresentou a influência e contribuição esperadas, mesmo quando sendo calculada para intervalos de 30 minutos. Esta expectativa vem da tendência de diminuição do VTEC ao longo do tempo precedendo ou acompanhando a cintilação, é claramente observável nas figuras que mostram a evolução do VTEC e do índice S4, de forma que o atributo {\bf vtec\_dt} deve ser em breve revisto ou recalculado,  de forma a se aproveitar essa informação na regressão. Em segundo lugar, o gradiente espacial foi avaliado apenas em relação a duas localizações, Pirassununga e Brasília, resultando, respectivamente, nas variáveis  {\bf gvtec1} e {\bf gvtec2}, Observou-se que a contribuição destas varáveis na regressão foi bem diferente, sendo que esta última não apresentou a relevância esperada, o que pode decorrer do fato de Brasília estar num meridiano magnético mais a oeste que o meridiano de São José dos Campos comparando-se com  meridiano magnético de Pirassununga. Entretanto, deve-se considerar que a evolução espaço-temporal das bolhas ionosféricas, bem como sua forma e extensão, podem variar muito, o que implica em diferentes influências do VTEC medido em diferentes estações de medição na regressão para estimar o índice S4 em São José dos Campos. Assim, embora seja desejável utilizar dados de diversas estações, não se pode esperar que sua contribuição nessa estimação seja a mesma.


\begin{table}
\begin{center}
\begin{tabular}{|l|c|c|c|c|} 
\hline
	                          & Erro médio relativo	& POD	& FAR	& ACC  \\ \hline
original                                        & 15.00\%  &  0.81  &  0.62  &  0.81 \\ \hline
original - {\bf vtec}                           & 18.17\%  &  0.81  &  0.69  &  0.75 \\ \hline
original - {\bf vtec\_dt}                       & 14.71\%  &  0.83  &  0.60  &  0.82 \\ \hline
original - {\bf vtec\_dt2}                      & 14.52\%  &  0.80  &  0.65  &  0.79 \\ \hline
original - {\bf gvtec1}                         & 15.20\%  &  0.77  &  0.64  &  0.81 \\ \hline
original - {\bf gvtec1\_dt}                     & 14.55\%  &  0.80  &  0.60  &  0.83 \\ \hline
original - {\bf gvtec2}                         & 16.41\%  &  0.87  &  0.66  &  0.77 \\ \hline
original - {\bf gvtec2\_dt}                     & 14.18\%  &  0.86  &  0.58  &  0.83 \\ \hline
original - {\bf gvtec2} - {\bf gvtec2\_dt}      & 15.86\%  &  0.88  &  0.63  &  0.80 \\ \hline
original + tempo                                & 14.13\%  &  0.86  &  0.59  &  0.83 \\ \hline
original + {\bf gvtec1\_dt\_lag\_9}             & 14.62\%  &  0.89  &  0.60  &  0.83 \\ \hline
original + {\bf gvtec2\_dt\_lag\_20}            & 14.78\%  &  0.86  &  0.63  &  0.80 \\ \hline
original + lag                                  & 15.45\%  &  0.87  &  0.66  &  0.78 \\ \hline
original + mdv1                                 & 14.53\%  &  0.76  &  0.62  &  0.82 \\ \hline
original + mdv2                                 & 14.11\%  &  0.73  &  0.63  &  0.81 \\ \hline
original + tempo + lag                          & 14.13\%  &  0.86  &  0.60  &  0.83 \\ \hline
original + tempo + mdv2                         & 13.33\%  &  0.76  &  0.59  &  0.84 \\ \hline
original + tempo + mdv2 + lag                   & 12.94\%  &  0.78  &  0.60  &  0.83 \\ \hline
original + tempo + lag + mdv1 + mdv2            & 13.20\%  &  0.85  &  0.58  &  0.84 \\ \hline
{\bf vtec}                                      & 14.89\%  &  0.29  &  0.75  &  0.81 \\ \hline
{\bf vtec} + {\bf gvtec1\_dt\_lag\_9}           & 17.15\%  &  0.71  &  0.66  &  0.79 \\ \hline
{\bf vtec} + {\bf gvtec2\_dt\_lag\_20}          & 14.51\%  &  0.56  &  0.64  &  0.82 \\ \hline
{\bf vtec} + {\bf vtec\_dt} + {\bf vtec\_dt2}   & 14.90\%  &  0.67  &  0.62  &  0.83 \\ \hline
{\bf vtec} + {\bf gvtec1} + {\bf gvtec2}        & 14.83\%  &  0.61  &  0.64  &  0.82 \\ \hline
{\bf vtec} + tempo                              & 15.95\%  &  0.80  &  0.55  &  0.85 \\ \hline
{\bf vtec} + tempo + mdv1                       & 14.05\%  &  0.55  &  0.56  &  0.85 \\ \hline
{\bf vtec} + tempo + lag                        & 14.30\%  &  0.80  &  0.52  &  0.87 \\ \hline
{\bf vtec} + tempo + lag + mdv1                 & 13.27\%  &  0.69  &  0.52  &  0.87 \\ \hline
{\bf vtec} + tempo + lag + mdv2                 & 13.07\%  &  0.83  &  0.53  &  0.86 \\ \hline
{\bf vtec} + tempo + lag + mdv1 + mdv2          & 13.42\%  &  0.78  &  0.55  &  0.86 \\ \hline
\end{tabular}
\end{center}
\vspace{12pt}
\caption{Desempenho da regressão para a estimação do índice S4, para diferentes conjuntos de atributos, considerando-se o limiar de cintilação de S4$=0.5$. Fonte: próprio autor.}
\label{tab:final_result1}
\end{table}

\begin{table}
\begin{center}
\begin{tabular}{|l|c|c|c|c|} 
\hline
	                          & Erro médio relativo	& POD	& FAR	& ACC  \\ \hline
original                                                & 14.20\% &  0.80  & 0.60  & 0.83 \\ \hline
original - {\bf vtec}                                   & 17.22\% &  0.82  & 0.68  & 0.77 \\ \hline
original - {\bf vtec\_dt\_lag\_3}                       & 14.31\% &  0.84  & 0.61  & 0.82 \\ \hline
original - {\bf vtec\_dt2}                              & 14.55\% &  0.83  & 0.60  & 0.83 \\ \hline
original - {\bf gvtec1}                                 & 14.92\% &  0.80  & 0.63  & 0.81 \\ \hline
original - {\bf gvtec1\_dt}                             & 13.85\% &  0.80  & 0.59  & 0.83 \\ \hline
original - {\bf gvtec2}                                 & 16.12\% &  0.85  & 0.65  & 0.79 \\ \hline
original - {\bf gvtec2\_dt}                             & 13.56\% &  0.82  & 0.58  & 0.84 \\ \hline
original - {\bf gvtec2} - {\bf gvtec2\_dt}              & 16.30\% &  0.86  & 0.64  & 0.79 \\ \hline
original + tempo                                        & 13.22\% &  0.78  & 0.57  & 0.84 \\ \hline
original + {\bf gvtec1\_dt\_lag\_9}                     & 15.24\% &  0.89  & 0.61  & 0.82 \\ \hline
original + {\bf gvtec2\_dt\_lag\_20}                    & 15.14\% &  0.88  & 0.62  & 0.81 \\ \hline
original + lag                                          & 14.24\% &  0.85  & 0.63  & 0.80 \\ \hline
original + mdv1                                         & 14.35\% &  0.74  & 0.61  & 0.83 \\ \hline
original + mdv2                                         & 14.95\% &  0.80  & 0.60  & 0.83 \\ \hline
original + tempo + lag                                  & 13.31\% &  0.87  & 0.53  & 0.85 \\ \hline
original + tempo + mdv2                                 & 12.68\% &  0.85  & 0.54  & 0.86 \\ \hline
original + tempo + mdv2 + lag                           & 13.94\% &  0.60  & 0.60  & 0.83 \\ \hline
original + tempo + lag + mdv1 + mdv2                    & 12.90\% &  0.78  & 0.58  & 0.84 \\ \hline
{\bf vtec} + {\bf vtec\_dt\_lag\_3} + {\bf vtec\_dt2}   & 15.51\% &  0.71  & 0.63  & 0.81 \\ \hline
\end{tabular}
\end{center}
\vspace{12pt}
\caption{Desempenho da regressão para a estimação do índice S4, para diferentes conjuntos de atributos, considerando-se o limiar de cintilação de S4$=0.2$, trocando-se a variável {\bf vtec\_dt} por {\bf vtec\_dt\_lag\_3}. Fonte: próprio autor.}
\label{tab:final_result2}
\end{table}

\begin{table}
\begin{center}
\begin{tabular}{|l|c|c|c|c|} 
\hline
 
	                          & Erro médio relativo	& POD	& FAR	& ACC  \\ \hline
original                                      &  23.19\%  & 0.82  & 0.53  & 0.70 \\ \hline
original - {\bf vtec}                         &  29.84\%  & 0.82  & 0.59  & 0.63 \\ \hline
original - {\bf vtec\_dt}                     &  23.81\%  & 0.87  & 0.50  & 0.72 \\ \hline
original - {\bf vtec\_dt2}                    &  24.14\%  & 0.84  & 0.55  & 0.67 \\ \hline
original - {\bf gvtec1}                       &  23.84\%  & 0.83  & 0.56  & 0.66 \\ \hline
original - {\bf gvtec1\_dt}                   &  21.64\%  & 0.84  & 0.47  & 0.75 \\ \hline
original - {\bf gvtec2}                       &  26.41\%  & 0.86  & 0.56  & 0.65 \\ \hline
original - {\bf gvtec2\_dt}                   &  23.64\%  & 0.85  & 0.49  & 0.73 \\ \hline
original - {\bf gvtec2} - {\bf gvtec2\_dt}    &  27.09\%  & 0.85  & 0.60  & 0.63 \\ \hline
original + tempo                              &  24.53\%  & 0.89  & 0.55  & 0.67 \\ \hline
original + {\bf gvtec1\_dt\_lag\_9}           &  22.75\%  & 0.84  & 0.53  & 0.70 \\ \hline
original + {\bf gvtec2\_dt\_lag\_20}          &  23.25\%  & 0.87  & 0.53  & 0.69 \\ \hline
original + lag                                &  22.67\%  & 0.88  & 0.52  & 0.70 \\ \hline
original + mdv1                               &  22.53\%  & 0.83  & 0.49  & 0.73 \\ \hline
original + mdv2                               &  20.60\%  & 0.76  & 0.48  & 0.75 \\ \hline
original + tempo + lag                        &  23.70\%  & 0.88  & 0.54  & 0.68 \\ \hline
original + tempo + mdv2                       &  20.49\%  & 0.69  & 0.47  & 0.75 \\ \hline
original + tempo + mdv2 + lag                 &  24.38\%  & 0.75  & 0.53  & 0.70 \\ \hline
original + tempo + lag + mdv1 + mdv2          &  18.44\%  & 0.79  & 0.44  & 0.78 \\ \hline
{\bf vtec}                                    &  26.72\%  & 0.62  & 0.59  & 0.65 \\ \hline
{\bf vtec} + {\bf gvtec1\_dt\_lag\_9}         &  31.83\%  & 0.81  & 0.61  & 0.60 \\ \hline
{\bf vtec} + {\bf gvtec2\_dt\_lag\_20}        &  23.77\%  & 0.74  & 0.53  & 0.70 \\ \hline
{\bf vtec} + {\bf vtec\_dt} + {\bf vtec\_dt2} &  27.08\%  & 0.84  & 0.55  & 0.67 \\ \hline
{\bf vtec} + {\bf gvtec1} + {\bf gvtec2}      &  21.68\%  & 0.66  & 0.46  & 0.75 \\ \hline
{\bf vtec} + tempo                            &  29.69\%  & 0.83  & 0.54  & 0.69 \\ \hline
{\bf vtec} + tempo + mdv1                     &  20.45\%  & 0.54  & 0.48  & 0.74 \\ \hline
{\bf vtec} + tempo + lag                      &  24.15\%  & 0.81  & 0.51  & 0.72 \\ \hline
{\bf vtec} + tempo + lag + mdv1               &  23.56\%  & 0.58  & 0.56  & 0.69 \\ \hline
{\bf vtec} + tempo + lag + mdv2               &  24.77\%  & 0.56  & 0.52  & 0.72 \\ \hline
{\bf vtec} + tempo + lag + mdv1 + mdv2        &  20.52\%  & 0.51  & 0.52  & 0.71 \\ \hline
\end{tabular}
\end{center}
\vspace{12pt}
\caption{Desempenho da regressão para a estimação do índice S4, para diferentes conjuntos de atributos, considerando-se o limiar de cintilação de S4$=0.2$, e somente o período entre o por do sol e o nascer do sol. Fonte: próprio autor.}
\label{tab:final_result3}
\end{table}

\clearpage

\section{Matriz de correlação}

Além da seção anterior, na qual a estimação do índice S4 a partir do {\bf vtec} e suas variáveis derivadas foi utilizada indiretamente para avaliar a correlação do S4 com estas variáveis, apresenta-se aqui o cálculo da correlação utilizando-se o coeficiente de correlação de Pearson (r), definido para N amostras de um par de variáveis $x$ e $y$, como sendo: 

\begin{equation}
 r = \frac{\sum_{i=1}^{N}(x_i-\overline{x})(y_i-\overline{y})}{\sqrt{\sum_{i=1}^{N}(x_i-\overline{x})^2}\sqrt{\sum_{i=1}^{N}(y_i-\overline{y})^2}}~
 \end{equation}
 
onde $\overline{x}=\frac{1}{N}\sum_{i=1}^Nx_i$ indica o valor médio de $x$ e analogamente para $y$.

As correlações duas-a-duas variáveis, incluindo o índice S4, aparecem na Figura \ref{fig:matriz_corr}, sendo a matriz de correlação apresentada numa escala de cores.

\begin{figure}[hhh]
\centering
\makebox[\textwidth][c]{\includegraphics[width=1.3\columnwidth]{./Figuras/matriz_corr.eps}}
\caption{Matriz de correlação para o índice S4 e o {\bf vtec} e suas variáveis derivadas. Fonte: próprio autor.}
\label{fig:matriz_corr}
\end{figure}

Pode-se observar que essa representação em cores da matriz de correlação confirma as análises feitas ao se utilizar essas variáveis na estimação do S4 por regressão apresentada na seção anterior. Ou seja, a correlação praticamente nula do S4 com as variáveis {\bf vm1}, {\bf vm2}, {\bf vd1}, {\bf vd2}, a baixa correlação do S4 com {\bf gvtec2}, {\bf gvtec2\_dt} e {\bf vtec\_dt2}, além das anti-correlações do S4 com {\bf gvtec1}, {\bf vtec\_dt} e {\bf vtec\_dt\_lag\_3}, embora estas duas últimas tenham sido menores em módulo.    


%\chapter{CONCLUSÃO}

Este trabalho busca correlacionar o índice de cintilação S4 com o conteúdo eletrônico total vertical VTEC e suas variáveis derivadas, de forma a prover subsídios para uma possível futura estimação de S4 a partir destas variáveis e de outras relacionadas ao estado da ionosfera terrestre, não apresentadas aqui. Estudos desenvolvidos anteriormente para realizar essa estimação \cite{REZENDE:2009, GLAUSTON:2014, GLAUSTON:2015} poderiam então ser revistos e ampliados com a inclusão do VTEC e suas variáveis derivadas. Isso justifica-se uma vez que a ocorrência de cintilação é causada por zonas de rarefação na ionosfera, denominadas bolhas ionosféricas, nas quais o valor de VTEC é obviamente baixo.

Analisar a evolução espaço-temporal das bolhas ionosféricas, como previsto inicialmente, não foi possível, pois demandaria uma rede geograficamente muito mais densa de estações de medição de S4 e de VTEC. Entretanto, foi possível correlacionar o índice S4 com o VTEC e suas variáveis derivadas, adotando-se uma estratégia de tentar estimar S4 a partir dessas variáveis por meio de uma técnica de regressão. Esta análise foi efetuada para um estudo de caso correspondente ao intervalo de tempo de 3 meses em um período de atividade magnética calma para a cidade de São José dos Campos, sendo avaliadas 3 técnicas de regressão, floresta aleatória, árvore de regressão CART e máquina de vetor de suporte, sendo a primeira a que apresentou melhor desempenho. Consequentemente o conjunto completo de testes de regressão para várias combinações de variáveis foi efetuado utilizando-se a floresta aleatória.

Concluiu-se que as variáveis avaliadas tem importância similar na estimação do S4 e que portanto podem ser consideradas correlacionadas com esse índice, conforme demonstrado nesse estudo de caso. Consequentemente, tais variáveis tem uso potencial numa proposta futura de predição de ocorrência de cintilação.

À continuação deste trabalho, pretende-se estender os testes aumentado-se o conjunto de amostras para incluir noites sem ocorrência de cintilação, refinar os parâmetros relativos ás variáveis derivadas do VTEC para otimizar a regressão, e repetir todo o estudo para um período de atividade magnética perturbado. Espera-se que o refinamento desses parâmetros possa melhorar muito o desempenho da regressão, uma vez que seria novamente guiada pelo conhecimento de especialistas da área. Em particular, pretende-se identificar automaticamente, a partir de dados do VTEC, a ocorrência do pico da pré reversão. Outras possibilidades são relativas ao uso de algoritmos ainda não testados, dada a diversidade de algoritmos de aprendizado de máquina disponíveis no ambiente Python. Em relação ao rastreamento espaço-temporal das bolhas ionosféricas, pretende-se obter mais dados e possivelmente utilizar a técnica de ``kriging", que permitiria estimar, no caso, o campo de VTEC sobre parte do território brasileiro, a partir das medidas de um conjunto de estações utilizando interpolação.



%% insira quantos capítulos desejar com o seguinte comando:
%\include{_pasta_do_arquivo_/_meu_arquivo_} %%sem a extensão
%% note que deverá haver um arquivo _meu_arquivo_.tex (com extensão) no diretório _pasta_do_arquivo_

%\include{./docs/conclusao}

%% Bibliografia %% não alterar %% obrigatório %testebib
\bibliography{./bib/referencia} %% aponte para seu arquivo de bibliografia no formato bibtex (p.ex: referencia.bib)


%\include{./docs/glossario} %% insira os termos do glossário no arquivo glossario.tex %% opcional

\inicioApendice %% opcional, comente esta linha e a seguintes se não houver apendice(s)
%%%%%%%%%%%%%%%%%%%%%%%%%%%%%%%%%%%%%%%%%%%%%%%%%%%%%%%
%Apêndice A
\hypertarget{estilo:apendice1}{} %% uso para este Guia
%Este apêndice foi criado apenas para indicar como construir um apêndice no estilo, não existia no original da tese.
%%%%%%%%%%%%%%%%%%%%%%%%%%%%%%%%%%%%%%%%%%%%%%%%%%%%%%
\renewcommand{\thechapter}{}%
\chapter{APÊNDICE A - AUTORIZAÇÃO PARA PUBLICAÇÃO}	% trocar A por B na próxima apêndice e etc
\label{apendiceA}	% trocar A por B na próxima apêndice e etc
\renewcommand{\thechapter}{A}%		% trocar A por B na próxima apêndice e etc

Há dois formulários de autorização para publicação, um para publicações de trabalhos acadêmicos e outro para publicações técnico-científicas, neste apêndice encontram-se os modelos dos formulários e suas respectivas instruções de preenchimento. 

\section{Autorização para Publicação de Trabalho Acadêmico - INPE-393}

\label{instr393}

	\begin{figure}[ht]
		\caption{Formulário Autorização para Publicação de Trabalho Acadêmico INPE-393.}
		\vspace{6mm}	% acrescentar o espaçamento vertical apropriado entre o título e a borda superior da figura
		\centering
   		\includegraphics[height=16cm]{./Figuras/form393.png}	   
 		\label{form393}
	\end{figure}


\subsection{Instruções do Formulário INPE-393} 

\begin{enumerate} 

 \item \textbf{série:} com este número o SID identifica as publicações do INPE, composto da sigla da Instituição, número sequencial geral da publicação, sigla e número sequencial do tipo de publicação, exemplo: INPE-14209-TDI/1110;
 
 \item \textbf{número:} será composto da sigla da unidade do SID, mais 4 (quatro) dígitos e do ano em curso. Este número de referência é de controle da unidade emissora. Ex.: SID-0001/2007;

 \item \textbf{título da publicação:} deve ser completo, evitando-se abreviar palavras;

 \item \textbf{nome do autor e do orientador:} estes campos devem ser preenchidos por extenso, da mesma forma em que irão constar da publicação;

 \item \textbf{origem da publicação:} sigla da unidade do servidor (autor da publicação), conforme TQ-001;

 \item \textbf{curso:} sigla do curso, de acordo com a Estrutura de Divisão de Trabalho - EDT do INPE;
 
 \item \textbf{tipo:} assinalar se é tese ou dissertação;

 \item \textbf{apresentação:} colocar a data de aprovação final;

 \item \textbf{revisão técnica:} o responsável designado pela Banca Examinadora para verificação de correções e, na ausência desse, o orientador da tese ou dissertação deve
carimbar, datar e assinar após a versão \emph{on line} do trabalho;

 \item \textbf{revisão de linguagem:} o responsável designado pela Banca Examinadora para verificação de correções, e na ausência deste o orientador deve assinalar a solicitação ou a dispensa da revisão de linguagem e, carimbar, datar e assinar; o revisor deve datar e assinar após a revisão;
 
 \item \textbf{distribuição:} O SID deve informar a quantidade de CD's e de cópias impressas da tese ou dissertação, conforme lista de distribuição;
 
 \item \textbf{verificação de normalização:}  Após a verificação da versão \emph{on line} do trabalho quanto às normas editoriais, o SID deve datar e assinar;
 
 \item \textbf{autorização final:} data e assinatura do Titular de Nível A, conforme TQ-001, a que o Serviço de Pós-Graduação estiver subordinado.
 
 \item \textbf{observações:} para outras informações necessárias. 

\end{enumerate}

\section{Autorização para Publicação - INPE-106}
\begin{figure}[ht!]
	\caption{Formulário Autorização para Publicação de Trabalho Acadêmico INPE-106 folha 1.} 
	\vspace{6mm}	% acrescentar o espaçamento vertical apropriado entre o título e a borda superior da figura
	\centering
	\includegraphics[height=18cm]{./Figuras/form106.png}
	\label{form106}
\end{figure}

\begin{figure}[ht!]
	\caption{Formulário Autorização para Publicação de Trabalho Acadêmico INPE-106 folha 2.} 
	\vspace{6mm}	% acrescentar o espaçamento vertical apropriado entre o título e a borda superior da figura
	\centering
	\includegraphics[height=18cm]{./Figuras/form106folha2.png}
	\label{form106a}
\end{figure}

\clearpage
\subsection{Instruções do Formulário INPE-106} 
\label{instr106}


\begin{enumerate}

 \item \textbf{série:} com este número o SID identifica as publicações do INPE, composto da sigla da Instituição, número sequencial geral da publicação, sigla e número sequencial do tipo de publicação, exemplo: INPE-5616-RPQ/671. 
 
 \item \textbf{número:} será composto da sigla da unidade constante da Estrutura Organizacional do INPE (TQ-001), mais 4 (quatro) dígitos e do ano em curso. Este número de referência é de controle da unidade solicitante. Ex: CEA-0001/2007;
 
 \item \textbf{título da publicação:} deve ser completo, evitando-se abreviar palavras;

 \item \textbf{nome do autor, tradutor e editor:}  estes campos devem ser preenchidos por extenso, da mesma forma em que irão constar da publicação;

 \item \textbf{origem da publicação:} sigla da unidade do servidor (autor da publicação), conforme TQ-001;

 \item \textbf{projeto:} sigla do projeto de acordo com a Estrutura de Divisão de Trabalho - EDT do INPE;

 \item \textbf{tipo de publicação:} assinalar o tipo de publicação proposta:

 \begin{enumerate}
  \item{Relátorio de Pesquisa (RPQ)},
  \item{Notas Técnico-Científicas (NTC)},
  \item{Propostas e Relatórios de de Projeto (PRP)},
  \item{Manuais Técnicos (MAN)},
  \item{Publicações Didáticas (PUD)},
  \item{Trabalhos Acadêmicos Externos (TAE)}.
 \end{enumerate}

 \item \textbf{divulgação:} assinalar, de acordo com os critérios de classificação. Se houver Lista de Divulgação, nesta deverá constar os nomes e endereços completos;

 \item \textbf{convênio:} descrever o nome da instituição, quando a publicação for realizada pelo INPE e outra organização, preencher somente para o tipo PRP; 
 
    \item \textbf{autorização preliminar:} data, carimbo e assinatura do Titular da Unidade a que o autor esteja subordinado e, assinatura do revisor que efetuou a revisão técnica aprovando a versão \emph{on line} do trabalho e do revisor que realizou a revisão de linguagem, quando solicitadas; 
    
  \item \textbf{verificação de normalização:} o SID deve datar e assinar após a revisão da adequação às normas editoriais;   
  
  \item \textbf{distribuição:} O SID deve informar a quantidade de CD's e de cópias impressas que deverão ser gravados conforme lista de distribuição;
  
 \item \textbf{autorização final:} data, carimbo e assinatura do Titular de Nível "A", conforme TQ-001, a que o autor da publicação estiver subordinado;
 
 \item \textbf{observações:} para outras informações necessárias, inclusive para descrever as justificativas de uma publicação.
\end{enumerate} %% insira apendices tal qual capítulos acima


\inicioAnexo
%%%%%%%%%%%%%%%%%%%%%%%%%%%%%%%%%%%%%%%%%%%%%%%%%%%%%%%
%Anexo
%Este anexo foi incluido para explicar como incluir um anexo no estilo, não existia no original desta tese.
%%%%%%%%%%%%%%%%%%%%%%%%%%%%%%%%%%%%%%%%%%%%%%%%%%%%%%%%%%%%%%%%%%%%%%%%%%%%%%%%%
\renewcommand{\thechapter}{}%
\chapter{ANEXO A - ABREVIATURA DOS MESES} %% Título do anexo sempre em maiúsculas. Trocar A por B no próximo anexo e etc
\label{anexoA} %% Rótulo aplicado caso queira referir-se a este tópico em qualquer lugar do texto. Trocar A por B no próximo anexo e etc
\renewcommand{\thechapter}{A}%		% trocar A por B no próximo anexo e etc

\begin{table}[!ht]
 \label{tab:abreviaturas}
  \begin{center}
 	\begin{tabular}{lll}
	 \hline
	  \textbf{Português}    & \textbf{Espanhol}  & \textbf{Italiano}\\ 
   \hline
       janeiro   = jan.   & enero = ene.       & gennaio = gen.\\
       fevereiro = fev.   & febrero = feb.     & febbraio = feb.\\
       março     = mar.   & marzo = mar.       & marzo = mar.\\
       abril     = abr.   & abril = abr.       & aprile = apr.\\
       maio      = maio   & mayo = mayo        & maggio = mag.\\ 
       junho     = jun.   & junio = jun.       & giugno = giu.\\ 
       julho     = jul.   & julio = jul.       & luglio = lug.\\
       agosto    = ago.   & agosto = ago.      & agosto = ago.\\
       setembro  = set.   & septiembre = sep.  & settembre = set.\\
       outubro   = out.   & octubre = oct.     & ottobre = ott.\\
       novembro  = nov.   & noviembre =nov.    & novembre = nov.\\
       dezembro  = dez.   & diciembre = dic.   & dicembre = dic.\\ 
     \hline
   \textbf{Francês}       & \textbf{Inglês}    & \textbf{Alemão}\\
     \hline
       janvier = jan.     & January = Jan.     & Januar = Jan.\\
       février = fév.     & February = Feb.    & Februar = Feb.\\
       mars = mars        & March = Mar.       & März = März\\
       avril = avr.       & April = Apr.       & April = Apr.\\
       mai = mai          & May = May          & Mai = Mai.\\
       juin = juin        & June = June        & Juni = Juni\\
       juillet = juil.    & July = July        & Juli = Juli\\
       août = août        & August = Aug.      & August = Aug.\\
       septembre = sept.  & September = Sept.  & September = Sept.\\
       octobre = oct.     & October = Oct.     & Oktober = Okt.\\
       novembre = nov.    & November = Nov.    & November = Nov.\\
       décembre = déc.    & December = Dec.    & Dezember = Dez. \\ 
    \hline
   \end{tabular}
   \end{center}
	 \FONTE{Adaptada de \citeonline[p.~22]{NBR6023:2002b}.}
\end{table}

\inicioIndice
%%%%%%%%%%%%%%%%%%%%%%%%%%%%%%%%%%%%%%%%%%%%%%%%%%%%%%%
%Contracapa
%%%%%%%%%%%%%%%%%%%%%%%%%%%%%%%%%%%%%%%%%%%%%%%%%%%%%%

\thispagestyle{empty}
 \begin{table}
  \begin{center}
  \begin{tabularx}{\textwidth}{X}
   \textbf{PUBLICAÇÕES TÉCNICO-CIENTÍFICAS EDITADAS PELO INPE}
  \end{tabularx} 
  \end{center}
 \end{table}
  
 \begin{table}
  \begin{center}
  \begin{tabularx}{\textwidth}{X X}
      
  \textbf{Teses e Dissertações (TDI)}              & \textbf{Manuais Técnicos (MAN)}\\
\\
Teses e Dissertações apresentadas nos Cursos de Pós-Graduação do INPE.	&
São publicações de caráter técnico que incluem normas, procedimentos, instruções e orientações.\\
\\
\textbf{Notas Técnico-Científicas (NTC)}           & \textbf{Relatórios de Pesquisa (RPQ)}\\
\\
Incluem resultados preliminares de pesquisa, descrição de equipamentos, descrição e ou documentação de programas de computador, descrição de sistemas e experimentos, apresentação de testes, dados, atlas, e documentação de projetos de engenharia. 
&	
Reportam resultados ou progressos de pesquisas tanto de natureza técnica quanto científica, cujo nível seja compatível com o de uma publicação em periódico nacional ou internacional.\\
\\
\textbf{Propostas e Relatórios de Projetos (PRP)}	& \textbf{Publicações Didáticas (PUD)} 
\\
\\
São propostas de projetos técnico-científicos e relatórios de acompanhamento de projetos, atividades e convênios.
&	
Incluem apostilas, notas de aula e manuais didáticos. \\
\\         
\textbf{Publicações Seriadas} 	& \textbf{Programas de Computador (PDC)}\\
\\
São os seriados técnico-científicos: boletins, periódicos, anuários e anais de eventos (simpósios e congressos). Constam destas publicações o Internacional Standard Serial Number (ISSN), que é um código único e definitivo para identificação de títulos de seriados. 
&	
São a seqüência de instruções ou códigos, expressos em uma linguagem de programação compilada ou interpretada, a ser executada por um computador para alcançar um determinado objetivo. Aceitam-se tanto programas fonte quanto os executáveis.\\
\\
\textbf{Pré-publicações (PRE)} \\
\\
Todos os artigos publicados em  periódicos, anais e como capítulos de livros. \\                 \end{tabularx}
  \end{center}
 \end{table}


\end{document}
