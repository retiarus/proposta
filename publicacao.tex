\documentclass[%
%%% PARA ESCOLHER O ESTILO TIRE O SIMBOLO %(COMENTÁRIO)
%SemVinculoColorido,
%SemFormatacaoCapitulo,
SemFolhaAprovacao,
%SemImagens,
%CitacaoNumerica, %% o padrão é citação tipo autor-data
%PublicacaoDissOuTese, %% (é também o "default") com ficha catal. e folha de aprovação em branco. Caso tenha lista de símbolos e lista de siglas e abreviaturas retirar os comentários dos arquivos siglas.tex e abreviaturasesiglas.tex. Retirar também os comentários indicados nesse arquivo, nos includes
%PublicacaoArtigoOuRelatorio, %% texto sequencial, sem quebra de páginas nem folhas em branco
PublicacaoProposta, %% igual tese/dissertação, mas sem ficha catal. e fol. de aprov.
%PublicacaoLivro, %% com capítulos
%PublicacaoLivro,SemFormatacaoCapitulo, %% sem capítulos
english,portuguese %% para os documentos em Português com abstract.tex em Inglês
%portuguese,english %% para os documentos em Inglês com abstract.tex em Português
,LogoINPE% comentar essa linha para fazer aparecer o logo do Governo
,CCBYNC	% as opções de licença são: CCBY, CCBYSA, CCBYND, CCBYNC, CCBYNCSA, CCBYNCND, GPLv3 e INPECopyright
]{tdiinpe}
%]{../../../../../iconet.com.br/banon/2008/03.25.01.19/doc/tdiinpe}

% PARA EXIBIR EM ARIAL TIRAR O COMENTÁRIO DAS DUAS LINHAS SEGUINTES
%\renewcommand{\rmdefault}{phv} % Arial
%\renewcommand{\sfdefault}{phv} % Arial

% PARA PUBLICAÇÕES EM INGLÊS:
% renomear o arquivo: abnt-alf.bst para abnt-alfportuguese.bst
% renomear o arquivo: abnt-alfenglish.bst para abnt-alf.bst


%%%%%%%%%%%%%%%%%%%%%%%%%%%%%%%%%%%%%%%%%%%%%
%%% Pacotes já previamente carregados:      %
%%%%%%%%%%%%%%%%%%%%%%%%%%%%%%%%%%%%%%%%%%%%%%%%%%%%%%%%%%%%%%%%%%%%%%%%
%%% ifthen,calc,graphicx,color,inputenc,babel,hyphenat,array,setspace, %
%%% bigdelim,multirow,supertabular,tabularx,longtable,lastpage,lscape, %
%%% rotate,caption2,amsmath,amssymb,amsthm,subfigure,tocloft,makeidx,  %
%%% eso-pic,calligra,hyperref,ae,fontenc                               %
%%%%%%%%%%%%%%%%%%%%%%%%%%%%%%%%%%%%%%%%%%%%%%%%%%%%%%%%%%%%%%%%%%%%%%%%
%%% insira neste campo, comandos de LaTeX %%%
%%% \usepackage{_exemplo_}
% etc.
%%%%%%%%%%%%%%%%%%%%%%%%%%%%%%%%%%%%%%%%%%%%%

%\watermark{Revisão No. ##} %% use o comando \watermark para identificar a versão de seu documento
%% comente este comando quando for gerar a versão final
%\usepackage{subcaption}
\usepackage{braket}
\usepackage{adjustbox}
\usepackage{rotating}
\usepackage{hyperref}
\usepackage{float}
\usepackage{svg}
\usepackage{epstopdf}
\usepackage{rotating}
\usepackage{dsfont}
\usepackage{comment}
\usepackage{algorithm}
\usepackage[noend]{algpseudocode}
\usepackage{tikz}
\usetikzlibrary{shapes.geometric, arrows}
%%%%%%%%%%%%%%%%%%%CAPA%%%%%%%%%%%%%%%%%%%%%%%%%%%%%%%%
%\serieinpe{INPE-NNNNN-TDI/NNNN} %% não mais usado

\titulo{XXXXXXXXXXX}
\title{Escrever o título em Inglês para publicações escritas em Português e em Português para publicações escritas em Inglês} %%
\author{Pedro Alexandre dos Santos} %% coloque o nome do(s) autor(es)
\descriccao{Monografia para Exame de Proposta do Curso de Pós-Graduação em Computação Aplicada, orientada pelo Dr. Stephan Stephany, aprovada em xxxxxx de agosto de 2019.}
\repositorio{aa/bb/cc/dd} %% repositório onde está depositado este documento - na omissão, será preenchido pelo SID
\tipoDaPublicacao{TDI}	%% tipo da publicação (NTC, RPQ, PRP, MAN, PUD, TDI, TAE e PRE) na ausência do número de série INPE, caso contrário deixar vazio
\IBI{xx/yy} %% IBI (exemplo: J8LNKAN8PW/36CT2G2) quando existir, caso contrário o nome do repositório onde está depositado o documento

\date{2019}%ano da publicação

%%%%%%%%%%%%%%%%%%%%%%%%%%VERSO DA CAPA%%%%%%%%%%%%%%%%%%%%%%%%%%%%%%%%%%%%%%%%%%%%%%%
\tituloverso{\vspace{-0.9cm}\textbf{\PublicadoPor:}}
\descriccaoverso{Instituto Nacional de Pesquisas Espaciais - INPE\\
Gabinete do Diretor (GB)\\
Serviço de Informação e Documentação (SID)\\
Caixa Postal 515 - CEP 12.245-970\\
São José dos Campos - SP - Brasil\\
Tel.:(012) 3945-6923/6921\\
Fax: (012) 3945-6919\\
E-mail: {\url{pubtc@sid.inpe.br}}
}

\descriccaoversoA{\textbf{\ConselhoDeEditoracao:}\\
\textbf{\Presidente:}\\
Marciana Leite Ribeiro - Serviço de Informação e Documentação (SID)\\
\textbf{\Membros:}\\
Dr. Gerald Jean Francis Banon - Coordenação Observação da Terra (OBT)\\
Dr. Amauri Silva Montes - Coordenação Engenharia e Tecnologia Espaciais (ETE)\\
Dr. André de Castro Milone - Coordenação Ciências Espaciais e Atmosféricas (CEA)\\
Dr. Joaquim José Barroso de Castro -  Centro de Tecnologias Espaciais (CTE)\\
Dr. Manoel Alonso Gan - Centro de Previsão de Tempo e Estudos Climáticos (CPT)\\
Drª Maria do Carmo de Andrade Nono - Conselho de Pós-Graduação\\
Dr. Plínio Carlos Alvalá - Centro de Ciência do Sistema Terrestre (CST)\\
\textbf{\BibliotecaDigital:}\\
Dr. Gerald Jean Francis Banon - Coordenação de Observação da Terra (OBT)\\
Clayton Martins Pereira - Serviço de Informação e Documentação (SID)\\
%Jefferson Andrade Ancelmo - Serviço de Informação e Documentação (SID)\\
%Simone A. Del-Ducca Barbedo - Serviço de Informação e Documentação (SID)\\
%Deicy Farabello - Centro de Previsão de Tempo  e Estudos Climáticos (CPT)\\
\textbf{\RevisaoNormalizacaoDocumentaria:}\\
Simone Angélica Del Ducca Barbedo - Serviço de Informação e Documentação (SID) \\
%Marilúcia Santos Melo Cid - Serviço de Informação e Documentação (SID)\\
Yolanda Ribeiro da Silva Souza - Serviço de Informação e Documentação (SID)\\
\textbf{\EditoracaoEletronica:}\\
Marcelo de Castro Pazos - Serviço de Informação e Documentação (SID)\\
André Luis Dias Fernandes - Serviço de Informação e Documentação (SID)\\
}

%%%%%%%%%%%%%%%%%%%FOLHA DE ROSTO

%%%%%%%%%%%%%%%FICHA CATALOGRÁFICA
% NÃO PREENCHER - SERÁ PREENCHIDO PELO SID

\cutterFICHAC{Cutter}
\autorUltimoNomeFICHAC{Sobrenome, Nomes} %% exemplo: Fuckner, Marcus André
\autorFICHAC {Nome Completo do Autor1; Nome Completo do Autor2} %% Campo opcional (se não usado prevalece \author)
\tituloFICHAC{Titulo da publicação}
\instituicaosigla{INPE}
\instituicaocidade{São José dos Campos}
\paginasFICHAC{\pageref{numeroDePáginasDoPretexto} + \pageref{LastPage}} %% número total de páginas
%\serieinpe{INPE-00000-TDI/0000} %% não mais usado
\palavraschaveFICHAC{1.~Palavra chave. 2.~Palavra chave 3.~Palavra chave. 4.~Palavra chave. 5.~Palavra chave  I.~\mbox{Título}.} %% recomenda-se pelo menos 5 palavras-chaves - \mbox{} é para evitar hifenização
\numeroCDUFICHAC{000.000} %% número do CDU

% Nota da ficha (para TD)
\tipoTD{Dissertação ou Tese} % Dissertação ou Tese
\cursoFA{Mestrado ou Doutorado em Nome do Curso}
\instituicaoDefesa{Instituto Nacional de Pesquisas Espaciais}
\anoDefesa{AAAA} % ano de defesa
\nomeAtributoOrientadorFICHAC{Orientador}	% pode ser: Orientador, Orientadora ou Orientadores
\valorAtributoOrientadorFICHAC{José da Silva} % nome(s) completo(s)

%%%%%%%%%%%%%%%FOLHA DE APROVAÇAO PELA BANCA EXAMINADORA
\tituloFA{\textbf{ATENÇÃO! A FOLHA DE APROVAÇÃO SERÁ INCLUIDA POSTERIORMENTE.}}
%\cursoFA{\textbf{}}
\candidatoOUcandidataFA{}
\dataAprovacaoFA{}
\membroA{}{}{}
\membroB{}{}{}
\membroC{}{}{}
\membroD{}{}{}
\membroE{}{}{}
\membroF{}{}{}
\membroG{}{}{}
\ifpdf

%%%%%%%%%%%%%%NÍVEL DE COMPRESSÃO {0 -- 9}
\pdfcompresslevel 9
\fi
%%% define em 80% a largura das figuras %%%
\newlength{\mylenfig}
\setlength{\mylenfig}{0.8\textwidth}
%%%%%%%%%%%%%%%%%%%%%%%%%%%%%%%%%%%%%%%%%%%

%%%%%%%%%%%%%%COMANDOS PESSOAIS
\newcommand{\vetor}[1]{\mathit{\mathbf{#1}}}
 %% faça as modificações pertinentes no arquivo configuracao.tex

\makeindex  %% não alterar, gera INDEX, caso haja algum termo indexado no texto

\begin{document} %% início do documento %% não mexer

%\marcaRegistrada{}	% comando opcional usado para informar abaixo da ficha catalográfica sobre marca registrada
\marcaRegistrada{Informar aqui sobre marca registrada (a modificação desta linha deve ser feita no arquivo publicacao.tex).}

\maketitle  %% não alterar, gera páginas obrigatórias (folha de rosto, ficha catalográfica e folha de aprovação), automaticamente

%%% Comente as linhas opcionais abaixo caso não as deseje
%%%%%%%%%%%%%%%%%%%%%%%%%%%%%%%%%%%%%%%%%%%%%%%%%%%%%%%%%%%%%%%%%%%%%%%%%%%%%%%%%
% Epígrafe %% opcional

\begin{epigrafe} %% insira sua epígrafe abaixo; estilo livre

\hypertarget{estilo:epigrafe}{} %% uso para este Guia
 
\textit{\large``A vida será mais complicada se você possuir uma curiosidade ativa, além de aumentarem as chances de você entrar em apuros, mas será mais divertida''.}

\vspace{1cm}

\hspace{4cm} \emph{\textsc{Edward Speyer}}\\\hspace{4cm} em \textsl{``Seis Caminhos a Partir de Newton''}, 1994

\end{epigrafe}
 %% Opcional
%%%%%%%%%%%%%%%%%%%%%%%%%%%%%%%%%%%%%%%%%%%%%%%%%%%%%%%%%%%%%%%%%%%%%%%%%%%%%%%%%
% Dedicatória %% opcional

\begin{dedicatoria} %% insira sua dedicatória abaixo; estilo livre

\hypertarget{estilo:dedicatoria}{} %% uso para este Guia
 
%% use 'a meus' em vez de 'aos meus', isto é, não use o artigo definido com pronomes possessivos

\newcommand{\mytext}{A meus pais \textbf{Nicanor} e \textbf{Jaci}, à minha irmã \textbf{Luciana} e ao meu esposo \textbf{William}}

\begin{comment}
%%% sugestão de estilo
\ifcalligra %% fonte calligra presente nas versões mais novas do MiKTeX (>= 2.4)
  \calligra\Large \mytext %% exemplo usando estilo de fonte caligráfica, caso haja
\else
	\itshape\Large \mytext 
\fi
\end{comment}

	\itshape\Large \mytext 

\end{dedicatoria} %% Opcional
%%%%%%%%%%%%%%%%%%%%%%%%%%%%%%%%%%%%%%%%%%%%%%%%%%%%%%%%%%%%%%%%%%%%%%%%%%%%%%%%%
% AGRADECIMENTOS %% opcional

\begin{agradecimentos}  %% insira abaixo seus agradecimentos

\hypertarget{estilo:agradecimentos}{} %% uso para este Guia
Agradecemos à MsC Andriana Susana Lopes de Oliveira Campanharo que gentilmente cedeu 
parte dos textos de sua dissertação para este estilo. O original de sua dissertação
encontra-se na Biblioteca Digital do INPE, no endereço \url {http://urlib.net/sid.inpe.br/MTC-m13@80/2006/11.07.12.37}.
Agradecemos também ao Dr. Gerald Jean Francis Banon pelo desenvolvimento e disponibilização deste estilo.
\end{agradecimentos}


 %% Opcional
%%%%%%%%%%%%%%%%%%%%%%%%%%%%%%%%%%%%%%%%%%%%%%%%%%%%%%%%%%%%%%%%%%%%%%%%%%%%%%%%
% RESUMO %% obrigatório

\begin{resumo}

%% neste arquivo resumo.tex
%% o texto do resumo e as palavras-chave têm que ser em Português para os documentos escritos em Português
%% o texto do resumo e as palavras-chave têm que ser em Inglês para os documentos escritos em Inglês
%% os nomes dos comandos \begin{resumo}, \end{resumo}, \palavraschave e \palavrachave não devem ser alterados

\hypertarget{estilo:resumo}{} %% uso para este Guia

%A ionosfera é uma camada da atmosfera que se estende de aproximadamente 60 km a 1000 km de altitude. Esta camada influi nos sinais de radiofrequência transmitidos por satélites para a superfície terrestre, sendo composta por gases ionizados principalmente pela radiação solar e elétrons livres. Variabilidades no fluxo solar geram alterações nos campos elétrico e magnético no espaço e, consequentemente, no campo magnético da Terra causando flutuações na ionização e, portanto, na quantidade de elétrons livres na ionosfera, alterando a transmissão de sinais de radiofrequência. Dentre as várias pertubações ionosféricas, há a anomalia magnética equatorial, que consiste na formação de uma região com alta densidade de elétrons ente 15 e 20 graus magnéticos ao norte e sul do equador. Entretanto, essa anomalia é mais significativa no Brasil, especificamente no Vale do Paraíba, estado de São Paulo. Os sinais de radiofrequência dos sistemas de navegação por satélites (GNSS - Global Navigation Satellite System) são afetados por flutuações da ionização, que constituem o fenômeno de cintilação ionosférica, que pode ser medidas pelo índices S4. A cintilação afeta sinais GNSS, especialmente no pico da anomalia, afetando a navegação aérea e outras atividades humanas que dependem de sinais recebidos de satélites.  Por outro lado, a quantidade de elétrons livres na ionosfera pode ser medida pelo conteúdo eletrônico total vertical (VTEC), sendo que regiões com baixos valores de VTEC em relação à sua vizinhança, caracterizam as denominadas de bolhas ionosféricas, associadas às cintilações. Dada a existência de redes de estações de medição que provém valores de S4 e VTEC, este trabalho busca correlacionar os valores destas variáveis, bem como analisar sua evolução espaço-temporal por meio de técnicas de mineração e visualização de dados, considerando como estudo de caso a cidade de São José dos Campos.

\palavraschave{%
  \palavrachave{Cintilação Ionosférica}%
%  \palavrachave{Bolha Ionosférica}%
  \palavrachave{Índice S4}%
%  \palavrachave{VTEC}%
  \palavrachave{Mineração de dados}%
  \palavrachave{GNSS}
%  \palavrachave{Vale do Paraíba}
}
 
\end{resumo}
 %% obrigatório
%%%%%%%%%%%%%%%%%%%%%%%%%%%%%%%%%%%%%%%%%%%%%%%%%%%%%%%%%%%%%%%%%%%%%%%%%%%%%%%%
% ABSTRACT


\begin{abstract}

%% neste arquivo abstract.tex
%% o texto do resumo e as palavras-chave têm que ser em Inglês para os documentos escritos em Português
%% o texto do resumo e as palavras-chave têm que ser em Português para os documentos escritos em Inglês
%% os nomes dos comandos \begin{abstract}, \end{abstract}, \keywords e \palavrachave não devem ser alterados

\selectlanguage{english}	%% para os documentos escritos em Português
%\selectlanguage{portuguese}	%% para os documentos escritos em Inglês

\hypertarget{estilo:abstract}{} %% uso para este Guia

The ionosphere is a layer of gas in the state of plasma that was ionized mainly by the effect of solar radiation. Ionospheric plasma distribution is not uniform in space and time, being  the generation of plasma irregularities triggered by the day-to-night transition. One of them is que Equatorial Ionization Anomaly, which coupled  with the plasma instability mechanism cause depletions, i.e. regions with low density of ions and electrons. Such structures are known as ionospheric bubbles and are generated at the magnetic equator just after sunset. They ascent to higher altitudes and migrate to low latitudes along the Earth magnetic field. RF signals are affected by the ionospheric bubbles. Ionospheric scintillation is the occurrence of perturbation of RF signals due to ionospheric irregularities, causing signal phase and amplitude fluctuations. This proposal addresses the use of knowledge discovery in databases in order to predict ionospheric scintillation in the Brazilian territory, in particular, in São José dos Campos. It is intended to employ historical data of ionospheric scintillation and other data including solar activity level, plasma vertical drift velocity and global magnetic activity. The algorithm used for the prediction, either formulated as a classification or a regression problem, is the Extreme Gradient Boosting (XGBoost), available in the Python programming environment, and preliminary results are presented here.

\keywords{%
	\palavrachave{Ionospheric Scintillation}%
	\palavrachave{S4 Index}%
	\palavrachave{Plasma Bubble}%
%	\palavrachave{VTEC}%
	\palavrachave{Data Mining}%
	\palavrachave{GNSS}
%	\palavrachave{Vale do Paraíba}
}

\selectlanguage{portuguese}	%% para os documentos escritos em Português
%\selectlanguage{english}	%% para os documentos escritos em Inglês

\end{abstract}
 %% obrigatório

\includeListaFiguras %% obrigatório caso haja mais de 3 figuras, gerado automaticamente
\includeListaTabelas %% obrigatório caso haja mais de 3 tabelas, gerado automaticamente

%%%%%%%%%%%%%%%%%%%%%%%%%%%%%%%%%%%%%%%%%%%%%%%%%%%%%%%%%%%%%%%%%%%%%%%%%%%%%%%%
%abreviaturas e siglas  %% opcional, mas recomendado

\begin{abreviaturasesiglas}  %% insira abaixo suas abreviaturas conforme o modelo.

%% sigla (separador: &--&) significado (quebra de linha: \\)
\\
GPS   &--& Sistema de Posicionamento Global\\
OACI   &--& Organização da Aviação Civil Internacional\\
GNSS &--& Sistema de Navegação Global por Satélite\\
EIA &--& Anomalia da Ionização Equatorial\\
VTEC &--& Conteúdo eletrônico total vertical\\
AACGM &--& Altitude Adjusted Corrected Geomagnetic Coordinates\\
KDD &--& Descoberta de Conhecimento em Base de dados\\
CART &--& Classification and Regression Trees\\
SVM &--& Support Vector Machine\\
\end{abreviaturasesiglas}
 %% opcional %% altere o arquivo siglaseabreviaturas.tex

%%%%%%%%%%%%%%%%%%%%%%%%%%%%%%%%%%%%%%%%%%%%%%%%%%%%%%%%%%%%%%%%%%%%%%%%%%%%%%%%%
% simbolos

\begin{simbolos}

%% o comando: \hypertarget{estilo:simbolos}{} abaixo é de uso para este Guia
%% e pode ser retirado

\hypertarget{estilo:simbolos}{}
\\
a   &--& primeira contante \\
b   &--& segunda constante \\
$\rho$  &--& densidade de um fluido\\
$\nu$   &--& viscosidade cinemática\\
$R_{e}$  &--& número de Reynolds\\
$\alpha$  &--& constante de Kolmogorov\\
$k$ &--&  número de onda\\
$K$ &--&  curtose\\
$D_{0}$ &--& dimensão de contagem de caixas\\
$D_{1}$ &--& dimensão de informação\\
$D_{2}$  &--& dimensão de correlação\\
$\lambda_{1}$  &--& expoente de Lyapunov dominante\\
 

\end{simbolos}

 %% opcional %% altere o arquivo simbolos.tex

\includeSumario  %% obrigatório, gerado automaticamente

\inicioIntroducao %% não altere este comando

\chapter{INTRODUÇÃO}

\chapter{IONOSFERA}

A ionosfera é uma região ionizada da alta atmosfera, estendendo-se de 60 até 1000 km de altitude, assim, englobando partes da mesosfera, termosfera, e exosfera. Esta camada constitui-se de íons e elétrons livres criados primariamente por processo de fotoionização, e gás neutro. A fotoionização ionosférica consiste de um processo físico-químico, onde algumas espécies químicas presentes na atmosfera ganham ou perdem elétrons decorrentes da absorção de radiação solar predominantemente nas faixas ultravioleta, extremo ultravioleta e raios-X \cite{RISNBETH:1969, NEGRETI:2012}. A ionização, também, pode ocorrer devido a colisões com partículas altamente energéticas, provindas do meio solar ou galácticas, o que é mais facilmente observado em altas latitudes em fenômenos como auroral boreal.

A composição da ionosfera, assim, como a densidade do gases variam em função da altitude. A densidade de elétrons livres também varia, pois conforme a radiação penetra na atmosfera mais densa, a produção de elétrons aumenta até atingir um valor de pico em uma dada altitude. Abaixo desta, mesmo havendo um aumento na densidade da atmosfera neutra, a produção de elétrons decresce, pois a maior parte da radiação ionizante foi absorvida ao longo do percurso, e a taxa de recombinação predomina sobre a taxa de produção de elétrons. Devido as diferenças marcantes em termos de processos físicos e químicos que governam o comportamento da ionosfera, a mesma, pode ser dividida em camadas, onde cada uma apresenta um processo predominante. Finalmente, devido as drásticas mudanças em quantidade de radiação absorvida devido a transição entre noite e dia, existirão camadas que aparecem em um dado periódo.

A camada D é a mais interna, estando entre 60 e 90 km acima da superfície da Terra. Apresenta móleculas ionizadas de $NO$, $N_2$ e $O_2$. E, tem a maior razão de recombinação. Apresenta uma taxa de absorção considerável para ondas de radio de baixas e médias frequências, principalmente, devido a absorção de energia pelo elétrons livres, o que aumenta suas chances de colisão. Este efeito desaparece durante à noite, devido a uma menor ionização. Pode apresentar valores elevados de ionização em altas latitudes em decorrência de erupções solares com grandes quantidades de matéria hadrônica, prótons, em sua maioria, com uma duração de 24 à 48 horas.

A camada E é intermediária e está situada entre 90 e 150 km acima da superfície da Terra. A ionização decorre principalmente devido ao espalhamento de raio-X leve e ultravioleta distante (UV) provindos do Sol em moléculas de oxigênio. A estrutura vertical da camada E é determinada em sua maior parte pela competição entre efeitos de ionização e de recombinação. É importante pela presença de correntes elétricas que nela fluem e interagem com o campo magnético \cite{KIRCHHOFF:1991}. A noite, a camada E quase desaparece, pois sua fonte primaria de ionização não está presente.

A camada F se estende de 150 a mais de 500 km acima da superfície da Terra. Apresenta a maior concentração de elétrons, portanto, sinais que são capazes de penetrar até esta subcamada são capazes de escapar para o espaço. Predominam, nesta, a ionização de átomos de oxigênio por meio de radiação solar no espectro do extremo ultravioleta. A camada é subdivida em duas regiões, a F2 que está presente durante o dia e a noite, e a F1 que aparece somente durante o dia. 

A subcamada F2 engloba toda a região superior da ionosfera, inclusive a região de pico da densidade de elétrons. Este máximo no perfil vertical de ionização decorre do balanço entre os processos de transporte de plasma e os processos físico-químicos. Acima deste pico, a ionosfera se encontra em equilíbrio difusivo, ou seja, o plasma se distribui com a sua própria escala de altura. A presença do campo magnético contribui para a distribuição da ionização.

\section{Regiões e Índices}

\subsection{Termosfera}

A termosfera se inicia aproximadamente a 90 km de altitude e não apresenta um limite superior bem definido, mas este deve estar entre 500 e 600 km de altitude. Nesta surgem os fenômenos de aurora, e nela também estão localizadas as órbitas da estação espacial, do ônibus espacial, assim como de vários satélites. Sua sua extensão, como as demais camadas da atmosfera, varia conforme a latitude. 

Temperatura é uma variável termodinâmica calculada como a média da energia cinética das moléculas, ou átomos, que compõem o sistema físico em questão. A temperatura na termosfera pode alcançar valores superiores a 2000 graus Celsius, entretanto, esta não pode ser medida por instrumentos convencionais, como por exemplo um termômetro, pois esta região é muito rarefeita, o que dificulta a transferência de energia térmica para o instrumento e, assim, sua quantificação. Assim, a temperatura é determinada por dados de satélites que alimentam expressões matemáticas baseadas em sua definição. Em geral, os altos valores se devem a grande taxa de absorção de radiação solar pelo nitrogênio e o oxigênio.

Assim como as demais camadas da atmosfera, também, sofre de processos convectivos que combinado com efeitos de gravidade e diferenças de pressão levam a formação de correntes, marés, análoga a dinâmica de outros fluidos como o oceano.

A ionosfera esta majoritariamente situada na termosfera, logo sua dinâmica esta diretamente acoplada a esta.

\subsection{Magnetosfera}

A magnetosfera é uma região delimitada pelo campo geomagnético. Pode ser tratada como um envoltório, recobrindo a Terra, e constitui a parte exterior da ionosfera, onde neste, a eletrodinâmica dos processos do plasma é dada pelo acoplamento com o campo magnético.

Por ser uma região mais externa tem interação direta com os ventos solares, o que leva  a uma distribuição não regular, pois a região voltada diretamente ao Sol sofre um processo de compressão, devido a pressão de radiação, enquanto no lado oposto existe a formação de uma estrutura análoga a um arrasto.

Em consequência de sua importante contribuição para a dinâmica ionosférica é interessante tratar e avaliar variáveis que forneçam informação sobre seu estado, como por exemplo, o Dst (Disturbance Storm Time) e o Kp (índice planetário).

\subsection{Índices Magnéticos}

O {\bf índice Dst} é uma medida da atividade geomagnética, geralmente utilizada para quantificar a intensidade de uma tempestade magnética. Apresenta resolução temporal de uma hora, e sua unidade de medida é o nanotesla (nT). É baseada no valor médio da componente horizontal do campo magnético, em quatro observatórios próximos ao equador magnético.

Em termos do índice Dst, pode-se considerar valores inferiores -30 nT como situações magnéticamente perturbadas. Os valores podem ser mapeados frente ao nível de pertubação da seguinte maneira, \cite{GONZALEZ:1994}: entre -30 nT e -50 nT tempestade geomagnética fraca; entre -50 nT -100 nT tempestade moderada; entre -100 nT e -250 nT tempestades muito intensas; abaixo de -250 nT supertempestades.

O {\bf índice Sym-H} descreve distúrbios no campo geomagnético em médias latitudes em termos de pertubações simétricas horizontais as linhas de campo magnético. Essencialmente é o mesmo que o índice DST, entretanto, apresenta uma resolução temporal de um minuto. Sua medida é realizada por um conjunto de seis estações em um sistema de coordenadas ligeiramente diferente. As estações que medem o Dst são diferentes daquelas que medem o Sym-H.

No trabalho \cite{WANLISS:2006} foi mostrado que os índices DST e Sym-H são equivalentes, em diversas configurações geomagnéticas, confirmando que o Sym-H poderia ser utilizado como uma alternativa de alta resolução temporal ao Dst.

O {\bf índice Kp} avalia as pertubações nas componentes horizontais do campo geomagnético global, e é baseado em medidas realizadas a cada 3 horas por magnetrômetros distribuídos ao redor do globo. Cada estação é calibrada segundo sua localização e reporta uma quantidade denominada de índice K dependendo da atividade geomagnética medida no local. Este é uma medida quase-logaritmica com resolução temporal de três horas de atividade geomagnética local com base em comparação com dias calmos. Cada estação mede o desvio máximo da componente horizontal. O índice Kp é gerado por uma combinação dos índices K e seu valor varia de 0 a 9 em intervalos discretos. 0 indica baixa atividade, enquanto 9 indica tempestades extremas, finalmente, fica associado os valores de 0 à 4 a períodos calmos, enquanto valores acima de 5 indicam tempestades magnéticas.

O {\bf índice ap}, que será utilizada neste trabalho, é semelhante ao índice Kp, e corresponde a uma transformação desse para uma escala linear. Varia entre 0 e 400 e sua relação com o Kp é dado na Tabela \ref{tab:kptoap}.

\begin{table}[hhh]
\begin{tabular}{|c|c|c|c|c|c|c|c|c|c|c|c|c|c|c|} \hline
Kp & 0  & 0+ & 1- & 1  & 1+ & 2- & 2   & 2+  & 3-  & 3   & 3+  & 4-  & 4   & 4+  \\ \hline
ap & 0  & 2  & 3  & 4  & 5  & 6  & 7   & 9   & 12  & 15  & 18  & 22  & 27  & 32  \\ \hline
Kp & 5- & 5  & 5+ & 6- & 6  & 6+ & 7-  & 7   & 7+  & 8-  & 8   & 8+  & 9-  & 9   \\ \hline
ap & 39 & 48 & 56 & 67 & 80 & 94 & 111 & 132 & 154 & 179 & 207 & 236 & 300 & 400 \\ \hline
\end{tabular}
\caption{Tabela de conversão entre os valores de Kp e ap. Fonte: https://www.ngdc.noaa.gov/stp/GEOMAG/kp\_ap.html.}
\label{tab:kptoap}
\end{table}

\section{Acoplamentos no Sistema Magnetosfera-Ionosfera}

É conhecido da eletrodinâmica clássica que em materiais condutores campos eletroestáticos devem ser nulos. E, a presença de campo elétrico só é permitido sob a ação de forças dinâmicas. Agora, um material condutor pode ser modelado como uma nuvem de gás de elétrons, onde apenas os portadores de cargas negativas podem se movimentar.

Essa aproximação de condutor pode ser usada como um ponto de partida para descrever o comportamento de um plasma sob campos elétricos. Assim, como o condutor, o plasma também é parcialmente uma nuvem de elétrons, entretanto, também possui portadores de cargas positivos e elementos neutros capazes de se mover. Em sua configuração de equilíbrio, isto é, na ausência de forças externas o campo elétrico médio deve ser nulo, assim, uma aproximação de campo ${\bf E_p}=0$ no referencial das partículas é adequado. Esta é chamada de aproximação magneto-hidrodinâmica (MHD), \cite{ROEDERER:1979} 

A teoria da relatividade restrita em uma aproximação de baixas velocidades, fornece a maneira pela qual campos elétricos e magnéticos se transformam sobre mudança de referencial, em um processo, análogo a composição de velocidades da física newtoniana.

Assim, no plano de referência das partículas a velocidade do plasma é nula, pois a partícula se move juntamente com o referencial. Agora, considere um sistema de referência fixo em relação a Terra e os seguintes campos vetoriais definidos neste: ${\bf v_p}$ um vetor de velocidade associada ao plasma, ${\bf B}$ um vetor de campo magnético e ${\bf E}$ um vetor de campo elétrico, fica valido a expressão

\begin{equation}
{\bf E_p}= {\bf E}+{\bf v_p}\times{\bf B}\mbox{.}
\end{equation}

Assim, um campo elétrico fica relacionado à velocidade do plasma e ao campo magnético por

\begin{equation}\label{eq:evb}
{\bf E}=-{\bf v_p}\times{\bf B}\mbox{.}
\end{equation}

A expressão \eqref{eq:evb} permite concluir a seguinte informação: um campo elétrico ${\bf E}$ perpendicular a ${\bf v_p}$ e ${\bf B}$ é gerado por partículas se movendo em um campo magnético. Finalmente, a componente de velocidade perpendicular a ${\bf B}$ se relaciona com ${\bf B}$ e ${\bf E}$ por

\begin{equation}\label{eq:veb}
{\bf v_d}=\frac{{\bf E}\times{\bf B}}{|{\bf B}|^2}\mbox{,}~
\end{equation}

onde ${\bf v_d}$ é a velocidade de deriva do plasma. As equações \eqref{eq:evb} e \eqref{eq:veb} expressam o mesmo fato, campo elétrico e velocidade estão conectados e um pode gerar o outro.

Alguns dos principais resultados destas relações, em aeronomia, é que estas servem como pedra angular para o desenvolvimento da teoria do dínamo ionosférico. Variações no campo geomagnético já eram conhecidas a mais de 300 anos, e a primeira causa proposta foi a presença de correntes elétricas fluindo na alta atmosfera. A necessidade da presença de portadores de carga para esta corrente levou a concepção do processo de ionização, e logo à existência da ionosfera.

Somente por volta de 1940 com a introdução da teoria do dínamo por Chapman E Bartels ficou bem elucidado o processo de formação das correntes na ionosfera. Segundo este modelo, ventos de marés produzem o movimento necessário das partículas carregadas. Em configurações fracamente excitadas, estes ventos podem ser divididos em duas componentes, uma com origem em variações solares, e a outra nas variações lunares. Isto permitiu pela a primeira vez o cálculo dos valores das correntes e suas magnitudes.

Os ventos de marés devido ao Sol são mais intensos, e são causados em sua maior parte por aquecimento, assim a contribuição gravitacional é menos intensa. Como resultado do movimento de marés e da viscosidade, os elétrons e os íons são arrastados através das linhas de campo magnético gerando campos elétricos, este fenômeno é o efeito dínamo.

Os campos elétricos do dínamo da região E tem sua origem nos ventos de marés associados a radiação solar ultravioleta absorvida na camada de ozônio, ao vapor de água na atmosfera, e efeitos gravitacionais gerados pela Lua. Em contra partida, o dínamo da região F tem sua origem nos ventos de marés térmicos que decorrem da absorção do radiação ultravioleta no extremo do espectro pela termosfera, \cite{ABDU:2005}.

A magnitude dos campos gerados pelo efeito dínamo esta diretamente acoplado a densidade de portadores de carga e, assim, as variações de iluminação ao longo do dia, variações do ciclo solar, com a longitude e latitude e com as estações do ano, por exemplo, \cite{FEJER:1999}.

\subsection{Dínamo da Região E} 

O dínamo da região E haje principalmente no lado diurno da Terra. O mecanismo de geração de correntes pode ser descrito de maneira simplista a partir da seguinte construção: seja ${\bf v}_m$ um campo vetorial de velocidade representativo do vento de maré, e ${\bf B}$ o campo geomagnético. O vento de maré desloca as partículas do plasma ao longo do campo magnético, aqui são observados três fenômenos: colisão entres portadores de cargas positivos, negativos e moléculas neutras, o processo de colisão transfere momento para as partículas carregadas o que gera uma corrente na direção leste; surgimento de movimentos em hélice devido ao acoplamento com o campo magnético; a separação de portadores com diferentes cargas e massas, o processo de transferência de momento por meio de colisão é dependente da massa, a separação entre as partículas gera um campo elétrico.

Assim, na região E os elétrons apresentam velocidades baixas, enquanto os íons serão transportados pelo vento. Logo, a separação de cargas induz um campo elétrico ${\bf E}$ dado por 

\begin{equation}
{\bf E_m} = {\bf v}_m\times{\bf B}\mbox{,}~
\end{equation}

o qual por sua vez induz uma corrente elétrica

\begin{equation}
{\bf J_m}=\sigma{\bf E_m} = \sigma{\bf v}_m\times{\bf B}\mbox{,}~
\end{equation}

onde $\sigma$ é a condutividade. O processo de separação de cargas deve também levar ao desenvolvimento de não homogeneidade na distribuição, isto é, acumulo de cargas, assim, a ionosfera sofre polarização. Finalmente, um campo elétrico de polarização é estabelecido e sua contribuição pode ser modelada em termos de um potencial elétrico, portanto, o campo elétrico total deve ser aproximadamente

\begin{equation}
{\bf E}_d = {\bf v}_m\times{\bf B}-{\bf \nabla}\phi\mbox{,}~
\end{equation}

onde $\phi$ é o potencial elétrico e ${\bf \nabla}\phi$ o campo elétrico devido a polarização. O sistema de correntes resultantes é dado por

\begin{equation}
{\bf J_d}=\tilde{\sigma}[{\bf v}_m\times{\bf B}-{\bf \nabla}\phi]\mbox{,}~
\end{equation}

onde $\tilde{\sigma}$ é o tensor de condutividade. A corrente associada ao dínamo da região E produz um movimento de deriva vertical (para cima) e zonal (para oeste) no período diurno.

\subsection{Dínamo da Região F}

Os ventos termosféricos na região F são horizontais e para oeste durante o dia. Isto provoca um movimento das partículas carregadas ao longo das linhas de campo magnético, proporcional a direção do vento na direção do campo. Existe também um movimento secundário com direção perpendicular tanto ao campo magnético quanto ao vento \cite{BATISTA:1986} dado por

\begin{equation}
{\bf v}=\frac{\nu\omega}{\nu^2+\omega^2}\frac{{\bf v_n}\times{\bf B}}{|{\bf B}|}\mbox{,}~
\end{equation}

onde ${\bf v}$ é a velocidade das partículas carregadas, ${\bf v_n}$ é a velocidade do vento neutro, ${\bf B}$ é o campo magnético, $\nu$ é a frequência de colisão entre partículas neutras e partículas neutras e $\omega={qB/m}$ é a girofrequência das partículas, sendo $q$ a carga e $m$ a massa.

Note a dependência com a carga na expressão da girofrequência, isto provoca uma distinção nos movimentos dos portadores de cargas positivos e negativos, ou seja, eles irão se separar, formando um campo elétrico de polarização. Durante o dia a região E pode ser modelada como um condutor que acoplado a região F leva a dispersão das cargas e restauração do equilíbrio; entretanto à noite este circuito não se fecha, pois a região E deixa de ser um condutor, e a formação do campo elétrico de polarização é possível.

As correntes do dínamo da região F são menos intensa do que as da região E, porém são de grande importância após o por do Sol. O campo elétrico de polarização da região F é vertical, e é o responsável pela intensificação da deriva vertical para cima ao anoitecer, o que é denominado de pico de pré-reversão.

\section{Anomalia de Ionização Equatorial}

As anomalias de interesse para este trabalho as bolhas de plasmas surgem de maneira acoplada a anomalia de ionização equatorial (AIE) que ocorre aproximadamente entre 15 e 20 graus de latitude magnética, tanto no hemisfério norte, quanto no hemisfério sul, na camada F2. A AIE consiste na formação de uma região de alta densidade eletrônica, e é uma anomalia, pois a densidade de plasma deveria ser maior em regiões equatoriais, e não em latitudes magnéticas mais altas.

Sua origem decorre da deriva vertical do plasma da camada F na região equatorial: os ventos termosféricos na camada F fazem surgir uma corrente elétrica apontando para leste, assim como um campo elétrico na mesma direção, enquanto o campo magnético aponta para o norte (lembrando que o dipolo magnético que modela o campo geomagnética da terra se encontra invertido, isto é, o norte magnético se situa no sul geográfico e o sul magnético se encontra no norte magnético), considerando então $\vec{E}\times\vec{B}$, tem-se o surgimento de uma força perpendicular ao campo magnético e ao campo elétrico, o que neste caso, aponta para cima, deslocando o plasmas para regiões de mais alta altitudes. Agora, quando em altas altitudes, o plasma for efeito gravitacional e diferença de pressão é trazido de volta à altitudes mais baixas, porém este movimento de descida é mais eficiente ao longo das linhas de campo magnético, levando a um aumento na densidade de plasma em regiões de médias latitudes.

\section{Bolhas de plasma}

Anomalias secundárias surgem durante o processo de formação da AIE quando considerado os demais acoplamentos como atração gravitacional, diferenças de pressões e processos de colisão. Em geral, estas apresentam um certo nível de organização devido às linhas de campo magnético da Terra. Isto ocorre pois partículas ionizadas ou carregas podem ser mover livremente ao longo das linhas magnéticas mas não entre elas.

Dentre as várias possíveis anomalias secundárias uma de particular interesse são as bolhas ionosféricas ou bolhas de plasma que podem ser definidas como regiões de baixa densidade de plasma ionosférico quando comparadas com a sua vizinhança. 

Em uma descrição grosseira pode-se dizer que surgem na região do equador magnético, após uma rápida elevação do plasma, devido a anomalia de ionização equatorial, isto é, o plasma ao acender cria regiões de baixa densidade.

pós sua formação podem evoluir para altas altitudes (centenas de quilômetros), estendendo-se ao longo das linhas de campo magnético (milhares de quilômetros) nas direções norte-sul, alcançado em torno de 20 graus de latitude magnética.


A cintilação ionosférica é uma variação rápida de amplitude e fase em sinais de radio frequência quando estes atravessam irregularidades no plasma ionosférico, como uma bolha de plasma, que é de particular interesse para este trabalho. Neste trabalho se realiza um estudo sobre algumas variáveis relacionadas com o efeito de cintilação ionosférica e com a estrutura de bolha no plasma.

\section{Algumas variáveis importantes para o estudo da cintilação ionosférica}

Existem várias quantidades que são necessárias para um estudo completo da dinâmica ionosférica como, por exemplo, medidas do fluxo de radiação solar, do campo magnético da Terra, da composição da atmosfera. 




Contudo, este trabalho foca em duas quantidades principais o VTEC e o índice S4, pois são as que fornecem mais informação à respeito do fenômeno de cintilação ionosférica, e as anomalias que a causam.


O índice S4 é utilizado para medir, avaliar, a cintilação ionosférica. Corresponde ao desvio padrão da intensidade do sinal de GPS de um minuto de dados, coletados com 50 amostras por segundo:

\begin{equation}
    S_4^2=\frac{\braket{I^2}-\braket{I}^2}{\braket{I}^2}\mbox{.}
\end{equation}

\chapter{IONOSFERA}

A ionosfera é uma região ionizada da alta atmosfera, estendendo-se de 60 até 1000 km de altitude, assim, englobando partes da mesosfera, termosfera, e exosfera. Esta camada constitui-se de íons e elétrons livres criados primariamente por processo de fotoionização, e gás neutro. A fotoionização ionosférica consiste de um processo físico-químico, onde alguma espécies químicas presentes na atmosfera ganham ou perdem elétrons decorrentes da absorção de radiação solar predominantemente nas faixas ultravioleta, extremo ultravioleta ultravioleta e raios-X \cite{RISNBETH:1969, NEGRETI:2012}. A ionização, também, pode ocorrer devido a colisões com partículas altamente energéticas, provindas do meio solar ou galácticas, o que é mais facilmente observado em altas latitudes, por fenômenos como auroral boreal.

A composição da ionosfera, assim, como a densidade do gases variam em função da altitude. Assim, a densidade de elétrons também varia, pois conforme a radiação penetra na atmosfera mais densa, a produção de elétrons, aumenta até atingir um valor de pico. Abaixo desta altitude, mesmo havendo um aumento na densidade da atmosfera neutra, a produção de elétrons decresce, pois a maior parte da radiação ionizante foi absorvida, e a taxa de recombinação predomina sobre a taxa de produção de elétrons. Devido as diferenças marcantes em termos de processos físicos e químicos que governam o comportamento daquela ionosfera, a mesma, pode ser dividida em camadas, onde em cada uma existe um processo predominante. Finalmente, devido as drásticas mudanças em quantidade de radiação absorvida devido a transição entre noite e dia, existirão camadas que não aparecem em cada instante ao longo de 24 horas.

Durante o período a camada F é a única que apresenta uma ionização significativa, enquanto as camadas E e D apresentam um valor extremamente baixo de ionização. Durante o dia, a camada D e E se tornam mais ionizadas, assim como a camada F, que se divide em duas regiões, F1 que é mais fracamente ionizada, e F2 que é mais intensamente ionizada. A camada F2 existe durante a noite e durante o dia, sendo a principal responsável pela reflexão e refração dos sinais de rádio.

A camada D é a mais interna, estando entre 60 e 90 km acima da superfície da Terra. Sua ionização é devido a radiação do hidrogênio ionizado na série de Lyman-alpha no comprimento de onda de 121.6 nm ionizando o óxido nítrico, $NO$, presente na camada. Além disso, raios X altamente energéticos, com comprimento de onda inferior de 1 nm podem ionizar as moléculas de $N_2$ e de $O_2$. A camada D apresenta apresenta a maior taxa de recombinação. Apresenta uma taxa de absorção considerável para ondas de radio de baixas e médias frequências, principalmente, devido a absorção de energia pelo elétrons livres, o que aumenta suas chances de colisão. Este efeito desaparece durante à noite, devido a uma menor ionização. Pode apresentar valores elevados de ionização em altas latitudes em decorrência de erupções solares com grandes quantidades de matéria hadrônica, prótons, em sua maioria, com uma duração de 24 à 48 horas.

A camada E é a intermediária e está situada entre 90 e 150 km acima da superfície da Terra. A ionização decorre principalmente devido ao espalhamento de raio-X leve (entre 1 e 10 nm) e ultravioleta distante (UV) provindos do Sol com moléculas de oxigênio. A estrutura vertical da camada E é determinada em sua maior parte pela competição entre efeitos de ionização e de recombinação. É importante pela presença de correntes elétricas que nela fluem e interagem com o campo magnético \cite{KIRCHHOFF:1991}. A noite, a camada E quase desaparece, pois sua fonte primaria de ionização não está presente.

A camada F se estende de 150 a mais de 500 km acima da superfície da Terra. Apresenta a maior concentração de elétrons, portanto, sinais que são capazes de penetrar até esta subcamada são capazes de escapar para o espaço. Predominam, nesta, a ionização de átomos de oxigênio por meio de radiação solar no espectro do extremo ultravioleta, entre, 10 e 100 nm. A camada é subdivida em duas regiões, a F2 que está presente durante o dia e a noite, e a F1 que aparece somente durante o dia. 

A subcamada F2 engloba toda a região superior da ionosfera, inclusive a região de pico da densidade de elétrons. Este máximo no perfil vertical de ionização decorre do balanço entre os processos de transporte de plasma e os processos físico-químicos. Acima deste pico, a ionosfera se encontra em equilíbrio difusivo, ou seja, o plasma se distribui com a sua própria escala de altura. A presença do campo magnético contribui para a distribuição da ionização.

\subsection{Termosfera}

A termosfera se inicia aproximadamente a 90 km de altitude e não apresenta um limite superior bem definido, mas que deve estar entre 500 e 600 km de altitude. Nesta surgem os fenômenos de aurora, e nela também estam localizadas as órbitas da estação espacial, do ônibus espacial, assim como de vários satélites. Assim, como todas as demais camadas da atmosfera, sua extensão varia conforme a latitude. A temperatura nesta camada podem alcançar valores superiores a 2000 graus Celsius.

\subsection{Magnetosfera}

\section{Anomalias na ionosfera}

A ionosfera apresenta várias anomalias, ou seja, várias irregularidades na distribuição de elétrons. Este trabalho tem interesse na anomalia equatorial. Esta aparece aproximadamente entre 15 e 20 graus de latitude magnética, tanto no hemisfério norte, quanto no hemisfério sul, na camada F2. Consiste na formação de uma região de alta densidade eletrônica, e é uma anomalia, pois a densidade de plasma deveria ser maior em regiões equatoriais, e não em latitudes magnéticas mais altas.

Sua origem decorre da deriva vertical do plasma da camada F na região equatorial: o processo de ionização da camada F faz surgir um campo elétrico, apontando para leste, enquanto o campo magnético aponta para o norte, considerando então $\vec{E}\times\vec{B}$, tem-se o surgimento de uma força perpendicular ao campo magnético e ao campo elétrico, o que neste caso, aponta para cima, deslocando o plasmas para regiões de mais alta altitudes. Agora, quando em altas altitudes, o plasma for efeito gravitacional e diferença de pressão é trazido de volta à altitudes mais baixas, porém este movimento de descida é mais eficiente ao longo das linhas de campo magnético, levando a um aumento na densidade de plasma em regiões de médias latitudes. 

A distribuição do plasma também pode ser alterada pela ação de outras variáveis, como o vento. O Vale do Paraíba, no estado de São Paulo, encontra-se na região da anomalia equatorial, mais especificamente no pico da anomalia, ou seja, na região onde a densidade de plasma, em altas altitudes, atinge seu valor máximo.

As anomalias que surgem na ionosfera apresentam um certo nível de organização devido às linhas de campo magnético da Terra. Isto ocorre pois partículas ionizadas ou carregas podem ser mover livremente ao longo das linhas magnéticas mas não entre elas. Assim, o estudo do campo magnético da Terra se faz relevante para um grande número de aplicações, \cite{LAUNDAL:2017}. Um resultado imediato deste estudo é que o norte geográfico e magnético não coincidem, e que simplificadamente o campo magnético poderia ser descrito por um dipolo magnético, com centro comum ao da Terra, porém inclinado em relação a linha que liga o norte e sul geográfico. Atualmente, existe vários sistemas de coordenadas magnéticas cujo proposito dependem da região, da aplicação e da faixa de altitude de interesse, para uma revisão entre os sistemas mais comuns consulte a referência \cite{LAUNDAL:2017}. Para este trabalho foi adotado o sistema AACGM, pois é mais adequado à altura ionosférica, contudo ele pertence a classe de sistemas não-ortogonais.

A cintilação ionosférica é uma variação rápida de amplitude e fase em sinais de radio frequência quando estes atravessam irregularidades no plasma ionosférico, como uma bolha de plasma, que é de particular interesse para este trabalho. Neste trabalho se realiza um estudo sobre algumas variáveis relacionadas com o efeito de cintilação ionosférica e com a estrutura de bolha no plasma.

\section{Algumas variáveis importantes para o estudo da cintilação ionosférica}

Existem várias quantidades que são necessárias para um estudo completo da dinâmica ionosférica como, por exemplo, medidas do fluxo de radiação solar, do campo magnético da Terra, da composição da atmosfera. Contudo, este trabalho foca em duas quantidades principais o VTEC e o índice S4, pois são as que fornecem mais informação à respeito do fenômeno de cintilação ionosférica, e as anomalias que a causam.

O Total Eletronic Content (TEC), ou conteúdo total de elétrons em português, é uma quantidade descritiva utilizada para avaliar a densidade de elétrons no plasma ionosférico. É o número total de elétrons integrado ao longo da trajetória entre um transmissor, no espaço, e um receptor, na Terra, em uma seção unitária \cite{HOFMANN:2013}. Por sua definição o TEC é uma quantidade que depende da trajetória, seu cálculo fica mais claro, ao dizer que é a integral de um densidade de elétrons dependente de posição ao longo de uma trajetória que atravessa a ionosfera:

\begin{equation}
    \mbox{TEC}=\int{n_{e}(s)}ds\mbox{,}~
\end{equation}

onde $ds$ especifica o elemento de integração na trajetória. Geralmente é reportada em unidades de TEC (TECU), definido por 1 $TECU=10^{16}$ elétrons/m$^2$. É importante para determinar a cintilação e os atrasos de fase e de grupo em ondas de rádio no meio. VTEC, ou conteúdo total de elétrons vertical é projeção do TEC, ao longo de uma linha normal a superfície da Terra, em outras palavras, ela fornece o conteúdo de elétrons ao longo da normal para cada ponto da superfície da Terra.

O índice S4 é utilizado para medir, avaliar, a cintilação ionosférica. Corresponde ao desvio padrão da intensidade do sinal de GPS de um minuto de dados, coletados com 50 amostras por segundo:

\begin{equation}
    S_4^2=\frac{\braket{I^2}-\braket{I}^2}{\braket{I}^2}\mbox{.}
\end{equation}

\section{Bolhas de Plasma}

As bolhas ionosféricas podem ser definidas como regiões de baixa densidade de plasma ionosférico quando comparadas com a sua vizinhança. Utilizando medidas de VTEC é possível definir essa diferença como 30-50 TECU \cite{TAKAHASHI:2006}.

São originadas na região equatorial, após a rápida elevação do plasma, devido a anomalia equatorial, isto é, o plasma ao acender cria regiões de baixa densidade. Após sua formação podem evoluir para altas altitudes (centenas de quilômetros), estendendo-se ao longo das linhas de campo magnético (milhares de quilômetros) nas direções norte-sul, alcançado em torno de 20 graus de latitude magnética.

\chapter{REZENDE}\label{ch:revisonrezende}

\section{Revisão}

O trabalho \cite{REZENDE:2009} foi pioneiro na utilização de modelos direcionados por dados para a exploração do problema de cintilação ionosférica, entretanto este trabalho não apresenta uma abordagem completamente reprodutível, visto ausência dos códigos e da base de dados. O objetivo desta revisão é entender as dificuldades deste trabalho, assim como sua incompletude em relação a proposta a ser estabelecida nessa dissertação.

\subsection{Principais Pontos}

A previsão de cintilação de curto prazo (uma hora) foi realizada utilizando {\bf amostras com intervalos de 5 minutos}, entretanto alguns dos atributos, como já discutido, apresentam resolução inferior, sendo portanto interpolados, ou copiados de sua vizinhança, enquanto outros apresentam resolução mais altas, e neste caso precisaram ser agrupados, segundo algum critério, por exemplo, por uma operação de máximo.

Os atributos foram:

\begin{itemize}
\item {\bf Hm\_Eq} representa a hora no equador (em São Luis), em intervalos de 5 minutos;
\item {\bf Vel\_Der} é a velocidade máxima de deriva vertial do plasma medida no equador entre as 17LT e 19LT (20UT e 22UT), com resolução de um valor por dia;
\item {\bf Kp} é a média do índice Kp definido pela expressão:
\begin{equation}
\sum_{i=1}^{n}\sum_{j=1}^{i}\frac{Kp_{j}}{ni}\mbox{,}~
\end{equation}
onde $Kp_1$ corresponde ao Kp medido entre 14-17LT, $Kp_2$ entre 11-14LT, até o valor medido entre 5-8LT, o valor de $n$ é 4. Este valor apresenta resolução diária;
\item {\bf F10.7} é o fluxo solar;
\item {\bf S4\_Eq} é o maior valor do índice S4 medido no equador, em um período de 5 minutos;
\item {\bf S4\_PA\_tempo\_real} é o maior valor do índice S4 medido no pico da anomalia (São José dos Campos), em um período de 5 minutos;
\item {\bf S4\_PA} é o S4 estimado com uma hora de antecedência para o pico da anomalia, com resolução de 5 minutos.
\end{itemize}

As primeiras 5 variáveis correspondem aos atributos preditores, enquanto a variável S4\_PA corresponde ao atributo reposta. Os atributos acima passam por algum pré-processamento antes e depois da seleção das amostras:

\begin{itemize}
\item {\bf S4}, antes da seleção, somente são utilizados os valores medidos para satélites com ângulo superior a 30 graus;
\item {\bf S4}, depois da seleção, a grande variabilidade destes dados leva a adoção de um filtro passa baixa, realizado por meio de uma suavização com média móvel, com 15 pontos;
\item {\bf Kp}, depois da seleção, somente mantidos amostras com valores de Kp inferiores a 3, pois valores maiores que este caracterizam forte pertubação magnética, associadas a eventos extremos como tempestades magnéticas, e nesta configuração a predição se torna inviável;
\item {\bf Dados}, depois da seleção, devem compreender o período  entre as 18-23LT (21-01UT) de forma a predizer os valores no intervalo 19-24LT (22-02UT).
\end{itemize}

Ao final, a base de dados deste trabalho apresentava um total de 80 dias de dados com 4680 amostras, coletadas entre 2000 e 2002. Este trabalho também realizou testes com predições com 1 dia de antecedência, entretanto não apresentaramm um resultado tão significativo e, portanto, não serão discutidos.

Restam definir dois elementos para que o problema fica completamente definido, as métricas, e os modelos. Uma vez que as métricas ficam restringidas segundo os modelos, ir-se-á estabelecer estes primeiros:

\begin{itemize}
\item agrupamento por expectation-maximization implementado no ambiente Weka;
\item regras de associação utilizando o algoritmo apriori, onde neste caso os atributos foram discretizados;
\item regressão, utilizando árvores CART com a estratégia de ensemble bagging, implementadas pelo autor, análogo à Floresta Randômica.
\end{itemize}

As métricas foram erro quadrático médio, e índice de correlação de Pearson para o problema de regressão, e inspeção para as demais. A aplicação de agrupamento permitiu concluir que se tratava de um problema altamente não linear, as regras de associação geram conclusões que já eram bem conhecidas da literatura do problema na área de aeronomia. Finalmente, os resultados mais interessantes foram estabelecidos pelo problema de regressão, com erro quadrático médio de 0.05 com correlação de Pearson de 0.985.

O trabalho concluí se apresentando como uma abordagem inédita para o problema da predição da cintilação ionosférica.

\subsection{Análise}

A análise consistiu em levantar e sintetizar alguns pontos que levam a uma definição parcialmente incompleta do problema, ou talvez errônea do problema:

\begin{enumerate}
\item uma vez que o código e a base de dados não é disponibilizada de maneira pública e nem devidamente documentada, a reprodução somente é possível em partes, visto que serão utilizados algoritmos que muito se assemelham, mas que devido a diferença devem levar a resultados diferentes;
\item do ponto de vista de implementação não foi definido uma representação para a variável {\bf Hm\_Eq} que pode ser então representada em segundos, minutos, entre outras opções, observar entretanto que isto não deveria levar a diferença significativas no resultado final;
\item a variável Kp é medida por padrão nos intervalos 00-03UT, 03-06UT, 06-09UT, 09-12UT, 12-15UT, 15-18UT, 18-21UT e 21-24UT, e não nos intervalos utilizados por, sendo portanto necessário definir como é feito este mapeamento, o que não está presente no texto;
\item a definição de como a média móvel é aplicada na quantidade S4 está um tanto incompleta, isto é, dado o i-ésimo elemento de um vetor, com um tamanho de janela de 15 pontos, ela poderia ser aplicada levando-se em consideração: os 14 pontos anteriores, $\{i-14, i-13, ..., i-1, i\}$; ou os 7 pontos anteriores e 7 posteriores, $\{i-7,...,i-1,i,i+1,...,i+7\}$, denominada de forma central; entre outras combinações;
\item levando em consideração que a forma centrada tenha sido adotada, a janela de 15 pontos exigirá que 7 pontos do futuro sejam conhecidos, neste caso, dado um instante $t$ seriam necessários 35 minutos de dados a frete deste para o cálculo da média e,  portanto, na verdade, o resultado não seria previsto com uma hora de antecedência, mas sim 25 minutos;
\item a métrica, erro quadrático médio, pode não contemplar o problema. Para entender tal proposição, considere a adaptação deste problema para uma classificação, ficará evidente que eventos com altos valores de cintilação são mais raros e, portanto, ter-se iá, um problema de classificação não balanceado. Retornando, a regressão, pode-se ocorrer do modelo predizer muito bem valores baixos, que então irão mascarar os efeitos de erros em valores mais altos, visto que irão predominar no processo de cálculo de média.
\end{enumerate}

\section{Reprodução}

\subsection{Original}

O trabalho \cite{REZENDE:2009} foi parcialmente, reproduzido no contexto desta proposta, pois somente uma linha de pesquisa do original foi explorado, o problema de regressão, e cuidados adicionais foram necessários, devido a utilização de mais anos. As variáveis adotadas são:

\begin{itemize}
\item {\bf ut} representa a hora em São Luiz em minutos, em intervalos de 5 minutos;
\item {\bf vhf} é a velocidade máxima de deriva vertical do plasma medida em São Luiz entre as 17LT e 18LT (20UT e 21UT), com resolução de um valor por dia;
\item {\bf ap} é a média do índice ap definido por:
\begin{equation}
\sum_{i=1}^{n}\sum_{j=1}^{i}\frac{ap_{j}}{ni}\mbox{,}~
\end{equation}
onde $ap_1$ corresponde à ap15\_18ut, $ap_2$ à ap12\_15ut até ap00\_03ut, mais ap21\_00ut e ap18\_21ut do dia anterior, totalizando um intervalo de 24 horas, com $n=8$. Este valor apresenta resolução diária;,
\item {\bf F10.7} é o fluxo solar;
\item {\bf s4\_sl} é o maior valor do índice S4 medido em São Luiz, em um período de 5 minutos;
\item {\bf s4\_sj} é o maior valor do índice S4 medido em São José dos Campos, em um período de 5 minutos;
\item {\bf s4\_sj\_shift\_1h} é o S4 estimado com uma hora de antecedência para o pico da anomalia, com resolução de 5 minutos.
\end{itemize}

E os cuidados, assim, como os pré-processamentos foram:

\begin{itemize}
\item A adoção do termo São Luis em relação a equador magnético foi preferida, pois esse esta em movimento, e ao longo do período coletado, assim como extensões deste período ele não estará no mesmo lugar, enquanto os dados são sempre coletados em uma estação fixa em São Luiz;
\item A adoção do termo São José dos Campos em relação a pico da anomalia ocorre, pois a localização do pico depende da quantidade de radiação emitida pelo Sol em seu regime, ciclo solar. Nos anos de 2000, 2001, 2002, o pico da anomalia estava em São José dos Campos, porém nos anos de 2018 e 2019 se encontra em Presidente Prudente. Finalmente, os dados foram sempre coletados na estação em São José dos Campos;
\item A altura hF não é amostrada em intervalos regulares ao longo do período de dados coletados, inicialmente, ela era amostrada em 15 min, e posteriormente passou a ser amostrado em 10 min, portanto neste trabalho se reamostrou toda a série para o intervalo de 10 min, que foi então interpolado por um spline de grau 3, até um máximo de x pontos ausentes na vizinhança do ponto a ser interpolado;
\item Quanto a variável S4, primeiro, somente serão aceitas medidas cuja elevação entre a estação e o receptor sejam maiores que 30 graus; segundo, os dados de cintilação apresentam resolução temporal de 1 min, e são coletados para cada satélite acima do plano de horizonte da estação, portanto, existem vários dados por minuto, com objetivo de ficar-se somente com um dado, estes foram agrupados tomando-se o maior valor de cintilação; os dados com resolução de um minuto são interpolados por spline de ordem 3 com limite de quatro valores ausentes; quarto, os dados são suavizados por um filtro de Savitz-Goley de ordem 3 com janela de tamanho 5, o que levaria a necessidade de apenas, 2 amostras de dados futuros; finalmente, os dados são reamostrados para um intervalo de 5 minutos;
\item A adoção de $ap$ em preferência a $kp$ se deve a esta apresentar uma escala linear e, portanto ser mais condizente com operações como média.
\end{itemize}

Uma abordagem de normalização para o intervalo $[0,05,\,0,95]$ é aplicado a todos os atributos preditores (uma para cada atributo) antes da utilização do algoritmo de aprendizagem de máquina. Este intervalo é adotado já prevendo a possibilidade de aplicação de redes neurais com função de ativação sigmoide, haja visto que esta sofre saturação para valores próximos de 0 e de 1.

Resta definir dois elementos para a serem definidos, o tipo de problema tratado e por consequência os algoritmos e as métricas a serem utilizados. Uma que vez que se trata de uma reprodução parcial, o problema de regressão foi tratado, neste caso utilizando a ferramente XGBOOST, que também consiste da utilização de árvores de decisão e regressão com algoritmos de ensemble, neste caso o boosting; as métricas adotadas foram o erro quadrático médio e o erro absoluto máximo, este último sendo capaz de lidar melhor com o desbalanceamento das amostras.

O segundo problema tratado foi de classificação, e para tal a variável {\bf s4\_sj\_shift\_1h} foi discretizada utilizando a proposta estabelecida por \cite{MUELLA:2008}, mais a adição de uma classe ausente:

\begin{table}
\begin{center}
\begin{tabular}{|c|c|}
\hline
{\bf INTENSIDADE} & {\bf $S_4$} \\ \hline
Saturado          & $S_4 > 1,0$ \\ \hline
Forte             & $0,6 \le S_4 \le 1,0$ \\ \hline
Moderado          & $0,4 \le S_4 \le 0,6$ \\ \hline
Fraco             & $0,2 \le S_4 \le 0,4$ \\ \hline
Ausente           & $ S_4 \le 0,2 $ \\ \hline
\end{tabular}
\end{center}
\end{table}

Neste caso, tem-se um problema com 5 classes como já mencionado não balanceado e uma abordagem de reasmotragem foi empregada de modo balancear o números de elementos tal que todos estejam próximos da cardinalidade da classe com o maior número de elementos. Esta etapa é realizada após a normalização e o algoritmo empregado foi o ADASYN. Finalmente, adotou-se como métrica a precisão balanceada, e a ferramenta utilizada também foi o XGBOOST.

\subsection{Resultados}



%% insira quantos capítulos desejar com o seguinte comando:
%\include{_pasta_do_arquivo_/_meu_arquivo_} %%sem a extensão
%% note que deverá haver um arquivo _meu_arquivo_.tex (com extensão) no diretório _pasta_do_arquivo_

%\include{./docs/conclusao}

%% Bibliografia %% não alterar %% obrigatório %testebib
\bibliography{./bib/referencia} %% aponte para seu arquivo de bibliografia no formato bibtex (p.ex: referencia.bib)


%\include{./docs/glossario} %% insira os termos do glossário no arquivo glossario.tex %% opcional

\inicioApendice %% opcional, comente esta linha e a seguintes se não houver apendice(s)
%%%%%%%%%%%%%%%%%%%%%%%%%%%%%%%%%%%%%%%%%%%%%%%%%%%%%%%
%Apêndice A
\hypertarget{estilo:apendice1}{} %% uso para este Guia
%Este apêndice foi criado apenas para indicar como construir um apêndice no estilo, não existia no original da tese.
%%%%%%%%%%%%%%%%%%%%%%%%%%%%%%%%%%%%%%%%%%%%%%%%%%%%%%
\renewcommand{\thechapter}{}%
\chapter{APÊNDICE A - AUTORIZAÇÃO PARA PUBLICAÇÃO}	% trocar A por B na próxima apêndice e etc
\label{apendiceA}	% trocar A por B na próxima apêndice e etc
\renewcommand{\thechapter}{A}%		% trocar A por B na próxima apêndice e etc

Há dois formulários de autorização para publicação, um para publicações de trabalhos acadêmicos e outro para publicações técnico-científicas, neste apêndice encontram-se os modelos dos formulários e suas respectivas instruções de preenchimento. 

\section{Autorização para Publicação de Trabalho Acadêmico - INPE-393}

\label{instr393}

	\begin{figure}[ht]
		\caption{Formulário Autorização para Publicação de Trabalho Acadêmico INPE-393.}
		\vspace{6mm}	% acrescentar o espaçamento vertical apropriado entre o título e a borda superior da figura
		\centering
   		\includegraphics[height=16cm]{./Figuras/form393.png}	   
 		\label{form393}
	\end{figure}


\subsection{Instruções do Formulário INPE-393} 

\begin{enumerate} 

 \item \textbf{série:} com este número o SID identifica as publicações do INPE, composto da sigla da Instituição, número sequencial geral da publicação, sigla e número sequencial do tipo de publicação, exemplo: INPE-14209-TDI/1110;
 
 \item \textbf{número:} será composto da sigla da unidade do SID, mais 4 (quatro) dígitos e do ano em curso. Este número de referência é de controle da unidade emissora. Ex.: SID-0001/2007;

 \item \textbf{título da publicação:} deve ser completo, evitando-se abreviar palavras;

 \item \textbf{nome do autor e do orientador:} estes campos devem ser preenchidos por extenso, da mesma forma em que irão constar da publicação;

 \item \textbf{origem da publicação:} sigla da unidade do servidor (autor da publicação), conforme TQ-001;

 \item \textbf{curso:} sigla do curso, de acordo com a Estrutura de Divisão de Trabalho - EDT do INPE;
 
 \item \textbf{tipo:} assinalar se é tese ou dissertação;

 \item \textbf{apresentação:} colocar a data de aprovação final;

 \item \textbf{revisão técnica:} o responsável designado pela Banca Examinadora para verificação de correções e, na ausência desse, o orientador da tese ou dissertação deve
carimbar, datar e assinar após a versão \emph{on line} do trabalho;

 \item \textbf{revisão de linguagem:} o responsável designado pela Banca Examinadora para verificação de correções, e na ausência deste o orientador deve assinalar a solicitação ou a dispensa da revisão de linguagem e, carimbar, datar e assinar; o revisor deve datar e assinar após a revisão;
 
 \item \textbf{distribuição:} O SID deve informar a quantidade de CD's e de cópias impressas da tese ou dissertação, conforme lista de distribuição;
 
 \item \textbf{verificação de normalização:}  Após a verificação da versão \emph{on line} do trabalho quanto às normas editoriais, o SID deve datar e assinar;
 
 \item \textbf{autorização final:} data e assinatura do Titular de Nível A, conforme TQ-001, a que o Serviço de Pós-Graduação estiver subordinado.
 
 \item \textbf{observações:} para outras informações necessárias. 

\end{enumerate}

\section{Autorização para Publicação - INPE-106}
\begin{figure}[ht!]
	\caption{Formulário Autorização para Publicação de Trabalho Acadêmico INPE-106 folha 1.} 
	\vspace{6mm}	% acrescentar o espaçamento vertical apropriado entre o título e a borda superior da figura
	\centering
	\includegraphics[height=18cm]{./Figuras/form106.png}
	\label{form106}
\end{figure}

\begin{figure}[ht!]
	\caption{Formulário Autorização para Publicação de Trabalho Acadêmico INPE-106 folha 2.} 
	\vspace{6mm}	% acrescentar o espaçamento vertical apropriado entre o título e a borda superior da figura
	\centering
	\includegraphics[height=18cm]{./Figuras/form106folha2.png}
	\label{form106a}
\end{figure}

\clearpage
\subsection{Instruções do Formulário INPE-106} 
\label{instr106}


\begin{enumerate}

 \item \textbf{série:} com este número o SID identifica as publicações do INPE, composto da sigla da Instituição, número sequencial geral da publicação, sigla e número sequencial do tipo de publicação, exemplo: INPE-5616-RPQ/671. 
 
 \item \textbf{número:} será composto da sigla da unidade constante da Estrutura Organizacional do INPE (TQ-001), mais 4 (quatro) dígitos e do ano em curso. Este número de referência é de controle da unidade solicitante. Ex: CEA-0001/2007;
 
 \item \textbf{título da publicação:} deve ser completo, evitando-se abreviar palavras;

 \item \textbf{nome do autor, tradutor e editor:}  estes campos devem ser preenchidos por extenso, da mesma forma em que irão constar da publicação;

 \item \textbf{origem da publicação:} sigla da unidade do servidor (autor da publicação), conforme TQ-001;

 \item \textbf{projeto:} sigla do projeto de acordo com a Estrutura de Divisão de Trabalho - EDT do INPE;

 \item \textbf{tipo de publicação:} assinalar o tipo de publicação proposta:

 \begin{enumerate}
  \item{Relátorio de Pesquisa (RPQ)},
  \item{Notas Técnico-Científicas (NTC)},
  \item{Propostas e Relatórios de de Projeto (PRP)},
  \item{Manuais Técnicos (MAN)},
  \item{Publicações Didáticas (PUD)},
  \item{Trabalhos Acadêmicos Externos (TAE)}.
 \end{enumerate}

 \item \textbf{divulgação:} assinalar, de acordo com os critérios de classificação. Se houver Lista de Divulgação, nesta deverá constar os nomes e endereços completos;

 \item \textbf{convênio:} descrever o nome da instituição, quando a publicação for realizada pelo INPE e outra organização, preencher somente para o tipo PRP; 
 
    \item \textbf{autorização preliminar:} data, carimbo e assinatura do Titular da Unidade a que o autor esteja subordinado e, assinatura do revisor que efetuou a revisão técnica aprovando a versão \emph{on line} do trabalho e do revisor que realizou a revisão de linguagem, quando solicitadas; 
    
  \item \textbf{verificação de normalização:} o SID deve datar e assinar após a revisão da adequação às normas editoriais;   
  
  \item \textbf{distribuição:} O SID deve informar a quantidade de CD's e de cópias impressas que deverão ser gravados conforme lista de distribuição;
  
 \item \textbf{autorização final:} data, carimbo e assinatura do Titular de Nível "A", conforme TQ-001, a que o autor da publicação estiver subordinado;
 
 \item \textbf{observações:} para outras informações necessárias, inclusive para descrever as justificativas de uma publicação.
\end{enumerate} %% insira apendices tal qual capítulos acima


\inicioAnexo
%%%%%%%%%%%%%%%%%%%%%%%%%%%%%%%%%%%%%%%%%%%%%%%%%%%%%%%
%Anexo
%Este anexo foi incluido para explicar como incluir um anexo no estilo, não existia no original desta tese.
%%%%%%%%%%%%%%%%%%%%%%%%%%%%%%%%%%%%%%%%%%%%%%%%%%%%%%%%%%%%%%%%%%%%%%%%%%%%%%%%%
\renewcommand{\thechapter}{}%
\chapter{ANEXO A - ABREVIATURA DOS MESES} %% Título do anexo sempre em maiúsculas. Trocar A por B no próximo anexo e etc
\label{anexoA} %% Rótulo aplicado caso queira referir-se a este tópico em qualquer lugar do texto. Trocar A por B no próximo anexo e etc
\renewcommand{\thechapter}{A}%		% trocar A por B no próximo anexo e etc

\begin{table}[!ht]
 \label{tab:abreviaturas}
  \begin{center}
 	\begin{tabular}{lll}
	 \hline
	  \textbf{Português}    & \textbf{Espanhol}  & \textbf{Italiano}\\ 
   \hline
       janeiro   = jan.   & enero = ene.       & gennaio = gen.\\
       fevereiro = fev.   & febrero = feb.     & febbraio = feb.\\
       março     = mar.   & marzo = mar.       & marzo = mar.\\
       abril     = abr.   & abril = abr.       & aprile = apr.\\
       maio      = maio   & mayo = mayo        & maggio = mag.\\ 
       junho     = jun.   & junio = jun.       & giugno = giu.\\ 
       julho     = jul.   & julio = jul.       & luglio = lug.\\
       agosto    = ago.   & agosto = ago.      & agosto = ago.\\
       setembro  = set.   & septiembre = sep.  & settembre = set.\\
       outubro   = out.   & octubre = oct.     & ottobre = ott.\\
       novembro  = nov.   & noviembre =nov.    & novembre = nov.\\
       dezembro  = dez.   & diciembre = dic.   & dicembre = dic.\\ 
     \hline
   \textbf{Francês}       & \textbf{Inglês}    & \textbf{Alemão}\\
     \hline
       janvier = jan.     & January = Jan.     & Januar = Jan.\\
       février = fév.     & February = Feb.    & Februar = Feb.\\
       mars = mars        & March = Mar.       & März = März\\
       avril = avr.       & April = Apr.       & April = Apr.\\
       mai = mai          & May = May          & Mai = Mai.\\
       juin = juin        & June = June        & Juni = Juni\\
       juillet = juil.    & July = July        & Juli = Juli\\
       août = août        & August = Aug.      & August = Aug.\\
       septembre = sept.  & September = Sept.  & September = Sept.\\
       octobre = oct.     & October = Oct.     & Oktober = Okt.\\
       novembre = nov.    & November = Nov.    & November = Nov.\\
       décembre = déc.    & December = Dec.    & Dezember = Dez. \\ 
    \hline
   \end{tabular}
   \end{center}
	 \FONTE{Adaptada de \citeonline[p.~22]{NBR6023:2002b}.}
\end{table}

\inicioIndice
%%%%%%%%%%%%%%%%%%%%%%%%%%%%%%%%%%%%%%%%%%%%%%%%%%%%%%%
%Contracapa
%%%%%%%%%%%%%%%%%%%%%%%%%%%%%%%%%%%%%%%%%%%%%%%%%%%%%%

\thispagestyle{empty}
 \begin{table}
  \begin{center}
  \begin{tabularx}{\textwidth}{X}
   \textbf{PUBLICAÇÕES TÉCNICO-CIENTÍFICAS EDITADAS PELO INPE}
  \end{tabularx} 
  \end{center}
 \end{table}
  
 \begin{table}
  \begin{center}
  \begin{tabularx}{\textwidth}{X X}
      
  \textbf{Teses e Dissertações (TDI)}              & \textbf{Manuais Técnicos (MAN)}\\
\\
Teses e Dissertações apresentadas nos Cursos de Pós-Graduação do INPE.	&
São publicações de caráter técnico que incluem normas, procedimentos, instruções e orientações.\\
\\
\textbf{Notas Técnico-Científicas (NTC)}           & \textbf{Relatórios de Pesquisa (RPQ)}\\
\\
Incluem resultados preliminares de pesquisa, descrição de equipamentos, descrição e ou documentação de programas de computador, descrição de sistemas e experimentos, apresentação de testes, dados, atlas, e documentação de projetos de engenharia. 
&	
Reportam resultados ou progressos de pesquisas tanto de natureza técnica quanto científica, cujo nível seja compatível com o de uma publicação em periódico nacional ou internacional.\\
\\
\textbf{Propostas e Relatórios de Projetos (PRP)}	& \textbf{Publicações Didáticas (PUD)} 
\\
\\
São propostas de projetos técnico-científicos e relatórios de acompanhamento de projetos, atividades e convênios.
&	
Incluem apostilas, notas de aula e manuais didáticos. \\
\\         
\textbf{Publicações Seriadas} 	& \textbf{Programas de Computador (PDC)}\\
\\
São os seriados técnico-científicos: boletins, periódicos, anuários e anais de eventos (simpósios e congressos). Constam destas publicações o Internacional Standard Serial Number (ISSN), que é um código único e definitivo para identificação de títulos de seriados. 
&	
São a seqüência de instruções ou códigos, expressos em uma linguagem de programação compilada ou interpretada, a ser executada por um computador para alcançar um determinado objetivo. Aceitam-se tanto programas fonte quanto os executáveis.\\
\\
\textbf{Pré-publicações (PRE)} \\
\\
Todos os artigos publicados em  periódicos, anais e como capítulos de livros. \\                 \end{tabularx}
  \end{center}
 \end{table}


\end{document}
